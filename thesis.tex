\documentclass[phd,icsa,twoside,logo,11pt]{infthesis}

\usepackage{amsfonts}
\usepackage{amsmath}
\usepackage{amsthm}
\usepackage[numbers]{natbib}
\usepackage[hidelinks]{hyperref}
\usepackage{cleveref}
\usepackage{graphicx}
\usepackage{listings}
\usepackage{xcolor}
\usepackage{calligra}
\usepackage[T1]{fontenc}
\usepackage[utf8]{inputenc}
\usepackage{tcolorbox}
\tcbuselibrary{theorems}
\usepackage{mathptmx}
\usepackage{enumitem}
\usepackage{pgfplots}
\usepackage{tabularx}
\usepackage{courier}
\usepackage{booktabs}
\usepackage{multirow}
\usepackage{rotating}
\usepackage{algorithm}
\usepackage[noend]{algpseudocode}

\usetikzlibrary{patterns}
\pgfplotsset{compat=1.13}

\newcolumntype{P}[1]{>{\centering\arraybackslash}p{#1}}

\hyphenpenalty=100000

\makeatletter
\newcommand{\tcb@cnt@definitionautorefname}{Definition}
\makeatother

\makeatletter
\newcommand{\tcb@cnt@notationautorefname}{Notation}
\makeatother

\newtcbtheorem[auto counter,number within=chapter]{definition}{Definition}{
label type=definition,
colback=white,
colframe=black,
fonttitle=\bfseries,
title={Definition \thetcbcounter}}{def}

\newtcbtheorem[auto counter,number within=chapter]{notation}{Notation}{
label type=notation,
colback=white,
colframe=black,
fonttitle=\bfseries,
title={Definition \thetcbcounter}}{not}

\newtcolorbox{blackbox}[1]{
top=0mm,
bottom=0mm,
left=0mm,
right=0mm,
boxsep=0mm,
toptitle=1mm,
bottomtitle=1mm,
colback=black,
colframe=black,
fonttitle=\centering\bfseries,
title=#1}

\newtcolorbox{graybox}[0]{
top=0mm,
bottom=0mm,
left=0mm,
right=0mm,
colback=white!90!black,
colbacktitle=white!90!black,
colframe=white!50!black,
lower separated=false
}

\definecolor{sh_comment}{rgb}{0.65, 0.00, 0.00 }
\definecolor{sh_keyword}{rgb}{0.15, 0.25, 0.25}
\definecolor{sh_string}{rgb}{0.08, 0.69, 0.08}

\lstset{
basicstyle         = \linespread{1}\ttfamily,
stringstyle        = \color{sh_string},
keywordstyle       = \color{sh_keyword}\bfseries,
commentstyle       = \color{sh_comment}\itshape,
numbers            = left,
numberstyle        = \scriptsize,
numbersep          = 10pt,
xleftmargin        = 1.5em,
frame              = lines,
framexleftmargin   = 1.5em,
framexbottommargin = 0em,
escapeinside       = {([}{])}
}

\lstdefinelanguage{CAnDL}
{
morekeywords = { Constraint, with, opcode, collect, include, function\_name,
                 data\_type, ir\_type, domination, strict\_domination,
                 calculated\_from, control\_origin, data\_origin, End },
morecomment  = [s]{\{}{\}}
}

\lstdefinelanguage{constraints}
{
morekeywords = { Constraint, with, and, or, at, as, End },
morecomment  = [s]{\{}{\}}
}

\lstdefinelanguage{LiLAC}
{
morekeywords = { HARNESS, IMPLEMENTS, Marshalling, PersistentVariables,
                 BeforeFirstExecution, AfterLastExecution, CppHeaderFiles,
                 INPUT, OUTPUT, COMPUTATION},
morekeywords = [2]{ CudaRead, CudaWrite, spmv_csr, Maximum, cuda, mkl,
                    dotproduct},
morekeywords = [3]{ out, in, size },
morecomment  = [s]{"}{"}
}

\title{Compiler Analysis as Constraint Programming\\
       for Idiomatic Heterogeneous Code Acceleration}
\author{Philip Ginsbach}

\abstract{
    The end of Moores Law and the end of Dennard Scaling have required new
    approaches in hardware.
    In response, accelerator processors have become widespread, first in the
    form of general purpose graphics processing units (GPGPU) and now
    increasingly with even more deep learning accelerators.

    Heterogeneous computing platforms is the natural and necessary reaction to
    the scaling limitations of general purpose processors, but they pose a huge
    challenge to the surrounding software ecosystem.
    As opposed to microarchitectural improvements, existing software is not able
    to automatically profit from new acccelerators designs.

    A new hardware software contract is needed instead and heterogeneous
    hardware should become a responsibility of the compiler.
    While compilers have lagged behind the developments in the hardware domain,
    research into this area can profit from experience with multi-core
    processors.

    Auto-parallelizing compilers have failed to solve the problems and have only
    had major success for specific kernels and using auto-tuning.
    Heterogeneous computing is a superset of parallel computing and so an
    approach to fully automatically optimize code is unlikely.

    At the same time, a library and DSL based approach has been successful,
    however fails to become mainstream due to adoption cost.
    What is promising therefore is a combination of hand-optimized libraries
    together with compiler automatisms.

    This requires a disruptive improvement in compiler analysis capabilites.
    The detection of higher level algorithmic structures in compilers has been
    investigated before, but established approaches cannot scale to what
    this challenge requires.
    Syntactic matching on programming languages has become
    unviable for the complexities of both modern programming languages and
    complex code bases.

    Instead, this thesis develops an entirely novel approach based on concepts
    from constraint solving, building a pragmatic methodology for detecting
    complex algorithmic structures -- computational idioms.
}

\begin{document}
\begin{preliminary}
\maketitle
\begin{acknowledgements}

\end{acknowledgements}
\begin{declaration}
    I declare that this thesis was composed by myself, that the work contained
    herein is my own except where explicitly stated otherwise in the text, and
    that this work has not been submitted for any other degree or professional
    qualification except as specified.
    Some of the material used in this thesis has been published in the following
    papers:

    \begin{itemize}
    \small
        \item Philip Ginsbach and Michael F.\ P.\ O'Boyle.\\
              {\bf Discovery and Exploitation of General Reductions: A
              Constraint Based Approach}.\\
              {\em Proceedings of the 15th Annual International
               Symposium on Code Generation and\\Optimization (CGO), 2017}
        \item Philip Ginsbach, Lewis Crawford and Michael F.\ P.\ O'Boyle.\\
              {\bf CAnDL: A Domain Specific Language for Compiler Analysis}.\\
              {\em Proceedings of the 27th International Conference on
               Compiler Construction (CC), 2018}
        \item Philip Ginsbach, Toomas Remmelg, Michel Steuwer, Bruno Bodin,
              Christophe Dubach and\\Michael F.\ P.\ O'Boyle.\\
              {\bf Automatic Matching of Legacy Code to Heterogeneous APIs: An
              Idiomatic Approach}.\\
              {\em Proceedings of the 23rd International Conference on
               Architectural Support for\\Programming Languages and Operating
               Systems (ASPLOS), 2018}
        \item Philip Ginsbach, Bruce Collie and Michael F.\ P.\ O'Boyle.\\
              {\bf Automatically Harnessing Sparse Acceleration Libraries}.\\
              In: {\em ???, 2019}
    \end{itemize}

    \par
\vspace{1in}\raggedleft({\em Philip Ginsbach})
\end{declaration}
\tableofcontents
\end{preliminary}
\chapter{Introduction}
    \label{chapter:introduction}
        For several decades, from the 1970s until the early 2000s, Moore's Law and
    Dennard Scaling accurately described the progress of processor development.
    The circuit density approximately doubled every two years, yet the power
    consumption and the accompanying thermal limitations remained relatively
    stable.
    This had important implications also for software development.
    Not only did the performance of computers improve exponentially, but these
    performance gains were also immediately available to already existing
    programs.
    The most fundamental contracts between software and hardware - the
    instruction set architectures of processors - evolved without paradigmatic
    changes.
    Software developers and users could therefore trust on ever increasing
    performance from hardware progress alone, without any intervention.
    This is best exemplified by the pervasive x86 instruction set architecture,
    which evolved without breaking backward compatibility from its original
    inception in 1978 and still dominates mainstream processors.

    This seemingly unending progress started breaking down around 2006,
    with the apparent end of Dennard Scaling.
    While the shrinking of transistors continued, this did not coincide with
    increased clock cycles and stable power consumption any longer.
    The resulting paradigmatic shift toward multi-processing left a deep mark on
    software development.
    New programming paradigms and languages, annotation systems, programming
    interfaces, libraries and compiler techniques were developed and continue to
    evolve.
    Despite immenense progress in the field, parallel computing remains an area
    of active research and automatic approaches often fell short.

    In recent years, it has become apparent that multi-processing as well has
    reached its limits, and another paradigm shift is following.
    With the end of both Dennard Scaling and Moore's Law, the hardware industry
    turns toward architectural innovation, trending away from general purpose
    processors and toward the use of dark silicon and specialised hardware.
    Accelerator processors have become widespread, first in the form of general
    purpose graphics processing units (GPGPU) and now increasingly with deep
    learning accelerators, which coincide with the meteoric rise of
    Convolutional Neural Networks and the enormous success of Deep Learning.

    Heterogeneous computing is a natural and necessary reaction to the scaling
    limitations of general purpose processors, but it poses an enormous
    challenge to the surrounding software ecosystem, and it puts into question
    many of the achievements in portability and longevity of programs that are
    taken for granted in modern computing.
    Existing software is not able to automatically benefit from entirely new
    acccelerator designs in the way that it profited from continuous
    microarchitectural improvements.
    Where previously, programs performed better on each succeeding hardware
    generation, with at most a recompilation required, new accelerators arrive
    with entirely novel and incompatible application interfaces.

    In particular, this novel hardware landscape greatly diminishes the
    scope of responsibilities and impact that traditional compilers for
    languages like C/C++ and Fortran used to have.
    Were previously, such compilers were responsible for orchestrating
    program execution on the entirety of available computing resources, they are
    increasingly downgraded to merely coordinating the launch of core workloads
    in the form of computational kernels, which are executed as separate and
    opaque programs on heterogeneous accelerators.
    In order to reclaim their central position in the software ecosystem,
    compilers require some fundamentally new approaches.

    An important distinction has to be made between {\em host compilers} and
    {\em kernel compilers} at this point, as these have developed quite
    distinctly.
    In fact, many of the kernel programs that remain opaque to the host compiler
    are themselved products of specialized kernel compilers.
    Kernel compilers successfully apply many of the techniques that have had
    only limited success for host compilers.
    Most importantly, they reason automatically about parallelism, understand
    data dependencies at a deeper level and they incorporate iterative
    compilation.
    This is made possible by a combination of several factors that apply
    uniquely to kernel compilers:
    Smaller programs allow for more expensive compilation techniques;
    more restrictive languages and intermediate representaions allow for
    stronger reasoning;
    and domain knowledge from areas such as image processing can be directly
    embedded in custom compiler technology, without the need to genrealize to
    fully generic programs.
    Furthermore, kernel compilers often run in relatively controlled
    environments, with less need for predictability, reproducability and
    stability.
    The scope of input programs might even be small enough to leave compilation
    to experts and vendors, which then ship the results in the form of already
    compiled library functions.

    As host compilers do not operate under these conditions, it is unsurprising
    that they have lagged behind these developments.
    Despite immense research efforts, auto-parallelization has ultimately failed
    to conclusively solve the problems of parallel computing.
    Heterogeneous computing adds another dimension of complexity on top of
    homogeneous parallelism, making a traditional compiler approach appear
    unfeasable.
    However, research can profit from the experiences with kernel compilers
    and from understanding the resons why concepts from kernel compilers fail to
    translate into host compilers.

    \section{Learning from Success Stories}

    Compilers need to think out of the box and incorporate novel ideas
    in order to leverage the know-how that has evolved in the surrounding
    software ecosystem.

    Programming models built around libraries and novel domain specific
    languages, kernel compilers and library interfaces have had major successes
    in the areas that established compilers have been weakest - namely reaching
    peak performance on rapidly evolving and highly parallel accelerator
    hardware.
    Two examples stand characteristically for the readth of these successes.
    Firstly, implementations of the Basic Linear Algera Subprograms (BLAS)
    interfaces, a standard dating back to the 1970s, are available for most
    accelerators.
    They are widely used and offer unrivaled performance, with implementations
    from hardware vendors and well as from the scientific community.
    Competing implementations use a plethora of approaches to achieve as close
    to peak performance as possible.
    This includes manually written assembly, as well as highly advanced
    compilation approaches, involving novel program representations, polyhedral
    compilation techniques, iterative compilation and many more.
    Secondly, the domain specific language Halide has developed for the domain
    of image processing.
    It has demonstrated an immense optimization potential for compiler based
    optimization under circumstances of highly constrained semantics.

    These success stories, however, still leave many important problems
    unaddressed.
    They are weak where approaches based on host compilers are strong:
    Adoption costs are significant and they are combined with often uncertain
    long-term prospects and very limited cross-platform portability.
    Even in the case of mostly agreed upon standards such as BLAS --
    arguably the best case scenario -- adoption of novel implementations is
    non-trivial in practice due to even slight differences in the interfaces.
    For academic-backed approaches like Halide on the other hand, complete
    rewrites are requires in entirely novel ecosystems with an entirely unclear
    future of support.

    What is promising therefore is a combination of hand-optimized libraries,
    domain specific languages together with compiler automatisms.
    This requires a disruptive improvement in compiler analysis capabilites.
    The detection of higher level algorithmic structures in compilers has been
    investigated before, but established approaches cannot scale to what
    this challenge requires.
    Syntactic matching on programming languages has become
    unviable for the complexities of both modern programming languages and
    complex code bases.

    Instead, this thesis develops an entirely novel approach based on concepts
    from constraint solving, building a pragmatic methodology for detecting
    complex algorithmic structures -- computational idioms.

\pagebreak
\section{Structure of the Thesis}

    This PhD thesis is divided into six chapters.
    Following the introduction in {\bf\cref{chapter:introduction}}, a broad
    overview of the related work is given in {\bf\cref{chapter:literature}}.
    This literature survey covers the four main research areas that this work
    is placed in.
    Firstly {\em constraint programming}, which is the basis of the methodoloy
    for most of this research.
    Secondly {\em compiler analysis and auto-parallelisation}, against which the
    results in this thesis are evaluated.
    Thirdly {\em heterogeneous computing}, which motivates many of the compiler
    approaches that this research enables.
    Lastly {\em computational idioms}, a term for several overlapping concepts
    of algorithmic patterns.

    {\bf\Cref{chapter:theory}} develops the foundations of the constraint
    programming methodology that is forms the core of the later
    \cref{chapter:candl,chapter:idioms,chapter:reductions,chapter:lilac}.
    Based on a novel mathematical model of static single assignment (SSA) form
    compiler representaions, the core of the constraint programming approach
    is derived and some particular challenges are discussed in detail.

    {\bf\Cref{chapter:candl,chapter:idioms,chapter:reductions,chapter:lilac}}
    are each based on a published research article and elaborate on different
    applications of constraint programming in compilers.

    {\bf\Cref{chapter:candl}} develops a full-fledged constraint programming
    language called CAnDL, with an implementation in the LLVM compiler
    infrastructure that automatically generates compiler analysis passes from
    declarative descriptions.
    Using CAnDL, the chapter explores severeal complex compiler analysis
    challenges, concluding with a full polyhedral code analysis.
    The work of this chapter has been published as
    {\bf\citet{Ginsbach:2018:CDS:3178372.3179515}}.

    {\bf\Cref{chapter:reductions}} develops an auto-parallelising compiler for
    complex reduction and histogram computations using CAnDL-powered analysis
    functionality.
    This covers many computations that are inaccesible to established approaches
    based on data flow or polyhedral analysis and is to demonstrate the greater
    flexibility of constraint programming.
    The work of this chapter has been published as
    {\bf\citet{ginsbach2017discovery}}.

    {\bf\Cref{chapter:idioms}} extends CAnDL into the Idiom Description Language
    (IDL) and applies it to algorithmic concepts that go beyond traditional
    compiler analysis: stencil computations, complex reductions and histograms
    as well as sparse and dense linear algebra.
    The resulting automatic detection passes in LLVM enable automatic
    heterogeneous acceleration of sequential code and result in significant
    speedups on established benchmark suites.
    The work of this chapter has been published as
    {\bf\citet{Ginsbach:2018:AML:3173162.3173182}}.

    {\bf\Cref{chapter:lilac}} builds on top of the idiom descriptions for linear
    algebra that were introduced in the previous chapter and demonstrates how
    compiler analysis based on constraint programming can be made accessible to
    non-experts.
    With the novel LiLAC language, constraint programs in IDL are automatically
    generated from a high-level specification for many different sparse linear
    algebra representations.
    This powers a fully integrated compiler approach, handling everything from
    algorithmic detection, library call insertion and run time data transfer
    optimisations.
    The work of this chapter has been published as \citet{lilacpaper}.


\chapter{Contraint Programming on Static Single Assignment Code}
    \label{chapter:theory}
    \section{Computational Idioms}

    The concept of computational idioms has been observed in different contexts
    and remains a rather vague concept.
    While terms such as \texttt{reduction}, \texttt{stencil} and
    \texttt{linear algebra} are commonly used, the concrete concepts can be
    surprisingly vague, although previous work has established several formal
    approaches.
    We will not try to create our own formal definitions in this chapter, but
    instead want to give an overview of the literature that sheds different
    perspectives on this topic.

    The basic observation is that software programs don't cover the space of
    possible programs evenly, instead, they tend to be structured among certain
    design principles.
    The same is true algorithmically and particularly for performance intensive
    applications.

\section{Literature Survey}

\subsection{Algorithmic Skeletons}

\subsection{Berkeley Parallel Dwarves}
\subsection{Computational Patterns}

\section{Idiom Specific Optimizations}

\subsection{Important Approaches}
\subsection{Ways of Encapsulating Expertise}


\subsection{Constraint Programming for Program Analysis}

    There is a large body of work that uses constraints for program analysis.
    Constraint systems have long been used in program analysis for classical
    purposes such as dataflow analysis and type inference
    \citep{Aiken:1999:ISC:339853.339897}.
    \citet{Gulwani:2008:PAC:1375581.1375616} showed how constraints can be
    used to for some  existing analysis problems.
    It can also be used to assist generation of programs from specifications
    \citep{Srivastava:2010:PVP:1707801.1706337}.

    There is considerable work on formally verifying existing compiler
    transformations using SMT solvers and theorem provers that operate on IR
    code, among them \cite{Zhao:2012:FLI:2103656.2103709}.
    Recent domain specific language for formally verifying compiler
    optimizations, such as Alive \cite{Lopes:2015:PCP:2737924.2737965} operate
    on single static assignment compiler IR as well.
    However, it has no support for control flow and is limited to simple
    peephole optimizations.

    Further approaches to generating compiler optimizations that focus on formal
    verification instead of programmer productivity include Rhodium
    \citep{Lerner:2005:ASP:1040305.1040335}, PEC
    \citep{Kundu:2009:POC:1543135.1542513} and Gospel
    \citep{Whitfield:1997:AEC:267959.267960}.
    The CompCert tool by \citet{Mullen:2016:VPO:2908080.2908109} verifies peephole
    optimizations directly on x86 assembly code and LifeJacket
    \cite{Notzli:2016:LVP:2931021.2931024} proves the correctness of floating
    point optimizations in LLVM, as does Alive-FP \cite{Menendez2016}, which
    also generates C++ code.
    The authors of \cite{Tate:2010:GCO:1706299.1706345} investigate the
    automatic generation of optimization transformations from examples.
    None of the approaches however, tackle the issue of efficiently writing
    arbitrary compiler analysis passes.

    \citet{Martin1998} present a specification language for program analysis
    functionality called PAG that is based on abstract interpretation.
    Domain specific languages for the conception of optimization passes have
    also been studied before using tree rewrites, among them
    \citet{Olmos:2005:CSD:2136624.2136643}.
    \citet{Lipps1989} propose the domain specific language OPTRAN for matching
    patterns in attributed abstract syntax trees of Pascal programs.
    These patterns can then be automatically replaced by semantically
    equivalent, more efficient implementations. 
    Generic programming
    concepts can also be used to generate optimization passes, as
    demonstrated by \cite{Willcock:2009:RGP:1621607.1621611}.  These
    schemes however are not able to work at the IR level essential for
    compiler implementation and do not scale beyond simple functions.

    As opposed to code transformation techniques based on the LLVM ASTMatcher
    library and LibTooling \citep{be0fa11ddb194bde86a9dab8589b779c}, we work on
    compiler intermediate representation and are independent from the compiler
    frontend.
    This makes us integrate well with the existing optimization infrastructure,
    allows us to be language independent and makes our approach robust to
    shallow syntactic changes.

    Another language for implementing optimizations that emphasizes developer
    productivity is OPTIMIX \citep{Assmann1996,Assmann98optimix}, based on graph
    rewrite rules.
    OPTIMIX programs are compiled into C code that performs the specified
    transformation.
    A domain specific language for the generation of optimization
    transformations was also used in the CoSy compiler \citep{Alt1994}.
    These are simple rewrite engines and have no knowledge of global program
    constraints.

    Other work has investigated the use of constraint solvers for
    detecting structure in LLMV IR.
    \citet{ginsbach2017discovery} use a solver to detect histograms and
    scalar reductions in order to automatically parallelize code.
    A different important approach to detecting structure in intermediate
    code is the polyhedral model.
    Compilers using this model, such as
    \citet{Lengauer2012Polly}, \citet{Baskaran:2010:ACC:2175462.2175482} and
    \citet{Verdoolaege:2013:PPC:2400682.2400713}, capture well behaved
    loops with affine array accesses and are able to perform advanced loop
    optimizations.
    Recent work has investigated performant reduction computations on GPUs with
    the polyhedral model \citep{Reddy2016Reduction}.
    However this approach is not easily extensible to other program structures
    and captures only a very specific class of well behaved programs.

    In \cite{Yuan:2017:TOS:3101282.3101287}, the authors propose the use of the
    Web Ontology Language (WOL) for the description of software architectures.
    This representation enables automatic reasoning and analysis about the
    interoperability of software architectures.

\chapter{Literature Survey}
    \label{chapter:literature}
    
    Four areas of research are of particular relevance to this thesis:
    {\bf Constraint programming and specification languages} are central to the
    introduced methodology.
    The relevant literature includes the research into constraint programming in
    the context of program analysis and the design of specification languages.
    The survey of previous approaches to
    {\bf compiler analysis and auto-parallelisation}
    establishes the baselines for the later evaluation sections.
    Related work on {\bf heterogeneous computing} motivates the proposed
    approaches, by presenting the plethora of other programming paradigms to
    overcome the specific challenges of emerging hardware.
    Lastly, the diverse research landscape around concepts related to
    {\bf computational idioms} puts the algorithmic structures that are detected
    in later chapters of this thesis into context.

\section{Constraint Programming and Specification Languages}

    Declarative Languages, constraint programming, and the application of
    constraints to program analysis problems are well-established in the
    literature.
    Previous work covers query languages, logic programming, applications to
    software security and formal verification, model checking and SMT,
    but also more compiler-centric data flow analysis and type inference
    problems.
    The limited scope of this section requires a focus on work that is
    particularly relevant to this thesis.

    For this research, constraint programming is most interesting within the
    context of research fields such as program analysis and model checking.
    Crucial background material for this thesis also comes from
    the programming language design community of declarative programming
    languages.
    Prolog and its many extensions and dialects particularly stand out as
    fully-fledged logic programming languages, but parallels can also be drawn
    to querying languages that apply database techniques to static analysis.

    These different fields vary significantly in their interests, motivations,
    and approaches, but the underlying challenges are often similar.
    Notably, the performance of backtracking solvers and the scalability to
    complex problems are a recurring theme.

\subsection{Constraint Programming for Program Analysis}

    \paragraph*{Constraint analysis on abstract languages}
    Constraint systems have long been used for program analysis.
    \citet{Aiken:1999:ISC:339853.339897} gives a comprehensive overview of
    earlier work, highlighting the crucial ability of constraint-based program
    analysis to separate {\it constraint specification} from
    {\it constraint resolution}.
    This separation is critical also for this thesis, as it enables the
    scalability of compiler analysis problems beyond what could reasonably be
    implemented with manual recognition routines.
    The constraint specification can be formulated briefly, by offloading the
    {\it constraint resolution} to a separate solver.
    However, the article does not present any techniques to capture higher-level
    algorithmic concepts like computational idioms.
    Instead, it focuses on more basic compiler analysis problems, such as
    data flow analysis and type inference.

    More recent work on constraint-based program analysis by
    \citet{Gulwani:2008:PAC:1375581.1375616} leverages the advancements in
    modern off-the-shelf SAT/SMT solver technology.
    The analysis problems are lowered to bit-vector formulations, and the
    {\it constraint resolution} is entirely externalised to independently
    developed SMT solvers.
    The motivation of the approach is mainly to verify program properties,
    as opposed to the application of parallelising code transformation in this
    thesis.
    Furthermore, \Cref{sec:SATcomp} showed that the confinement to
    conventional SMT solvers is inefficient for the resolution of
    SSA constraint problems.

    \citet{Kundu:2009:POC:1543135.1542513} propose constraints to verify the
    correctness of program transformations with their system for
    Parameterized Equivalence Checking (PEC).
    This system improves on previous work performing translation validation.
    Translation validation is the validity checking of transformations on
    concrete input programs by comparing the semantics before and after
    modification.
    PEC implements a hybrid approach that allows some aspects of the program
    to be underspecified, yet does not check the soundness of transformations
    in full generality.
    The checking is done via a custom solver for the generated constraint
    problems.
    The hybrid nature allows the system also to validate program transformations
    that significantly modify the control flow, e.g.\ loop unswitching.
    However, it cannot discover transformation opportunities, only verify them
    after the transformation was applied.

    \paragraph*{Constraint analysis on compiler IR code}
    There is previous work on using constraint-based program analysis for real
    compiler intermediate representations, mostly in the area of security and
    the formal verification of software systems.
    This includes investigations into using SMT on LLVM intermediate
    representation, which is also used for the implementations in this thesis.
    \citet{Zhao:2012:FLI:2103656.2103709} built a model of LLVM IR for such
    solvers.
    However, this model serves an entirely different purpose to the SSA model
    of this thesis.
    The model provides operational semantics, but cannot be used to detect
    large-scale algorithmic structures in user programs, as is required for
    automatic heterogeneous acceleration.
    Instead, the focus is on formally verifying the correctness of existing
    compiler transformations for all possible user input.

    Recent domain-specific languages, such as Alive
    \citep{Lopes:2015:PCP:2737924.2737965}, operate on subsets of LLVM IR.
    The individual instructions are reformulated on bit-vectors, and the
    correctness of conditions is checked with an SMT solver.
    However, Alive only implements a subset of LLVM's integer and pointer
    arithmetic instructions.
    It has no support for control flow and does not scale to the applications
    that are used in this thesis for evaluation.
    Instead, it is designed for formally verifying already existing
    compiler optimisations that operate on only a handful of integer
    instructions at a time.
    Alive is meant to improve compilers, not user programs.

    LifeJacket \citep{Notzli:2016:LVP:2931021.2931024} proves the correctness of
    floating-point optimisations in LLVM, as does Alive-FP \citep{Menendez2016}.
    Both of these projects are extensions of the SMT-based Alive system.
    They extend the scope of the system to model a wider range of
    instructions as bit-vectors, enabling the verification of more compiler
    optimisations.
    LifeJacket and Alive-FP were successfully used to identify incorrectly
    implemented  optimising transformations in compilers.
    However, the fundamental limitations of Alive remain, and control flow is
    not supported.
    Therefore, only peephole optimisations can be evaluated by these approaches.

    The Alive-Infer system \citep{Menendez:2017:ADP:3062341.3062372} also builds
    on Alive but goes beyond the verification of existing compiler
    optimisations.
    The tool uses an SMT solver to automatically generate preconditions that
    need to hold for transformations to be applied.
    This moves the Alive system away from verifying optimisations and
    closer to automatically detecting algorithmic structure in parts of user
    programs.
    However, Alive-Infer still requires the separate specification of the actual
    transformation.
    It only generates additional conditions and does not handle control flow.

    \paragraph*{Constraint analysis on other program models}
    Other advanced approaches to extracting high-level code structures
    from programs that use constraints and verification systems have been
    proposed.
    \citet{Mendis2015Helium, Kamil2016Verified} suggest temporal logic as the
    foundation to formulate the necessary conditions for rephrasing
    well-structured Fortran and assembly code in restrictive models.
    These techniques leverage counter-example guided inductive synthesis to find
    provably correct translations into the high-level Halide language.
    The Halide compiler specialises the code again, exploring the
    optimisation space via powerful transformations that are enabled by its
    restrictive semantics.
    However, the focus is on a small class of computations, involving only
    dense memory access.
    This allows formal reasoning about correctness but is too restrictive for
    interesting computational idioms, such as sparse linear algebra.

    \citet{Mullen:2016:VPO:2908080.2908109} study low-level program
    transformations that are implemented for the formally verified CompCert
    compiler \citep{CompCert-ERTS-2018}, directly on x86 assembly.
    Instead of an automatic verification after modelling optimisations
    as SMT problems, the presented Peek system was checked with the interactive
    theorem prover Coq.
    This required approximately 30000 lines of manually written Coq code and
    proof lines.
    Some of the transformations consider rudimentary control flow constraints,
    but they cannot scale to computational idioms.

\subsection{Declarative Programming Languages for Program Analysis}

    \paragraph*{Languages for querying program properties}
    From the perspective of language design, the declarative programming
    languages Prolog and SQL are perhaps most influential.
    The two languages differ fundamentally.
    Prolog ("programmation en logique") is a logic programming language that
    originated in academia for analysing natural language
    \citep{Colmerauer:1993:BP:154766.155362}.
    By contrast, SQL ("Structured Query Language") was developed at IBM for
    managing data in relational database management systems
    \citep{Chamberlin:1974:SSE:800296.811515}.
    Nonetheless, specification languages for structures in program code have
    been designed taking inspiration from both backgrounds.

    The first such specification language was the Omega system by
    \citet{Linton:CSD-83-164}.
    It uses a relational database to store all the relevant properties of a
    program.
    The captured information is based on the abstract syntax tree of programs
    that are implemented in a subset of the Ada programming language.
    Additional edges are inserted to connect shared variables of
    successive expressions, given some indication of data flow between
    instructions.
    The system then allows database-style queries formulated in QUEL
    \citep{Stonebraker:1976:DII:320473.320476}, an SQL-style language.

    The CodeQuest system \citep{Hajiyev:2006:CSS:2171327.2171331} first combined
    the ideas of Omega and its database-oriented successors with the use
    of logic programming.
    The queries are translated into Datalog, a Prolog derivative that is
    implemented on top of SQL.
    This allows CodeQuest to be fundamentally more expressive, allowing
    recursive queries that are required for meaningful CFG inspection.
    Nonetheless, the approach is based on querying for source language features.
    This makes the detection of large-scale algorithmic structures in complex
    programming languages such as C++ infeasible, as demonstrated in
    \Cref{sec:syntacticmatching}.

    \paragraph*{Languages for generating compiler passes}
    Custom specification languages for generating compiler analysis and
    transformation passes have been presented in the literature.
    \citet{Martin1998} introduced a specification language for program analysis
    functionality called PAG, based on abstract interpretation.
    The generated functionality was integrated into C and Fortran compilers via
    a well-specified interface and applied successfully to real benchmark codes.
    However, the tool is focused on relatively simple compiler optimisations
    such as constant propagation.

    Domain-specific languages for compiler transformation passes were also
    studied by \citet{Olmos:2005:CSD:2136624.2136643}.
    The proposed Stratego system uses rewrite rules to apply tree
    transformations to the abstract syntax tree of source programs.
    However, this was only evaluated on the Octave language, and the general
    applicability on large-scale programs remains unclear.

    \citet{Lipps1989} designed the domain-specific language OPTRAN for matching
    patterns in attributed abstract syntax trees of Pascal programs.
    Semantically equivalent, more efficient implementations can then
    automatically replace the matching code patterns.
    With the focus on Pascal, it remains unclear how the proposed concepts
    translate to the complex C++ programs with pointer calculations that were
    used for the evaluation of this thesis.

    Another language for implementing compiler optimisations from
    declarative specifications is OPTIMIX \citep{Assmann1996,Assmann98optimix}.
    Similarly to the presented work in \Cref{chapter:candl}, OPTIMIX emphasises
    developer productivity.
    However, it is based on graph rewrite rules.
    OPTIMIX programs are compiled into C code that performs the specified
    transformation.
    Such a domain-specific language for the generation of optimisation
    transformations was also used in the CoSy compiler \citep{Alt1994}.
    Both OPTIMIX and the CoSy method are simple rewrite engines that have no
    knowledge of global program constraints.

    Different code transformation techniques use LibASTMatchers
    and LibTooling \citep{be0fa11ddb194bde86a9dab8589b779c} from the
    LLVM project.
    These tools do not provide a complete standalone language but are instead
    implemented in C++ as an embedded domain-specific language for pattern
    matching, relying heavily on other LLVM libraries.
    The approach is deeply integrated with the Clang compiler and exposes the
    abstract syntax tree (AST) of the compiler frontend directly.
    There are more than a thousand separate classes that implement types
    of AST nodes in Clang, introducing considerable complexity for any
    non-trivial pattern.
    Therefore, this is an entirely impractical approach for detecting complex
    algorithmic structures such as computational idioms.

    \citet{Willcock:2009:RGP:1621607.1621611} designed a complex system for
    generating generic optimisation passes using concepts from generic
    programming.
    However, such schemes do not work at the IR level of established
    compiler frameworks.
    Instead, they require program rewrites by the user.

    \citet{Whitfield:1997:AEC:267959.267960} praise Gospel, their framework and
    specification scripture for the exploration of the properties of
    code-improving transformations.
    The project furthermore includes the Genesis tool, which automatically
    generates transformers as specified in Gospel.
    Several standard optimisations were implemented with Gospel and Genesis,
    such as constant folding and common subexpression elimination.
    Similar approaches to generating compiler optimisations from specification
    languages include Rhodium \citep{Lerner:2005:ASP:1040305.1040335}.
    The language expresses optimisations using explicit data flow facts, which
    are manipulated by local propagation and transformation rules.
    The transformations are applied to a custom intermediate language and can
    be proven correct with a theorem prover.
    Neither Gospel nor Rhodium provides means to tackle the issue of efficiently
    enabling large-scale program transformations.

\section{Compiler Analysis and Auto-Parallelisation}

    The core motivation for the work in this thesis is the automatic
    heterogeneous parallelisation of sequential code.
    The derived methods are evaluated on two metrics: how broadly they apply to
    real code, and how significant their performance impact is when applied.
    These metrics are evaluated against other compiler analysis and
    parallelisation approaches.
    This section focuses on three areas of the vast research landscape
    that are particularly relevant for this: polyhedral compilation, the
    parallelisation of reductions, and dynamic approaches.

\subsection{Compilation with the Polyhedral Model}

    The polyhedral model \citep{Karp:1967:OCU:321406.321418} is an established
    mathematical framework for modelling, analysing, and transforming
    well-behaved loop nests.
    Iterations in loop nests are treated as lattice points in a
    multi-dimensional grid.
    The iteration space can then be transformed with affine maps, potentially
    uncovering new parallelisation opportunities.
    This basic approach has been applied extensively in compilers.
    Furthermore, the required conditions have been relaxed in different ways,
    allowing the application of the approach to more input code.

    \paragraph*{Polly in LLVM}
    Polyhedral optimisers have been integrated into mainstream C/C++ compilers.
    Most notably, \citet{Lengauer2012Polly} implemented the Polly extensions for
    LLVM.
    Polly recognises parts of LLVM IR that are expressible in the polyhedral
    model and transforms them into that representation.
    Polyhedral optimisations can then be applied with PLuTo
    \citep{Bondhugula:2008:PAP:1375581.1375595} before the model is
    translated back into optimised LLVM IR for further treatment by the core
    compiler.
    This enables the seamless application of polyhedral techniques on
    large-scale applications without source code changes via the many
    frontends of the LLVM infrastructure.
    However, this impacts only code that the tool can translate
    into a polyhedral representation.

    Polly-ACC \citep{polly-acc} is an extension of the Polly compiler that
    provides code generation for heterogeneous hardware.
    The tool uses the recognition functionality of standard Polly to detect
    code sections in LLVM IR that can be represented in the polyhedral model.
    These code sections are optimised with established polyhedral
    transformation techniques from \citet{Lengauer2012Polly}.
    However, the optimised polyhedral code sections are then translated into
    CUDA code to be executed on the GPU.
    This results in significant speedups of some benchmark programs, but the
    impact remains limited to code that fits the polyhedral model.

    \citet{Doerfert2015Polly} extended the applicability of polyhedral
    transformations within the Polly compiler to a broader set of input
    programs.
    Dependencies between iterations that originate from reduction variables
    cannot be eliminated with affine transformations.
    Therefore, they prohibit DOALL parallelism in a way that standard Polly is
    unable to resolve.
    However, the reduction-enabled scheduling approach for Polly can parallelise
    such loop despite the reduction dependencies.
    This ability significantly improved the achieved speedup of Polly on
    benchmark programs that contain reductions.

    \citet{Doerfert:2017:OLO:3049832.3049864} also investigated another method
    for widening the scope of polyhedral code transformations.
    This approach allows some conditions that are required for the legal
    application of transformations to remain unproven at compile time.
    These conditions are then checked at runtime, providing a fallback to the
    original code when assumptions are not met.
    The checks that this work allows to be delayed include the absence of
    aliasing, finite loop boundaries, and in-bounds memory accesses.
    This enabled Polly to cover 3.9$\times$ as many loops in the SPEC and NPB
    benchmarks at a negligible runtime overhead.

    \paragraph*{Other tools with automatic detection}
    The Polyhedral Parallel Code Generator (PPCG)
    \citep{Verdoolaege:2013:PPC:2400682.2400713} is a source-to-source compiler
    that takes sequential C programs and generates optimised CUDA kernels to
    target GPU acceleration.
    The extraction of polyhedral code sections from the C input is based on the
    Polyhedral Extraction Tool \citep{Verdoolaege12polyhedralextraction}.
    The extraction method can automatically detect relevant code regions, but it
    is implemented on syntax level and relies on purpose-built C code with all
    arrays declared in variable-length C99 array syntax.
    This is not robust enough to reliably cover larger programs from benchmark
    collections such as NPB or Parboil, which are used for evaluation in this
    thesis.

    C-to-CUDA \citep{Baskaran:2010:ACC:2175462.2175482} is another compiler that
    offers heterogeneous acceleration of sequential C code by representing it in
    the polyhedral model.
    However, the focus is on code generation and the application of optimising
    transformations.
    The automatic recognition in the abstract syntax tree of parallel loops that
    can be represented in the polyhedral model remains ad-hoc and handles only a
    small set of benchmarks.

    \paragraph*{Increased applicability of polyhedral transformations}
    Recent work by \citet{baghdadi2015PENCIL} has extended the polyhedral model
    beyond affine programs to some forms of sparsity with the
    Platform-Neutral Compute Intermediate Language (PENCIL).
    This intermediate language is intended for heterogeneous systems and
    provides backends for accelerator programming.
    This platform provides extensions, which can be used to model important
    features of sparse linear algebra, such as counted loops
    \citep{Zhao:2018:PCF:3178372.3179509}, meaning loops with dynamic, memory
    dependent bounds but statically known strides.
    Such loops are central to sparse linear algebra.

    Tiramisu \citep{Baghdadi:2019:TPC:3314872.3314896} is a polyhedral framework
    for targeting heterogeneous hardware, providing backends for CPUs, GPUs,
    distributed architectures and FPGAs.
    Optimisations are performed on four layers of intermediate representation,
    resulting in performance that almost matches dedicated library functions.
    However, the tool does not detect polyhedral code sections in existing
    source code.
    Instead, it requires the programmer to implement the algorithms manually
    with a dedicated C++ API.

    \citet{Zhang:2016:CTG:3018843.3018849} propose an extension to polyhedral
    approaches that allows the capturing of some sparse linear algebra
    calculations in the polyhedral model.
    They introduce a novel non-affine split transformation for this purpose.
    Using the inspector-executor model, the approach achieved significant
    speedup on some benchmark programs.
    However, the paper does not address the automatic recognition of sparse
    linear algebra routines within existing programs.
    Furthermore, the approach is not evaluated against state-of-the-art
    library implementations such as Intel MKL and cuSPARSE.

    Many approaches have been proposed for parallelising loop nests with
    reduction variables in the polyhedral model, among them
    \citet{jouvelot1989unified,redon1994scheduling,chi1997optimizing,
    gupta2006simplifying,stock2014framework}.

\subsection{Reduction Parallelism}

    Discovering and exploiting scalar reductions in programs has been studied
    for many years based on dependence analysis and idiom detection.
    Early work by
    \citet{pottenger1995idiom,suganuma1996detection,fisher1994parallelizing}
    focused on well-structured Fortran code and often paid little attention to
    robust detection in more complex programs.
    \citet{rauchwerger1999lrpd} went beyond previous static approaches and
    developed a dynamic test to speculatively exploit reduction parallelism.
    Work by
    \citet{Gutierrez:2000,gutierrez2003optimization,gutierrez2008analytical}
    has focused on the exploitation of reductions rather than discovery.
    Approaches to heterogeneous acceleration examined trade-offs in
    implementation \citep{yu2006adaptive} or exploitation of novel hardware
    \citep{ravi2010compiler,Huo2011HiPC}.

    The treatment of more general reduction operations has received less
    attention.
    \citet{das2010experiences} proposed dynamic profile analysis to guide
    manual analysis and show there is potential for finding generalised
    reductions.
    \citet{kim2012dynamic} explored the use of dynamic analysis further,
    but state that detecting reductions on arrays remains challenging.

    The difficulty in automatically detecting reductions has led to languages
    and annotation-based approaches, where it is the responsibility of the user
    to mark reductions in the program.
    Such a system was proposed by \citet{deitz2002high}.
    An annotation approach is also described by \citet{Reddy2016Reduction},
    based on the Platform-Neutral Compute Intermediate Language (PENCIL)
    \citep{baghdadi2015PENCIL}.
    This used the PPCG code generator by
    \citet{Verdoolaege:2013:PPC:2400682.2400713} to generate CUDA and OpenCL
    code for multiple computing platforms.

    There has also been recent work extending on \citet{rauchwerger1999lrpd}
    with more aggressive speculation and dynamic analysis
    \citep{aguilar2015unified} to exploit reduction parallelism.
    \citet{Han2010Speculative} present an approach for the
    parallelisation of a wide class of scalar reductions.
    They start from the observation that many reductions in real benchmark
    programs are not detected by current static analysis approaches.
    They propose a hardware-assisted speculative parallelisation approach for
    likely runtime reductions, denoted ``partial reduction variables''.
    Candidates for speculative parallelisation are determined by searching for
    update-chains in the data flow graph.
    The approach was evaluated on some of the SPEC2000 benchmarks using a
    simulator.
    They achieve up to $46\%$ speedup by including speculative reductions.
    However, this approach requires hardware speculation support.
    Despite this hardware support, it is unable to detect histogram reductions.

    Privateer \citep{Johnson:2012:SSP:2254064.2254107} is a complex system
    featuring compiler support and a runtime to enable speculative
    parallelisation.
    The core approach is the privatisation of memory for each thread and an
    exception mechanism with recovery routines for accesses that violate
    parallelism.
    The authors explicitly allow for reduction parallelism involving only a
    single scalar associative and commutative operator.
    The evaluation only covers a set of five benchmark programs, yielding
    a geometric mean speedup of 11.4x on a 24 core machine.
    The runtime overhead was up to $>50\%$.
    Despite this complexity, the approach only exploits simple scalar reductions.

\subsection{Dynamic Analysis Approaches}

    There is an extensive body of work on automated decision making for
    selecting the appropriate hardware for a given piece of code.
    This is often under the assumption that the functional porting is trivial,
    for example given an OpenCL implementation.
    These approaches are often based on dynamic monitoring of programs
    and machine learning algorithms for the selection of backends.

    \citet{Wen:2017:MSM:3038228.3038235} present a system for scheduling OpenCL
    kernels on a system with CPU and GPU processors, extending the approach
    presented in \citet{7116910}.
    The presented machine learning-based predictive model decides at runtime
    whether kernels are merged or executed separately on appropriate devices.
    However, such a system can only be applied when the functional translation
    to accelerators is available.
    This is not the case for the sequential C code that this thesis is evaluated
    on.
    Furthermore, the approach optimises OpenCL scheduling, but does not consider
    the impact of library backends, which often significantly outperform generic
    OpenCL implementations on appropriate tasks.

    \citet{Tournavitis:2009:THA:1542476.1542496} characterised the significant
    weaknesses of established static data dependence analysis techniques.
    Profile-driven parallelism detection and machine-learning based mapping
    approaches are suggested in order to improve on the state-of-art
    parallelising compilers.
    \citet{Wang:2014:IPP:2591460.2579561} implemented such a system that
    automatically discovers parallelism based on profile-driven parallelism
    detection.
    The approach improves significantly over purely static approaches by
    replacing the traditional target-specific and inflexible mapping heuristics
    with a prediction mechanism that uses machine learning.
    The model is trained via an offline supervised learning scheme, using both
    static and dynamic features, such as cache miss rates and branch miss
    prediction rate.
    Dynamic approaches can eliminate spurious dependencies and profitability
    models based on powerful machine learning techniques greatly improve on
    simple heuristics.
    However, such an approach cannot unlock the potential of dedicated backends
    and requires significant manual tuning effort.

    \citet{Ogilvie:2014:ALA:2628071.2628128} propose a system for performance
    prediction on heterogeneous systems that is based on active learning.
    This significantly reduces the tuning that is required for
    machine learning-based scheduling algorithms on CPU/GPU systems.
    However, this approach still requires the availability of functional mapping
    to heterogeneous devices, and cannot work on unchanged sequencial C/C++
    inputs.

    \citet{Manilov:2018:GPI:3178372.3179511} propose a dynamic approach to
    detecting a wide class of iterators using dynamic profiling data.
    Such iterators describe the traversal of data structures that are difficult
    to capture with tranditional static techniques.
    This is an important prerequisite for implementing compiler parallelisation
    approaches to pointer-based data structures.
    However, the approach only captures a small, if crucial, part of the
    calculations, and does not capture full computational idioms such
    as sparse linear algebra.

\section{Heterogeneous Computing}

    Heterogeneous computing has been a particularly active field of research
    since the widespread adoption of GPUs for general purpose computations in
    the last decade.
    This includes research from both software and hardware perspectives.
    The hardware research investigates the most promising directions of
    diversification for processors in heterogeneous systems
    \citep{Tomusk:2016:SHC:3012405.3014165}.

    However, the related work in the context of this research is from
    the software perspective.
    This section focuses on the different programming approaches that
    have been championed for targeting existing heterogeneous accelerators.
    These methods broadly fall into two categories: library approaches, and
    domain-specific languages.

\subsection{Libraries}

    Library interfaces are often the most performant ways to exploit
    heterogeneous performance.
    However, they provide narrow interfaces, accelerating only very particular
    computations.
    The established way of encapsulating fast linear algebra is via dedicated
    library implementations based on the BLAS interface specification
    \cite{2002:USB:567806.567807}.
    These are generally very fast on their specific hardware platforms, but
    require application programmer effort and offer little performance portability.
    Implementations of dense linear algebra are available for most suitable
    hardware platforms, such as cuBLAS \cite{cublas} for NVIDIA GPUs, clBLAS
    \cite{clblas} for AMD GPUs and the Intel MKL library \cite{mkl} for Intel
    CPUs and accelerators.

    Dense linear algebra is the best-supported class of calculations.
    However, implementations of sparse linear algebra also exist for
    the most important platforms, including cuSPARSE \cite{cusparse} for NVIDIA
    GPUs and clSPARSE \cite{clsparse} built on top of OpenCL.

    While most individual library implementations focus on a single target
    hardware platform, some BLAS implementations attempt platform independent
    acceleration and heterogeneous compute.
    Among them are systems by \citet{Wang:2016:BHP:2925426.2926256,
    10.1007/978-3-319-64203-1_33, Diego2017Multi}.

    More expressive computational idioms, such as reductions and stencils, are
    not suitable for library implementation.
    These idioms are parameterised with kernel functions, which can be
    implemented as callbacks, but prevent a direct execution on heterogeneous
    hardware.
    Instead, domain-specific languages provide the appropriate abstraction
    level for these computations.

    \paragraph*{CPU-GPU data transfer optimisations}
    Library implementations often require the manual management of CPU-GPU
    data transfers.
    These transfers have been studied extensively as important bottlenecks for
    parallelisation efforts.
    Work by \citet{Jablin:2011:ACC:1993316.1993516} established a
    method for the automatic management of CPU-GPU communication.
    Similarly, \citet{Lee:2009:OGC:1594835.1504194} implemented a system to
    optimising data transfers using data flow analysis, although this was in
    the context of moving OpenMP code to GPUs.

\subsection{Domain-Specific Languages}

    Many domain-specific languages have been proposed for the efficient and
    easy programming of heterogeneous systems.
    They allow implementers to restrict the compiler and runtime away from
    general purpose programming concepts that are difficult to support on
    specific hardware.
    Domain specific languages can be stand alone with an entire tool chain and
    runtime ecosystem of be embedded in existing languages, such as C++ or
    Scala.
    DSLs range in complexity from only marginally more flexible than library
    interfaces to full-fledged programming languages such as OpenCL and CUDA.

    \paragraph*{Functional languages}
    Lift \citep{steuwer15rewrite} provides composable constructs that enable the
    functional implementaion of data-parallel algorithms and operations.
    The language is especialy suitable for dense linear algebra applications
    \citep{Steuwer:2016:MMB:2968455.2968521} and stencil codes
    \citep{Hagedorn:2018:HPS:3179541.3168824}, but extensions to support
    some forms of sparsity exist as well
    \citep{Pizzuti:2019:PAA:3331553.3342614}.
    Lift performs optimisations by applying functional rewrite rules.
    This extensible set of rewrite rules allows the tool to cover a very large
    space of possible program transformations.
    However, selecting the best version from this large available configuration
    space requires guidance from profiling runs.
    Profiling runs can be computationally expensive, taking approximately
    one day to evaluate a sufficient number of variants for tuning the matrix
    multiplication kernel \citep{Steuwer:2016:MMB:2968455.2968521}.
    Such user effort can be prohibitive, but promises highly tuned OpenCL
    outputs.

    There exist multiple other functional approaches to generating code for
    heterogeneous hardware.
    Among them, \citet{chakravarty11accelerating,mcdonell13optimising} propose
    Accelerate, a domain-specific language that is embedded in Haskell.
    Accelerate applies sharing recovery and loop fusion optimisations to
    generates efficient GPU code.
    However, many of these techniques target particular challenges that arise
    from the untypical nature of Haskell, especially the methods for interfacing
    heterogeneous accelerators from a lazily evaluated environment.
    This makes Accelerate unsuitable for evaluation in this thesis, which
    focuses of benchmarks that are provided as C/C++ programs.


    Copperhead \citep{catanzaro11copperhead}, is a data parallel language
    embedded in Python.
    It exposes parallelism via higher-order functions such as {\it map},
    {\it gather}, and {\it reduce}.
    However, Copperhead is unable to compile the formulated programs into
    standalone binaries, leaving the programs tightly integrated in a Python
    environment.
    This makes the interface with C/C++ code nontrivial and is unsuitable for
    the acceleration of existing benchmarks in the evaluation of this thesis.

    \citet{collins14nova} introduced NOVA, a functional language targeted at
    code generation for GPUs.
    The evaluation showed comparable performance to dedicated library
    implementations on several important applications, including sparse
    matrix-vector multiplication.
    However, the work is highly focused on the generation of CUDA code and does
    not provide an OpenCL backend, which is crucial for evaluation on GPUs of
    multiple vendors.

    \paragraph*{Intermediate languages}
    Delite \citep{Sujeeth:2014:DCA:2601432.2584665} was presented as an
    intermediate representation that facilitates the rapid construction of
    domain-specific languages.
    The provided infrastructure targets heterogeneous
    platforms, with backends avaliable for OpenMP, CUDA, and even MPI for
    cluster computing.
    Delite-based DSLs are proposed for machine learning, data querying, graph
    analysis, and scientific computing.
    However, the Delite approach is tightly integrated with the Scala
    language and does not offer a readily available end-to-end solution.

    Halide, as proposed by \citet{Ragan-Kelley:2013:HLC:2499370.2462176}
    was designed for image processing, but is flexible enough to also allow the 
    formulation of matrix multiplications and other computations.
    \citet{Suriana:2017:PAR:3049832.3049863} demonstrates that this can extend
    to reduction computations as well.
    Its core design decision is the scheduling model that allows the separation
    of the computation schedule and the actual computation.
    There has been follow-up work on automatically tuning the schedules, e.g.\ 
    \citet{Mullapudi:2016:ASH:2897824.2925952}, but by default the burden of
    implementing efficient schedules is put on the application programmer.

    \paragraph*{Embedded languages}
    MILK \citep{Kiriansky:2016:OIM:2967938.2967948} is a pragma-based
    domain-specific language to annotate indirect memory accesses in C++.
    The approach is inspired by OpenMP, and are supported by modified versions
    of Clang and LLVM.
    This allows low level optimisations that are particularly applicable to
    sparse linear algebra.
    The authors report performance gains of up to 3x, but the approach is unable
    to utilise the much greater potential of heterogeneous compute and requires
    detailed programmer intervention.

    \paragraph*{Other approaches}
    Spatial \citep{Koeplinger:2018:SLC:3192366.3192379} is a domain-specific
    language for higher-level descriptions of application accelerators.
    The language provides hardware-centric abstractions, but also takes
    programmer productivity into consideration.
    However, the language does not target the most established heterogeneous
    CPU-GPU systems.
    Instead, the focus is on on Field Programmable Gate Arrays (FPGA) and
    Coarse Grain Reconfigurable Architectures (CGRA).
    This makes it unsuitable for comparative evaluation with the methods
    proposed in this thesis.

    There have been multiple domain-specific libraries proposed specifically
    for linear algebra computations.
    \citet{Spampinato:2014:BLA:2581122.2544155,
    Spampinato:2016:BLA:2854038.2854060} propose and extend the high-level
    language LGen for linear algebra, based on standard mathematical notation.
    The implemented routines are optimised with an autotuning compiler,
    exploring many transformations such as tiling, loop fusion, and
    vectorisation.
    The tool improves over Intel MKL on specific small-scale matrices, but is
    unable to generate code for GPUs.

    Recent research has highlighted the challenges of generating code that
    performs well on different heterogeneous hardware architectures.
    PetaBricks \citep{Ansel:2009:PLC:1542476.1542481,PhothilimthanaARA13} was
    one of the first languages to address this performance portability challenge
    by encoding algorithmic choices which are then empirically evaluated and
    automatically taken by the
    compiler.
    Similarly, \citet{MuralidharanRHG16} explored automatic selection of code
    variants using machine learning.

\section{Computational Idioms}

    The ideas behind the concept of {\it computational idioms} have been
    observed in different contexts.
    The basic observation is that software programs do not cover the
    space of possible programs evenly.
    Instead, they tend to be structured among certain design principles.
    This is true in particular for performance intensive programs and the
    core bottlenecks of large applications.

    The concrete concepts are partially overlapping and can be vague.
    Software design patterns are a way of understanding program code
    strucutres as specialised implementation of a class of standard approaches
    in the discipline of software engineering.
    Terms such as {\em map and reduce}, {\em stencil code}, and
    {\em linear algebra} are commonly used when designing libraries and
    domain-specific languages.
    Scientific computing is mostly concerned with the architectural
    implications of specific memory access patterns that are instrinsic to
    the choice of certain algorithmic approaches.
    This section apptempts to demarcate a meaningful conception of
    {\it computational idioms} by comparison with the existing literature of
    these different domains with related concepts.

\subsection{Higher-Order Functions}

    Many functional programming languages, such as OCaml and Haskell,
    encapsulate high-level algorithmic choices and common programming patterns
    as higher-order functions \citep{Hughes:1989:WFP:63410.63411}.
    These are functions that are parameterised with other functions.
    Examples of higher-order functions are {\it map}, which applies a function
    to each element in a data structure, and {\it fold} / {\it reduce}, which
    accumulates the elements in a data structure with a reduction operator.

    Common computational workloads can be expressed as instances of
    higher-order functions.
    For example, the popularity of the MapReduce framework
    \citep{Dean2008MapReduce} stems from the observation that many big data
    workloads exhibit characteristics that can be expressed efficiently with
    combinations of {\it map} and {\it reduce}.
    The framework provides an idiomatic approach to the development of big
    data applications, enabling shorter development times and more predictable
    performance.

    The use of computational idioms for automatic heterogeneous acceleration
    requires a more restrictive view of types than what is common in
    functional programming languages.
    For example, the {\it reduce} operator allows the implementation of
    the insertion sort algorithm, as well as a simple sum over an array of
    floating-point values.
    These two algorithms do not share parallelisation opportunities.
    Therefore, the detection of {\it reduce} instances is insufficient for
    enabling compiler parallelisation approaches.
    However, more restrictive versions of {\it reduce} are suitable for
    compiler detection.
    \Cref{chapter:reductions} studies in detail the class of Complex Reduction
    and Histogram Computations, which is formulated as a computational idiom.
    This restricted class of {\it reduce} calculations shares a common
    parallelisation approach.

\subsection{Parallel Dwarfs}

    The {\it Berkeley Dwarfs} are a collection of 13 computational methods
    that together comprise a large portion of the most common parallel computing
    workloads \citep{Asanovic06thelandscape}.
    Each Dwarf is a computational pattern that is common in important
    applications.
    The core observation of the authors is that the nature of the dwarfs has
    persisted more or less identical for many years, even as concrete
    applications changed.
    The Dwarfs are inspired by numerical computations that arise in the
    scientific computing community, although the authors claim that the
    knowledge from this domain may prove useful in other areas as well.

    The Berkeley Dwarfs are studied from the perspective of architecture
    requirements, not with automatic compiler support in mind.
    Therefore, some of the dwarfs are specified too broadly for use as
    computational idioms in this paper.
    However, dense linear algebra, sparse linear algebra, and strcutured grid
    computations (stencils) are important idioms in \Cref{chapter:idioms}.

\subsection{Algorithmic Skeletons}

    Another abstraction that relates to computational idioms as used in this
    thesis is the notion of Algorithmic Skeletons \citep{Cole1991Algorithmic}.
    This concept was introduced to classify the behaviour of parallel programs
    according to their organisation of workload distribution.
    The motivation of this approach was to enable the introduction of new,
    higher-level programming models and tools for parallel programming.
    Higher-order functions from functional programming were a major inspiration,
    observing a lack of similar abstractions on more mainstream programming
    languages.
    Among the Algorithmic Skeletons are Fixed Degree Divide \& Conquer and the
    Task Queue.

    The concept of Algorithmic Skeletons has been used to implement many
    programming frameworks and libraries.
    The eSkel library was sketched by \citet{Cole2004Bringing}, providing a
    higher-level programming model based on skeletons on top of C and MPI.
    Skandium \citep{Leyton2010Skandium} is a parallel skeleton library for
    multi-core architectures.
    Eden \citep{Loogen2005Parallel} provides skeletons for parallel programming
    in Haskell.
    SkelCL \citep{Steuwer2011SkelCL} provides implementations of algorithmic
    skeletons that target GPUs via CUDA.
    This is implemented in C++, providing templated versions of higher-order
    functions such as {\it map}, {\it reduce}, and {\it zip}.
    Finally, the Thread Building Blocks library
    \citep{Reinders2007Intel} implements Slgorithmic Skeletons.

    The definitions for Algorithmic Skeletons are not specified formally to
    enable automated reasoning, but are drafted for human understanding and to
    guide the design of libraries and DSLs.
    This abstraction level is similar to the Berkley Dwarfs.
    However, Algorithmi Skeletons were heavily inspired by higher-order
    functions and similarly describe the algorithmic structure of computations.
    This distinguishes them from the Berkeley Dwarfs, which are more focused on
    mathematical domains and architectural requirements.


\chapter[Introducing the Compiler Analysis Description Language]
        {The Compiler Analysis Description Language
         \footnote{This chapter is based on published research in
                  \citet{Ginsbach:2018:CDS:3178372.3179515}.}}
    \label{chapter:candl}
    
    \Cref{chapter:theory} derived an approach for constraint programming on
    Static Single Assignment (SSA) compiler intermediate representation.
    This chapter develops a novel domain-specific constraint programming
    language based on that methodology and presents an implementation within the
    production-quality LLVM compiler infrastructure.

    In the first sections, the design of the Compiler Analysis
    Description Language (CAnDL) is motivated as an approach for simplifying
    the implementation of LLVM transformation passes.
    Optimising compilers have to use elaborate program transformations to
    exploit increasingly complex hardware.
    Implementing the required analysis functionality for such optimisations to
    be safely applied is a time-consuming and error-prone activity.
    This is a barrier to the rapid prototyping and evaluation of innovative new
    compiler optimisations.
    CAnDL automatically generates such compiler analysis functionality from
    constraint specifications.

    The individual language constructs are introduced with their syntax and
    functionality in the third section of this chapter.
    The first introduced language features directly expose parts of the
    underlying SSA model from \Cref{chapter:theory}.
    Building on that, CAnDL provides higher-level constructs that allow for
    modularity in specifications and the reduction of repetitive constraints.
    Using these higher-level constructs, a collection of CAnDL specifications
    of standard compiler concepts is introduced as the CAnDL standard library.
    This collection of common building blocks includes single-entry single-exit
    regions, loops, and array-based memory accesses.

    Finally, several case studies are presented for the experimental evaluation
    of CAnDL.
    They show that CAnDL scales to a wide range of compiler analysis tasks.
    These tasks range from the detection of peephole optimisation opportunities,
    over graphics shader optimisations, to fully capturing Static Control Parts
    (SCoPs) for polyhedral code analysis.
    All of them can be expressed more succinctly in CAnDL than with previous
    approaches.

\section{Introduction}

    Compilers are intricate pieces of software responsible for the generation of
    efficient code.
    They transform input source code through several compilation stages,
    resulting in a binary program.
    In order to generate fast programs, state-of-the-art compilers rely on an
    elaborate middle end.
    At this stage, the user code is typically expressed in an SSA intermediate
    representation, and improved by successively applying a wide range of
    optimisations.

    Most compiler optimisations require two steps:
    analysis and transformation.
    First, analysis routines find sections in user programs that enable the
    application of specific transformations.
    They further verify the necessary conditions to ensure the transformation
    can be applied legally without changing the program semantics.
    It is crucial that optimisations retain the semantics of the original
    program, as otherwise the resulting binary might be corrupted.
    Transformations are then applied to the analysis results in a second step.
    This often involves heuristic cost models to gauge the effect on runtime,
    code size, and other metrics.

    The complexity of the necessary analysis is an impediment to
    the implementation of new compiler passes, preventing the rapid
    prototyping of new ideas.
    For example, simple peephole optimisations in the LLVM ``{\tt instcombine}''
    pass require approximately 30000 lines of C++ code, despite the
    transformations being simple.
    \citet{Menendez:2017:ADP:3062341.3062372} showed that ``{\tt instcombine}''
    is an important source of bugs, and bugs in the middle end of a compiler are
    particularly pernicious \citep{Yang:2011:FUB:1993316.1993532}.
    They tamper with the user programs but can remain unnoticed and often only
    trigger in corner cases.
    Ideally, there would be a simpler way of implementing such analysis that
    reduces boilerplate code and opens the way for new compiler innovation.

    This chapter presents the Compiler Analysis Description Language (CAnDL), a
    domain-specific language for compiler analysis.
    It is a constraint programming language, operating on the SSA intermediate
    representation of the LLVM compiler infrastructure (LLVM IR).
    Instead of writing compiler analysis code inside the main codebase of the
    compiler infrastructure, it lets compiler writers specify optimisation
    functionality external to the main C++ codebase.
    The CAnDL compiler then generates C++ functions that implement LLVM analysis
    passes, and are linked together with the Clang compiler binary.
    The formulation of optimising transformations in CAnDL is faster, simpler
    and less error-prone than writing them in C++.
    The language has a strong emphasis on modularity, which facilitates
    debugging and the formulation of highly readable code.

    CAnDL is based on the constraint programming methodology introduced in
    \Cref{chapter:theory}.
    It uses a solver that is integrated into the LLVM codebase.
    CAnDL is developed as a complete programming language, with a full parser
    and code generator.
    The system is evaluated on a range of use cases from
    different domains, including standard LLVM optimisation passes,
    custom optimisations for graphics shader programs, and the detection of
    Static Control Parts (SCoPs) \citep{Lengauer2012Polly} for polyhedral
    program transformations \citep{Karp:1967:OCU:321406.321418}.

\section{Motivating Example}

\begin{figure}[b]
\centering
\vspace{3.6mm}
\begin{minipage}[t]{0.68\textwidth}
\begin{lstlisting}[language=CAnDL,label={fig:candlspec},caption=
   {The left side of \Cref{fig:root} as specified in CAnDL}]
Constraint SqrtOfSquare
( opcode{sqrt_call} = call
∧ {sqrt_fn} = {sqrt_call}.args[0]
∧ function_name{sqrt_fn} = sqrt
∧ {square} = {sqrt_call}.args[1]
∧ opcode{square} = fmul
∧ {a} = {square}.args[0]
∧ {a} = {square}.args[1])
End
\end{lstlisting}
\end{minipage}
\end{figure}

    For an example of the CAnDL workflow, consider \Cref{fig:root}.
    This basic algebraic equation can be interpreted as a recipe for a compiler
    optimisation:
    Assuming an environment without the particularities of floating-point
    arithmetic (i.e.\ assuming the ``{\tt-ffast-math}'' flag is active), the
    compiler could use this equality to eliminate some square root invocations
    in user code.
    This is desirable, as the square root has to be approximated with relatively
    expensive numerical methods, whereas computing the absolute value is
    computationally cheap.
    \begin{align}
    \label{fig:root}
    \forall a\in \mathbb{R}\colon\ \sqrt{a*a}=|a|
    \end{align}

    The compiler should use the equation left-to-right.
    It should analyse the user code in order to find segments that correspond to
    the left side of the equation and then transform all those occurrences
    analogous to the right side of the equation.
    The compiler, therefore, must detect $\sqrt{a*a}$ in the LLVM IR code and
    replace it with calls to the ``{\tt abs}'' function.
    The generation of the new function call is trivial, but the detection of
    even this simple pattern requires some care when implemented manually in
    a sophisticated codebase such as LLVM.

    The traditional approach in the Clang compiler is to integrate such
    optimisations into the previously mentioned ``{\tt instcombine}'' pass.
    which already applies an extensive collection of peephole optimisations.
    However, this code makes heavy use of raw pointers and dynamic type casts,
    spans ${\sim}30000$ lines, and has been identified as a frequent source of
    bugs in
    \citet{Menendez:2017:ADP:3062341.3062372,Yang:2011:FUB:1993316.1993532}.
    This is impractical and an impediment to compiler development.

    Instead, CAnDL allows a declarative description of the analysis problem.
    It is easier to follow, has no interaction with other optimisations and
    is concise, as presented in \Cref{fig:candlspec}.
    The first line of the program assigns a name to the specification, which is
    then defined by the interaction of seven {\em atomic constraint}s.
    These individual statements must simultaneously hold on the values of
    ``{\tt sqrt\_call}'', ``{\tt sqrt\_fn}'', ``{\tt square}'' and ``{\tt a}''.
    Lines 2--8 each stipulate one of these constraints, and they are joined
    together with logical conjunctions ``$\land$''.

    The CAnDL compiler translates the declarative program into a C++ function,
    which is then used for the analysis step in an LLVM optimisation pass.
    This is demonstrated in \Cref{fig:candlexample}, which shows the application
    of the analysis function generated from the CAnDL specification in
    \Cref{fig:candlspec} to a user program.
    The input program ({\bf a}) is a simple C function that calls the
    ``{\tt sqrt}'' function twice with squares of floating-point values.
    This is translated using the Clang compiler into LLVM IR code ({\bf b}),
    using standard optimisation passes during the compilation.
    The expression from the user program is here represented as a list of
    individual instructions and register assignments, with the occurrences
    of ``{\tt SqrtOfSquare}'' clearly visible: the two ``{\tt fmul}''
    instructions (lines 4 and 6) compute squares via a floating-point
    multiplication, and these are then used as arguments to ``{\tt sqrt}''
    function invocations (lines 5 and 7).

    The optimised LLVM IR ({\bf b}) is used as the input to the generated
    analysis function, which detects two optimisation opportunities.
    These are identified as the first ({\bf c}) and second ({\bf d}) solution
    of the constraint problem.
    Each of the solutions assigns values from within the LLVM IR code to
    all the CAnDL variables in \Cref{fig:candlspec}, such
    that all constraints are satisfied.
    The validity of these solutions is demonstrated in the middle row of the
    figure ({\bf e}-{\bf f}).
    Substituting the variables in the CAnDL program with the concrete instances
    from the solutions, the individual atomic constraints can be checked
    individually:
    \begin{itemize}
    \item {\tt \%4} and {\tt \%6} are function calls, and their first argument
          (the function to be called) is {\tt @sqrt}.
    \item {\tt @sqrt} is the square root function.
          Note that it is identified by name.
    \item The second arguments of the function call instructions (the first
          function arguments) are {\tt \%3} and {\tt \%5}, respectively.
    \item {\tt \%3} and {\tt \%5} are square values, i.e.\ floating-point
          multiplications of a value with itself.
    \end{itemize}

    Using the solutions identified by the CAnDL system, a separate C++ function
    ({\bf g}) applies the transformation.
    This function is easy to implement.
    The solutions to the constraint problem are internally provided as C++
    dictionaries of type
    ``\lstinline[language=MyCpp]!map<std::string,llvm::Value*>!'', containing
    the information needed to apply code transformations.
    A new function call to ``{\tt abs}'' is generated, with the value
    determined for ``{\tt a}'' from \Cref{fig:candlspec} as the only argument
    (line 6).
    This instruction replaces (line 4) the call instruction that was captured
    in ``{\tt sqrt\_call}'' (line 5).
    Conveniently, the LLVM infrastructure already provides all the necessary
    functions to create and replace instructions in the intermediate
    representation.
    After post-processing with standard dead code elimination, this results in
    the optimised code shown at the bottom of the figure ({\bf h}).

    Although this is a small example, it illustrates the main steps of the
    CAnDL scheme.
    In practice, the strength of the system is its ability to scale to very
    complex specifications, as demonstrated toward the end of the chapter.
    The following sections describe the CAnDL language in detail and outline
    how it is implemented on top of the constraint programming methodology
    from \Cref{chapter:theory}.

\begin{figure}[p]
    \centering
\begin{minipage}[t]{\textwidth}
\centering
\begin{minipage}[t]{\textwidth}
\centering
\vspace{-1em}
\begin{lstlisting}[language=C,basicstyle=\small\ttfamily,
                   numbers=none,framexleftmargin=0pt,xleftmargin=0pt,
                   captionpos=t,title={{\bf(a)} C program code:}]
 double example(double a, double b) {return sqrt(a*a) + sqrt(b*b); }
\end{lstlisting}
\end{minipage}
\begin{minipage}[t]{7.1cm}
\centering
\vspace{-0.19cm}
\begin{lstlisting}[language={LLVM},
                   captionpos=t,title={{\bf(b)} Resulting LLVM IR:}]
define double @example(    
 double %0,                
 double %1) {              
 %3 = fmul double %0, %0   
 %4 = call double @sqrt(%3)
 %5 = fmul double %1, %1   
 %6 = call double @sqrt(%5)
 %7 = fadd double %4, %6   
 ret double %7 }
declare double @sqrt(double)      
\end{lstlisting}
\end{minipage}
\hfill
\begin{minipage}[t]{3.5cm}
\centering
\vspace{-0.19cm}
\begin{lstlisting}[language=LLVM,
                   numbers=none,framexleftmargin=0pt,xleftmargin=0pt,
                   captionpos=t,title={{\bf(c)} First solution:}]

a = %0

square = %3
sqrt_call = %4 




sqrt_fn = @sqrt
\end{lstlisting}
\end{minipage}
\hfill
\begin{minipage}[t]{3.5cm}
\centering
\vspace{-0.19cm}
\begin{lstlisting}[language=LLVM,
                   numbers=none,framexleftmargin=0pt,xleftmargin=0pt,
                   captionpos=t,title={{\bf(d)} Second solution:}]


a = %1


square = %5
sqrt_call = %6


sqrt_fn = @sqrt
\end{lstlisting}
\end{minipage}
\end{minipage}

\begin{minipage}[t]{\textwidth}
\begin{minipage}[t]{0.493\textwidth}
\centering
\vspace{-0.19cm}
\begin{lstlisting}[language=CAnDL,
                   numbers=none,framexleftmargin=0pt,xleftmargin=0pt,
                   captionpos=t,title={{\bf(e)} Verification of first solution:}]
( opcode{%4} = call
([$\tt \land$]) {@sqrt} = {%4}.args[0]
([$\tt \land$]) function_name{@sqrt} = sqrt
([$\tt \land$]) {%3} = {%4}.args[1]
([$\tt \land$]) opcode{%3} = fmul
([$\tt \land$]) {%0} = {%3}.args[0]
([$\tt \land$]) {%0} = {%3}.args[1])
\end{lstlisting}
\end{minipage}
\hfill
\begin{minipage}[t]{0.493\textwidth}
\centering
\vspace{-0.19cm}
\begin{lstlisting}[language=CAnDL,
                   numbers=none,framexleftmargin=0pt,xleftmargin=0pt,
                   captionpos=t,title=
   {{\bf(f)} Verification of second solution:}]
( opcode{%6} = call
([$\tt \land$]) {@sqrt} = {%6}.args[0]
([$\tt \land$]) function_name{@sqrt} = sqrt
([$\tt \land$]) {%5} = {%4}.args[1]
([$\tt \land$]) opcode{%5} = fmul
([$\tt \land$]) {%1} = {%5}.args[0]
([$\tt \land$]) {%1} = {%5}.args[1])
\end{lstlisting}
\end{minipage}
\end{minipage}

\begin{minipage}[t]{\textwidth}
\centering
\vspace{-0.19cm}
\begin{lstlisting}[captionpos=t,title={{\bf(g)} C++ transformation code:}]
using namespace std;
using namespace llvm;
void transform(map<string,Value*> solution, Function* abs) {
    ReplaceInstWithInst(
       dyn_cast<Instruction>(solution["sqrt_call"]),
       CallInst::Create(abs, {solution["a"]}));
}
\end{lstlisting}
\end{minipage}

\begin{minipage}[t]{\textwidth}
\centering
\vspace{-0.19cm}
\begin{lstlisting}[captionpos=t,title=
   {{\bf(h)} Transformed LLVM IR after dead code elimination:}]
define double @example(double %0, double %1) {              
 %3 = call double @abs(double %0) 
 %4 = call double @abs(double %1)
 %5 = fadd double %3, %4   
 ret double %5 }
\end{lstlisting}
\end{minipage}

\caption{Demonstration of CAnDL specification in \Cref{fig:candlspec}
         on an example C program ({\bf a}):
         In the generated LLVM IR code ({\bf b}), two instances
         ({\bf c},{\bf d}) of {\tt SqrtOfSquare} are detected that fulfil all
         the constraints ({\bf e}, {\bf f}).
         Applying a transformation is simple ({\bf g}) and results in efficient
         code ({\bf h}).\parfillskip=0pt}
    \label{fig:candlexample}
\end{figure}

\section{Language Specification}

    The Compiler Analysis Description Language is a domain-specific
    programming language for the specification of compiler analysis problems. 
    Individual CAnDL programs define specific computational structures that
    exist in user programs and can be exploited by applying code
    transformations.
    These structures are specified as constraint programs on the LLVM IR
    of user code.
    CAnDL builds on generic concepts from \Cref{chapter:theory}.
    These are independent of LLVM, so the methodology translates to other SSA
    representations.

    The expressed structures can scale from simple instruction patterns that
    enable peephole optimisations, over control flow structures such as loops,
    to complex algorithmic concepts such as code regions that are suitable for
    polyhedral code transformations.

    Like traditional constraint programs, CAnDL specifications have two
    fundamental features: \textbf{variables} and \textbf{constraints}.
    The basic constraint building blocks are well-established compiler analysis
    tools, such as constraints on data and control flow, data types and
    instruction opcodes.
    These are composed with logical connectors and several higher-level language
    features, such as range expressions, with finally a system of modularity and
    extensibility on top.
    This section introduces the language features, starting from the overall
    program structure.\footnote{The complete grammar file that was used to
    generate the parser of the CAnDL compiler is in Appendix
    \ref{appendix:CAnDLgrammar}.}

\subsection{Top-Level Structure of CAnDL Programs}

    The following notational conventions are used for the description of CAnDL
    syntax in this section:
    terminal symbols are {\bf bold}, non-terminals are {\it italic},
    $\left<\text{\bf s}\right>$ is an identifier (alphanumeric string), and
    $\left<\text{\bf n}\right>$ is an integer literal.
    CAnDL uses Unicode characters such as ``$\land$'', ``$\in$'', ``$\Phi$'' and
    is encoded as UTF-8.
    An individual CAnDL program contains constraint formulas that are
    bound to identifiers.
    As previously shown in \Cref{fig:candlspec}, the syntax for this is as
    follows:
\begin{figure}[H]
\centering
\begin{tabular}{|c|}
    \hline
    $\text{\it specification} ::= \text{\bf Constraint}\ \left<\text{\bf s}\right>\ \text{\it formula}\ \text{\bf End}$\\
    \hline
\end{tabular}
\end{figure}

    \Cref{fig:candlspec} already demonstrated how logical conjunctions are used
    to combine simpler {\it formula}s.
    More generally, a {\it formula} can be any of the following:
\begin{figure}[H]
\centering
\begin{tabular}{|c|}
    \hline
    $\text{\it formula} ::= \text{\it atomic}\mid\text{\it conjunction}\mid\text{\it disjunction}\mid\text{\it conRange}\mid \text{\it disRange}\mid\text{\it include}\mid\text{\it collect}$\\
    \hline
\end{tabular}
\end{figure}

    \noindent
    The fundamental elements of every CAnDL program are {\it atomic}
    constraints.
    They are bound together by logical connectives ``$\land$''
    and ``$\lor$'' ({\it conjunction} and {\it disjunction}), as well as other
    higher-level constructs.
    These include two kinds of range structures ({\it conRange},
    {\it disRange}), and a system for modularity ({\it include}).
    Lastly, the {\it collect} construct allows for the formulation of more
    complex constraints that require quantifiers from first-order logic.
    The individual classes of atomic constraints are introduced next,
    followed by the higher-level constructs.

\subsection{Atomic Constraints}

\begin{table}[t]
  \centering
  \definecolor{tableShade}{gray}{0.8}
  \rowcolors{1}{}{tableShade}
  \begin{tabular}{cc}
    \toprule
    {\bf Syntax} & {\bf SSA model formulation} \\
    \midrule
    $\text{\bf data\_type}\ \text{\it variable}\ \text{\bf =}\ \left<\text{\bf s}\right>$ &  $(s,x)\in T_\mathcal F$\\
    $\text{\bf opcode}\ \text{\it variable}\ \text{\bf =}\ \left<\text{\bf s}\right>$ & $(s,x)\in I_\mathcal F$\\
    $\text{\bf ir\_type}\ \text{\it variable}\ \text{\bf =}\ \text{\bf literal}$ &  $x\in C_\mathcal F^*$\\
    $\text{\bf ir\_type}\ \text{\it variable}\ \text{\bf =}\ \text{\bf argument}$ & $x\in P_\mathcal F^*$\\
    $\text{\bf ir\_type}\ \text{\it variable}\ \text{\bf =}\ \text{\bf instruction}$ & $x\in I_\mathcal F^*$\\
    $\text{\bf function\_name}\ \text{\it variable}\ \text{\bf =}\ \left<\text{\bf s}\right>$ & $(s,x)\in G_\mathcal F$\\
    \bottomrule
  \end{tabular}
  \caption{The simplest atomic constraints operate on a single variable and
           check element-of properties for the different sets in the SSA model.
           Function names are from the global model.}
  \label{onevaratomics}

  \vspace{3.7mm}
  \definecolor{tableShade}{gray}{0.8}
  \rowcolors{1}{}{tableShade}
  \begin{tabular}{cc}
    \toprule
    {\bf Syntax} & {\bf SSA model formulation} \\
    \midrule
    $\text{\it variable}\ \text{\bf =}\ \text{\it variable}\text{\bf.args[}\left<\text{\bf n}\right>\text{\bf]}$ & $(n,x,y)\in DFG_\mathcal F$\\
    $\text{\it variable}\ \text{\bf $\in$}\ \text{\it variable}\text{\bf .args}$ & $(x,y)\in DFG_\mathcal F^*$\\
    $\text{\it variable}\ \text{\bf =}\ \text{\it variable}\text{\bf.successors[}\left<\text{\bf n}\right>\text{\bf]}$ & $(n,y,x)\in CFG_\mathcal F$\\
    $\text{\it variable}\ \text{\bf $\in$}\ \text{\it variable}\text{\bf .successors}$ & $(y,x)\in CFG_\mathcal F^*$\\
    $\text{\it variable}\ \text{\bf =}\ \text{\it variable}$ & $x=y$\\
    $\text{\it variable}\ \text{\bf !=}\ \text{\it variable}$ & $x\neq y$\\
    \bottomrule
  \end{tabular}
  \caption{The second class of atomic constraints operate on pairs of variables.
           The constraints check for graph edges in the data flow or control
           flow graphs, or directly for shallow equivalence.}
  \label{twovaratomics}
\end{table}

    Based on the SSA model from \Cref{def:ssamodel} in \Cref{chapter:theory},
    CAnDL provides a wide range of atomic constraints.
    The simplest group consists of those that operate on only a single variable,
    listed in \Cref{onevaratomics}.
    These operate immediately on the underlying mathematical structures, testing
    element-of properties between variables and the sets.
    This can constrain data types ($T_\mathcal{F}$) and 
    instruction opcodes ($I_\mathcal{F}$), or 
    restrict variables to constant literals ($C_\mathcal{F}^*$), function
    parameters ($P_\mathcal{F}^*$), and instructions ($I_\mathcal{F}^*$).
    Ending the list is the ``{\bf function\_name}'' constraint.
    Function names can be statically determined only for direct function calls.
    In that case, the called object is necessarily a global value, and the
    function name can be identified in the global model $G_\mathcal F$.

    There are additional atomic constraints that operate on pairs of variables.
    These are listed in \Cref{twovaratomics}.
    Most importantly, they check for specific edges in the control flow
    ($CFG_\mathcal F$) and data flow graphs ($DFG_\mathcal F$).
    CAnDL also provides weaker versions that do not enforce specific edge
    labels.
    Furthermore, two constraints are available for shallow comparisons of
    variables.

    Besides those constraints that operate immediately on the sets of the
    SSA model, there are atomic constraints that enforce graph properties.
    Such graph properties include dominance relationships and the interaction of
    data flow and control flow for $\Phi$-instructions.

\subsubsection{Constraining $\Phi$-Instructions}

    CAnDL provides the following syntax for expressing the data flow
    of $\Phi$-instructions:
\begin{figure}[h]
    \centering
    \begin{tabular}{|c|}
        \hline
        $\text{\it variable}\ \text{\bf ->}\ \text{\it variable}\ \Phi\ \text{\it variable}$\\
        \hline
    \end{tabular}
\end{figure}

    \noindent
    The expression $\{A\}\text{\bf ->}\{B\}\Phi\{C\}$ means that
    $C$ takes the value of $A$ when reached from $B$.
    The underlying condition on the SSA model is more difficult to express than
    for the previous atomic constraints.
    Specifically, ``reached from $B$'' means that $B$ was the last branch
    taken before arriving at $C$.
    Using the SSA model, this is equivalent to the following:
    \begin{align*}
        (C,phi)\in I_\mathcal F\mathrel\land{}&(B,h(C))\in CFG_\mathcal F^*\mathrel\land(A,C,n)\in DFG_\mathcal F,\\
            \text{where }h(C):={}&min\{c\mid (phi,x)\in I_\mathcal F\text{ for all }c\leq x\leq C\}\\
            \text{and }n:={}&\{b\leq B\mid (b,h(C))\in CFG_\mathcal F^*\}.
    \end{align*}
    Firstly, $C$ is a $\Phi$-instruction.
    Secondly, $B$ has control flow to the basic block that contains $C$.
    The first instruction of this basic block is identified as $h(C)$, to
    account for the possibility of more than one $\Phi$-instruction in the
    basic block.
    Thirdly, $A$ is the $n$th argument of $C$, where $n$ is the index of $B$ in
    the list of jump instructions that target $h(C)$.

\subsubsection{Identifying Graph Dominators}

    In order to express domination in the control flow graph
    \citep{Cytron:1989:AGD:73141.74823}, the following syntax is used:
\begin{figure}[h]
    \centering
    \begin{tabular}{|c|}
        \hline
        $\text{\bf domination(}\text{\it variable}\text{\bf,} \text{\it variable}\text{\bf)}$\\
        $\text{\bf strict\_domination(}\text{\it variable}\text{\bf,} \text{\it variable}\text{\bf)}$\\
        \hline
    \end{tabular}
\end{figure}

    \noindent
    For both these constraints, the values are implicitly limited to
    instructions.
    The expression ``{\bf domination}(\{A\},\{B\})'' means that $A$ is a
    dominator of $B$, i.e.\ any path through the control flow graph from the
    entry node to $B$ must go through $A$.
    Strict domination additionally requires that $A$ and $B$ are distinct.
    Complementing these constructs, there are post-dominator versions
    ``{\bf post\_domination}'' and ``{\bf strict\_post\_domination}'', as well
    as the following generalisation:
\begin{figure}[h]
    \centering
    \begin{tabular}{|c|}
        \hline
        ${\bf all}\ {\bf control}\ {\bf flow}\ {\bf from}\ variable\ {\bf to}\ variable\ {\bf passes}\ {\bf through}\ variable$\\
        \hline
    \end{tabular}
\end{figure}

    \noindent
    This constraint is similar to a standard control flow domination, but
    instead of taking paths from the control flow origin, a third variable is
    used for parametrisation, as required for \Cref{CAnDLSESE}.

\subsubsection{Additional Atomic Constraints}

    The set of atomic constraints that are supported by CAnDL can easily be
    extended.
    Possible additions include constraints on function attributes and value
    constraints on literals.
    This will be further explored in \Cref{chapter:reductions,chapter:idioms}.

\subsection{Range Constraints}

\begin{figure}[t]
\vspace{3.6mm}
\begin{lstlisting}[language=CAnDL]
Constraint ValueChain
 {element[i] ∈ {element[i+1]}.args foreach i=0..4
End
\end{lstlisting}
\begin{lstlisting}[language=CAnDL,label={fig:forall},caption=
   {Example for the expansion of range constraints in CAnDL:
    The specification at the top can be ``unrolled'' manually, resulting in the
    equivalent, but more verbose, specification below.}]
Constraint ValueChain
( {element[0]} ∈ {element[1]}.args
∧ {element[1]} ∈ {element[2]}.args
∧ {element[2]} ∈ {element[3]}.args
∧ {element[3]} ∈ {element[4]}.args)
End
\end{lstlisting}
\end{figure}

    Building on top of the atomic constraints and the fundamental conjunction
    and disjunction constructs, there are range based constraints that operate
    on arrays of variables:
\begin{figure}[h]
  \centering
  \begin{tabular}{|c|}
    \hline
    $conRange ::= \text{\it formula}\ \text{\bf foreach}\ \left<\text{\bf s}\right>\ \text{\bf =}\ \text{\it index}\ \text{\bf ..}\ \text{\it index}$\\
    $disRange ::= \text{\it formula}\ \text{\bf forany}\ \left<\text{\bf s}\right>\ \text{\bf =}\ \text{\it index}\ \text{\bf ..}\ \text{\it index}$\\
    \hline
  \end{tabular}
\end{figure}

    \noindent
    These constructs allow the replication of a constraint formula over a range
    of indices.
    This is demonstrated in \Cref{fig:forall}, which shows two equivalent
    CAnDL programs, the first one formulated with {\it conRange} and the
    second one without.
    In both cases, the program specifies an array of five variables with data
    flow from each element to the next.
    This shows how {\it conRange} can be expanded by duplicating the contained
    formula, with logical conjunctions binding the replicas of the
    formula together.
    Otherwise identical, the {\it disRange} construct is based on logical
    disjunctions instead.

    The syntactic structure of variable identifiers carries no semantic
    information for atomic constraints.
    Clearly, this is not true for range expressions, which rely on index
    calculations in order to evaluate the underlying variables.
    Therefore, it is important to introduce next the precise syntax for variable
    names, which are constructed as follows:

\begin{figure}[h]
  \centering
  \begin{tabular}{|c|}
    \hline
    $variable ::= \left<\text{\bf s}\right>\mid\text{\it variable}\ \text{\bf [}\ \text{\it calculation}\ \text{\bf ]}\mid\text{\it variable}\ \text{\bf.}\ \left<\text{\bf s}\right>$\\
    $calculation ::= \left<\text{\bf s}\right>\mid\left<\text{\bf n}\right>\mid\text{\it calculation}\ \text{\bf+}\ \text{\it calculation}\mid\text{\it calculation}\ \text{\bf-}\ \text{\it calculation}$\\
    \hline
  \end{tabular}
\end{figure}

    \noindent
    The syntax of variable identifiers in CAnDL aligns closely to C/C++
    conventions.
    They can contain simple index calculations to support the range constructs,
    as well as a hierarchical structure.
    This hierarchical structure corresponds to the modularity capabilities of
    CAnDL that are introduced in the next section.

\subsection{Modularity}

    Modularity is central to CAnDL, and it is achieved using the {\it include}
    construct.
    \begin{figure}[h]
        \centering
        \begin{tabular}{|c|}
            \hline
            $\text{\bf include}\ \left<\text{\bf s}\right>
                                [\ \text{\bf (}\text{\it variable}\ \text{\bf ->}\ \text{\it variable}\ \{\ \text{\bf ,}\ \text{\it variable}\ \text{\bf ->}\ \text{\it variable}\ \}\ \text{\bf )}\ ]\ 
                                [\ \text{\bf @}\ \text{\it variable}\ ]$\\
            \hline
        \end{tabular}
    \end{figure}

    \noindent
    Note that the syntax in square brackets  is optional and the syntax in curly
    brackets may be repeated.
    The basic version of {\it include}, using neither of the two optional
    structures, is simple.
    It copies the formula that corresponds to the identifier verbatim into
    the current specification.
    If ``$[\text{\bf @}\text{\it variable}]$'' is specified, then all the
    variable names of the inserted constraint formula are prefixed with the
    given variable name, separated by a dot.
    This namespaces the inserted formula and prevents unwanted
    interactions with surrounding constraints.
    The other optional syntax is used to rename variables in the included
    formula.
    In contrast to the prefix syntax, this increases the interaction with
    surrounding constraints by injecting other variables.
    Accordingly, if both optional constructs are used, the prefix is only
    applied to variables that are not renamed.

    \Cref{fig:inheritsandrenameandrebase} illustrates this with two
    equivalent constraint programs.
    Both programs specify an addition of four values, first adding pairwise and
    then adding the intermediate results.
    In the code at the top, a formula for the addition of two values is
    bound to the name ``{\tt Sum}''.
    This is then included three times in another formula named
    ``{\tt SumOfSums}''.
    Using the optional grammatical constructs, the formula operates on a
    different set of variables each time, such that the third addition
    takes the results of the previous two as input.

\begin{figure}[h]
\begin{lstlisting}[language=CAnDL]
Constraint Sum
( opcode{out} = add
([$\tt \land$]) {in1} = {out}.args[0]
([$\tt \land$]) {in2} = {out}.args[1])
End
Constraint SumOfSums
( include Sum@{sum1}
([$\tt \land$]) include Sum@{sum2}
([$\tt \land$]) include Sum({sum1.out}->{in1},{sum2.out}->{in2}))
End
\end{lstlisting}
\begin{lstlisting}[language=CAnDL,
                   label={fig:inheritsandrenameandrebase},caption=
   {Example for the expansion of {\it include} constraints:
    Both specifications are equivalent.\parfillskip=0pt}]
Constraint SumOfSums
( opcode{sum1.out} = add
([$\tt \land$]) {sum1.in1} = {sum1.out}.args[0]
([$\tt \land$]) {sum1.in2} = {sum1.out}.args[1]
([$\tt \land$]) opcode{sum2.out} = add
([$\tt \land$]) {sum2.in1} = {sum2.out}.args[0]
([$\tt \land$]) {sum2.in2} = {sum2.out}.args[1]
([$\tt \land$]) opcode{out} = add
([$\tt \land$]) {sum1.out} = {out}.args[0]
([$\tt \land$]) {sum2.out} = {out}.args[1])
End
\end{lstlisting}
\end{figure}

\subsubsection{Collect-all Constraints}

    The \text{\it collect} construct captures all possible solutions of a given
    formula, as in \Cref{def:collectconstr}.
\begin{figure}[H]
    \centering
    \begin{tabular}{|c|}
        \hline
        $\text{\bf collect}\ \left<\text{\bf s}\right>\ \text{\it index}\ \text{\it formula}$\\
        \hline
    \end{tabular}
\end{figure}

    \noindent
    In \Cref{fig:simplecollect}, the variables
    ``{\tt arg[0]}'',\dots,``{\tt arg[7]}''
    are specified to contain all of the direct data dependencies of
    ``{\tt ins}''.
    In this example, solutions of ``{\tt arg[i]}'' for a given value
    of ``{\tt ins}'' are identified.
    The second argument gives an upper limit to the number of collected
    variables.
    The constant upper bound ``\texttt{8}'' is required here to keep the
    dimensionality of the search space finite.
    If ``\texttt{ins}'' has less than 8 arguments, the remaining
    ``\texttt{arg[]}'' are constrainted to \textit{unused}.
    The first argument of {\it collect} specifies the name of an index
    variable that is used to detect which variables belong to the collected set.

\begin{figure}[t]
\vspace{3.6mm}
\begin{lstlisting}[language=CAnDL,label={fig:simplecollect},caption=
   {Simple {\it collect} example in CAnDL: Direct data dependencies of
    ``{\tt ins}'' are collected.\leftskip=0pt\rightskip=0pt}]
Constraint CollectArguments
( ir_type{ins} = instruction
∧ collect i 8 ({arg[i]} ∈ {ins}.args))
End
\end{lstlisting}
\end{figure}

    This example is now extended to show how {\it collect} can be used to
    implement quantifiers.
    Consider the task of detecting instructions with only floating-point
    arguments.
    This involves the ``$\forall$'' quantifier, as it is equivalent to
    the following equation:
    \begin{align}
        \texttt{ins}\text{ has only }\textit{float}\text{ arguments}\iff
        \left[\forall x\colon\ (x,\texttt{ins})\in DFG_\mathcal F^*\implies(\textit{float},x)\in I_\mathcal F\right]
        \label{fig:implication}
    \end{align}
    This can be rewritten to an equivalent formulation on sets:
    \begin{align*}
        S_1= \textit{select}(\texttt{ins},\textit{rev}(DFG_\mathcal F^*))\subset\textit{select}(\textit{float},T_\mathcal F)= S_2.
    \end{align*}
    
    Elementary set theory gives  $S_1\subseteq S_2\iff S_1 = S_1\cap S_2$.
    Therefore, if a set $S$ is constrained to be equal to both $S_1$ and
    $S_1\cap S_2$, then \Cref{fig:implication} gives that this is satisfiable if
    and only if ``{\tt ins}'' has only floating-point arguments.
    This condition can be expressed in CAnDL, as is shown in
    \Cref{fig:collectexample}.
    With the first {\it collect} statement at line 3, the set ``{\tt arg}''
    is constrained to be equal to $S_1$ and with the second one at lines 4-5, it
    is constrained to be equal to $S_1\cap S_2$.

\begin{figure}[H]
\vspace{3.6mm}
\begin{lstlisting}[language=CAnDL,label={fig:collectexample},caption=
   {{\it Collect} restricts ``{\tt ins}'' to instructions with only
    (up to 8) floating-point operands.\leftskip=0pt\rightskip=0pt}]
Constraint FloatingPointInstruction
( ir_type{ins} = instruction
∧ collect i 8 ( {ins} ∈ {arg[i]}.args)
∧ collect i 8 ( {ins} ∈ {arg[i]}.args
              ∧ data_type{arg[i]} = float))
End
\end{lstlisting}
\end{figure}

    The exact same approach can be used for much more complex analysis tasks.
    For example, the final case study at the end of this chapter uses
    {\it collect} statements to restrict all array accesses within a
    loop to be affine in the loop iterators.
    First, {\it collect} all memory accesses in the loop
    (i.e.\ all ``{\tt load}'' and ``{\tt store}'' instructions), and then use a
    second {\it collect} statement to enforce affine calculations of the access
    indices that are used for these instructions.

\subsection{Expressing Larger Structures}

\begin{figure}[p]
\lstset {
 basicstyle=\linespread{0.910}\ttfamily
}
\vspace{3mm}
\begin{lstlisting}[language=CAnDL, label=CAnDLSESE, caption=
   {Single-entry single-exit region in CAnDL:
    The region spans ``{\tt begin}'' to ``{\tt end}'', with control flow
    ``{\tt precursor}'' to ``{\tt begin}'' as the entry, control flow
    ``{\tt end}'' to ``{\tt successor}'' as the exit.}]
Constraint SESE
( opcode{precursor} = branch
∧ {begin} ∈ {precursor}.successors
∧ opcode{end} = branch
∧ {successor} ∈ {end}.successors
∧ domination({begin}, {end})
∧ post_domination({end}, {begin})
∧ strict_domination({precursor}, {begin})
∧ strict_post_domination({successor}, {end})
∧ all control flow from {begin} to {precursor}
         passes through {end}
∧ all control flow from {successor} to {end}
         passes through {begin}) End
\end{lstlisting}
\begin{lstlisting}[language=CAnDL, label=CAnDLLoop, caption=
   {Loops are defined in CAnDL as single-entry single-exit regions with a back
    edge.
    The back edge does not break the ``single-exit'' condition because it does
    not exit the region.}]
Constraint Loop
( include SESE
∧  {begin} ∈ {end}.successors) End
\end{lstlisting}
\begin{lstlisting}[language=CAnDL, label=localconstant, caption=
   {This specification restricts ``{\tt value}'' to remain constant during loop
    execution.
    It is underspecified on its own and should be included in larger CAnDL
    programs that also include ``{\tt Loop}'', renaming ``{\tt value}'' to the
    specific variable that needs to be constrained in this way.}]
Constraint LocalConst
( ir_type{value} = literal
∨ ir_type{value} = argument
∨ strict_domination({value}, {scope.begin})) End
\end{lstlisting}
\begin{lstlisting}[language=CAnDL, label={fig:arrayloop}, caption=
   {Building blocks are combined to restrict the permitted memory access within
    a loop.\leftskip=0pt\rightskip=0pt}]
Constraint PointerAccess
( ( (opcode{access} = store ∧ {pointer} = {access}.args[1])
  ∨ (opcode{access} = load  ∧ {pointer} = {access}.args[0]))
∧ domination({scope.begin}, {access})
∧ post_domination({scope.end}, {access})) End

Constraint ArrayAccess
( include PointerAccess
∧ opcode{pointer} = gep
∧ {base_pointer} = {pointer}.args[0]
∧ include LocalConst({base_pointer}->{value})) End

Constraint LoopWithOnlyArrayAccesses
( include Loop @ {loop}
∧ collect i N (include PointerAccess({loop}->{scope})
                                           @{acc[i]})
∧ collect i N (include ArrayAccess  ({loop}->{scope})
                                           @{acc[i]})) End
\end{lstlisting}
\end{figure}

    The modularity of CAnDL allows for the creation of building blocks that are
    used by multiple CAnDL specifications.
    This includes classic control flow structures, such as single-entry
    single-exit (SESE) regions and different classes of loops, as well as memory
    access patterns.
    These definitions a standard library that enables higher-level programming
    with CAnDL.
    This section gives an overview of how some of these standard building
    blocks are defined.

    \Cref{CAnDLSESE} gives the specification of SESE regions in CAnDL.
    Such regions are spanned by the variables ``{\tt begin}'' and ``{\tt end}'',
    with an incoming control flow edge from ``{\tt precursor}'' to
    ``{\tt begin}'' (line 3) and
    an outgoing control flow edge from ``{\tt end}'' to ``{\tt successor}''
    (line 5).
    Graph domination constraints at lines 8 and 9 guarantee that these are the
    only entry and exit of the region.
    Additional domination constraints at lines 6--7 and lines 10--13 make sure
    that there are no jumps from outside the region into an instruction of the
    region that is not ``{\tt begin}'', and similarly that there are no early
    exits of the region.
    Finally, regions are restricted to align with basic block boundaries in
    lines 2 and 4.

    Note that the use of variables ``{\tt precursor}'' and ``{\tt successor}''
    prevents the detection of SESE regions without a precursor or successor.
    These two special cases could easily be captured with additional constraints
    that account for the possibility of an {\it unused} successor or precursor.
    However, this situation is only relevant in the case of basic blocks that end
    with a return statement or that are the entry point of a function.
    In the context of the loop-based algorithmic structures that this
    thesis focuses on, basic blocks can be assumed to have a precursor and
    successor.

    Natural loops are easily defined in CAnDL, as shown in \Cref{CAnDLLoop}.
    The specification includes ``{\tt SESE}'' and adds only a single additional
    constraint for the back edge of the loop at line 3.
    Additional extensions to this specification are added to define more
    restricted loop structures, such as for-loops.
    This involves the identification of the loop iterator, the breaking
    condition, and the corresponding iteration space boundaries.
    In order to be a valid for-loop, the end of the iteration space must be
    determined before the loop is entered.
    This is expressed in \Cref{localconstant}, restricting ``{\tt value}'' to be
    either a function argument, a constant, or an instruction that strictly
    dominates the loop entry.
    Note that this formula is underspecified on its own.
    In order to get a useful constraint specification, it has to be included in
    larger CAnDL programs that also include ``{\tt Loop}''.

    Another class of important building blocks is different categories of
    memory access.
    These form a hierarchy of restrictiveness and include multi-dimensional
    array access and array access that is affine in some loop iterators.
    \Cref{fig:arrayloop} combines several of the introduced building blocks to
    specify a loop in which all memory access locations are calculated as
    offsets from base pointers that are constant within the loop.
    This, for example, excludes loops with pointer-chasing.
    The ``{\tt ArrayAccess}'' specification can be extended with more restrictive
    memory access patterns as previously mentioned.
    This is crucial to define advanced compiler analysis passes, such as the
    detection of regions suitable for polyhedral code analysis.

\section{Implementation}

    CAnDL is integrated into the LLVM framework.
    CAnDL programs are read by the CAnDL compiler during LLVM build time, which
    then generates C++ source code to implement the specified LLVM analysis
    functionality.
    This code depends on the generic backtracking solver derived in
    \Cref{chapter:theory}, which is incorporated directly into the LLVM
    codebase.
    The generated code is compiled and linked together with the existing LLVM
    libraries.
    The Clang compiler, which is built on LLVM, then automatically invokes this
    solver during the compilation of user programs, after the optimisation
    passes.

    The resulting Clang binary, built on the modified version LLVM, uses the
    solver to search for the specified computational structures and outputs the
    found instances into report files.
    It also makes the results available to ensuing transformation passes in the
    form of C++ structures.

\subsection{Normalisation of LLVM IR}

    For reliable detection of code structures, the normalisation of LLVM IR via
    optimisation passes is critical.
    The promotion of memory to registers, loop-independent code motion,
    constant folding, loop normalisations, inlining, and other standard
    transformations result in predictable code structures that simplify the
    formulation of CAnDL specifications.

    For example, elements of C++ \texttt{vector}s are accessed via the
    overloaded \texttt{operator[]}.
    Such operators appear in LLVM IR as opaque function calls and potentially
    hinder analysis with CAnDL specifications.
    However, these operators are inlined during optimisation, and parts of the
    resulting code are propagated out of loops by loop-independent code motion.
    The remaining array access can easily be captured by CAnDL using the same
    specifications that treat C code.
    \autoref{LLVMnormalisation} demonstrates this normalising effect of
    optimisations.
    The top of the figure shows the C++ implementation of a function that
    computes a reduction over two sequences of integers.
    The first sequence is stored in a plain C array, whereas the second is in
    an instance of the \texttt{vector} class.
    In the optimised LLVM IR code, the difference between these implementations
    has disappeared from the reduction loops at lines 24--32 and 39--47.

\begin{figure}[p]
\lstset{
 basicstyle=\linespread{0.739}\ttfamily
}
\begin{lstlisting}[language=MyCpp]
int reduce(int* input1, size_t input1_size,
           std::vector<int> input2) {
    int result1 = 0, result2 = 0;
    for(size_t i = 0; i != input1_size; i++)
        result1 += input1[i];
    for(size_t i = 0; i != input2.size(); i++)
        result2 += input2[i];
    return result1 + result2;
}
\end{lstlisting}
\begin{lstlisting}[language=LLVM,label={LLVMnormalisation},caption={
    Standard LLVM optimisations as normalising passes: The same computation is
    expressed with a plain C array and a C++ \texttt{std::vector} respectively.
    During compilation, the interaction of function inlining and loop-independent
    code motion results in equivalent intermediate representation code for the
    reduction loops at lines 24--32 and at lines 39--47.}]
define i32 @_Z23reducePimSt6vectorIiSaIiEE(i32*, i64,
                %"class.std::vector"* ) {
  %4 = icmp eq i64 %1, 0
  br i1 %4, label %5, label %17

; <label>:5:
  %6 = phi i32 [ 0, %3 ], [ %22, %17 ]
  %7 = getelementptr inbounds %"class.std::vector",
                %"class.std::vector"* %2,
                i64 0, i32 0, i32 0, i32 0, i32 1
  %8 = bitcast i32** %7 to i64*
  %9 = load i64, i64* %8, align 8
  %10 = bitcast %"class.std::vector"* %2 to i64*
  %11 = load i64, i64* %1
  %12 = icmp eq i64 %9, %11
  %13 = inttoptr i64 %11 to i32*
  br i1 %12, label %25, label %14

; <label>:14:
  %15 = sub i64 %9, %11
  %16 = ashr exact i64 %15, 2
  br label %28

; <label>:17:
  %18 = phi i64 [ %23, %17 ], [ 0, %3 ]
  %19 = phi i32 [ %22, %17 ], [ 0, %3 ]
  %20 = getelementptr inbounds i32, i32* %0, i64 %18
  %21 = load i32, i32* %20
  %22 = add nsw i32 %21, %19
  %23 = add nuw i64 %18, 1
  %24 = icmp eq i64 %23, %1
  br i1 %24, label %5, label %17

; <label>:25:
  %26 = phi i32 [ 0, %5 ], [ %33, %28 ]
  %27 = add nsw i32 %26, %6
  ret i32 %27

; <label>:28:
  %29 = phi i64 [ 0, %14 ], [ %34, %28 ]
  %30 = phi i32 [ 0, %14 ], [ %33, %28 ]
  %31 = getelementptr inbounds i32, i32* %13, i64 %29
  %32 = load i32, i32* %31
  %33 = add nsw i32 %32, %30
  %34 = add i64 %29, 1
  %35 = icmp eq i64 %34, %16
  br i1 %35, label %25, label %28
}
\end{lstlisting}
\end{figure}

    Most standard optimisations that LLVM provides have this implicit
    effect of normalising the code, but there are some notable exceptions.
    Most importantly, this involves optimisations that are not reliably applied
    uniformly on all code but utilise opaque cost heuristics or that require
    complex preconditions.
    Notably, this includes loop unrolling and vectorization, which are hence
    disabled.
    The CAnDL-enabled compiler invokes the detection functionality
    directly after optimisations and expects the pipeline to be
    configured with the command-line options
    ``{\tt -Os -fno-unroll-loops -fno-vectorize -fno-slp-vectorize -ffast-math}''.

    Some other LLVM optimisations also obfuscate the resulting LLVM IR but cannot
    be disabled via command-line options.
    Furthermore, there is some semantic information about multidimensional arrays
    that LLVM retains from the input language that effectively results in
    multiple distinct ways to express array accesses that are equivalent.
    To alleviate this, in addition to standard optimisations, the CAnDL-enabled
    compiler applies some custom transformations in order to pass on predictable
    LLVM IR code to the solver:
    \begin{itemize}
        \item Strength reductions result in special cases that would have to be
              covered explicitly by CAnDL specifications.
              To remove the requirement of considering these special cases, two
              strength reduction optimisations are reversed:
              {\tt add}$\mapsto${\tt or} and {\tt mul}$\mapsto${\tt shift}.
              Such instruction specialisations run counter to the goal of
              normalising the IR.
        \item Multi-dimensional array access can be expressed directly in
              LLVM IR.
              However, this only works for statically allocated arrays.
              As CAnDL should work on dynamically-sized arrays as well, this
              representation cannot be relied on and, in effect, introduces
              additional complexity.
              In order to remove this complication, complex
              ``{\tt getelementptr}'' instructions are simplified to
              add an offset to a raw pointer.
              The index calculations are then performed entirely as integer
              arithmetic on the indices.
              This process effectively flattens multi-dimensional arrays.
              Chains of ``{\tt getelementptr}'' instructions are merged, again
              resulting in integer arithmetic instead.
        \item Some interactions of ``{\tt select}'' and
              ``{\tt getelementptr}'' instructions are transformed as
              {\tt (c?a[i]:a[j])}$\mapsto${\tt a[c?i:j]} to minimise the number
              of ``{\tt getelementptr}'' instructions.
    \end{itemize}
    Finally, some structures in the LLVM IR are not modified by normalising
    transformations, but they are omitted in the SSA model that underpins the
    CAnDL specifications.
    This applies to all integer extension instructions and pointer conversions.
    These constructs are represented in LLVM IR as instructions with a single
    argument.
    In the SSA model, data flow edges skip these instructions, pointing from the
    argument of the instruction to any use of it.
    The conventional use of 32-bit integers on 64-bit systems obfuscates the
    SSA model without these omissions, particularly in the context of 32-bit
    loop interators that are automatically elevated to 64-bit integers during
    optimisation but then have to be downcast in every iteration before applying
    a comparison operator.

\subsection{The CAnDL Compiler}

\begin{figure}[t]
\centering
\begin{minipage}{0.7\textwidth}
\centering
\includegraphics[width=0.85\textwidth]{figures/candlstages.pdf}
\caption{Flow within the CAnDL compiler:
         The CAnDL source code gets lowered in several steps to generated C++
         source code.}
\label{fig:compilerflow}
\end{minipage}
\end{figure}

    The CAnDL compiler is responsible for parsing CAnDL programs and
    generating C++ code from them.
    An overview of its  flow is shown in \Cref{fig:compilerflow}.
    The frontend reads in  CAnDL source code and builds an abstract syntax tree.
    This syntax tree is simplified in two steps to eliminate some of the
    higher-order constructs of CAnDL.
    The {\it include} clauses are inlined and contained variables
    transformed accordingly.
    Furthermore, {\it conRange} and {\it disRange} are lowered to
    conjunction and disjunction constructs by duplicating the contained
    constraint code and renaming its variables appropriately for each iteration.
    The remaining core language consists only of atomics, conjunctions,
    disjunctions and collections.

    The CAnDL compiler then applies optimisations to speed up the solving
    process using the later generated C++ code.
    For example, nested conjunctions and disjunctions are flattened wherever
    possible.
    Furthermore, if shallow equivalence of two variables is enforced in a
    conjunction, one of the two variables is chosen as having higher priority.
    All other occurrences of the other variable are then replaced with that
    one.

    Finally, the compiler generates C++.
    This essentially means generating a function which at runtime constructs
    the constraint problem as a graph structure that is accessible to the
    solver.

\subsubsection{C++ Code Generation}

\begin{figure}[t]
\centering
\vspace{3.6mm}
\begin{lstlisting}[language=CAnDL]
Constraint SimpleAddition
( opcode{addition} = add
∧ {addition}.args[0] = {left}
∧ {addition}.args[1] = {right}) End
\end{lstlisting}
\begin{lstlisting}[language=MyCpp,label={fig:codegen},caption=
   {C++ code generation: The code is generated to first instantiate atomic
    constraints, then compose higher-level constructs, and finally assemble a
    backtracking solution for solving.}]
// Step 1: Instantiate Atomic Constraints
auto constr0 = make_shared<AddInstruction>(model);
auto constr1 = make_shared<FirstArgument> (model);
auto constr2 = make_shared<SecondArgument>(model);
// Step 2: Compose Higher-Level Constraints
auto constr3 = make_shared<Conjunction>(
                   constr0,
                   select<0>(constr1),
                   select<0>(constr2));
// Step 3: Assemble Backtracking Solution
vector<pair<string,shared_ptr<BacktrackingPart>>> result(3);
result[0] = make_pair("addition", constr3);
result[1] = make_pair("left",     select<1>(constr1));
result[2] = make_pair("right",    select<1>(constr2));
\end{lstlisting}
\end{figure}

    The code generation process is demonstrated with an example in
    \Cref{fig:codegen}.
    Every atomic constraint in CAnDL results in a line of C++ code that
    constructs an object of a corresponding class:
    In this case, the involved atomic constraints are implemented by
    ``{\tt AddInstruction}'', ``{\tt FirstArgument}'' and
    ``{\tt SecondArgument}''.
    These objects are instantiated as shared pointers.

    The compiler then generates analogous objects for the conjunction,
    disjunction and collect structures.
    In our example, this only affects the ``{\tt addition}'', which is part
    of a conjunction clause.
    This results in an additional object construction that instantiates the
    ``{\tt Conjunction}'' class corresponding to the ``$\land$'' operator
    in CAnDL.

    Constraint classes that implement constraints operating on a single
    variable directly expose the ``{\tt BacktrackingPart}'' interface
    introduced in \Cref{cppsolver}.
    In the example, this applies to ``{\tt AddInstruction}'' and
    ``{\tt Conjunction}''.
    For more complex constraints, the ``{\tt select<>}'' template is used to
    specify which variable of a constraint is being considered.
    At lines 8--9, this is used to extract the parts of backtracking solutions
    referring to the ``{\tt addition}'' variable, and to then pass them as
    arguments to the conjunction.

    Finally, the generated objects are inserted into a vector, together with the
    corresponding variable names.
    The variables are in the order of appearance in the CAnDL code.
    This vector corresponds to the backtracking solution of the constraint
    problem and is passed to the solver.

\subsection{Developer Tools}

\begin{figure}[t]
\centering
\setlength{\fboxsep}{-0.5pt}\setlength{\fboxrule}{0.5pt}
\fbox{\includegraphics[width=\textwidth]{figures/visual_gui2.png}}
\caption{Interactive CAnDL test tool: The left hand panel shows a Static Control
        Part (SCoP) in Polybench jacobi-2d, the right hand panel shows the
        constraint solutions found by the solver.}
\label{fig:gui}
\end{figure}

    CAnDL simplifies the construction of compiler analysis functionality, but
    reasoning about the semantics of compiler intermediate representation
    remains difficult.
    The solver detects whatever the programmer specifies, without any additional
    effort, but it is difficult to ensure that the CAnDL code actually
    specifies the structures that the programmer intended.
    Generally, the accuracy of CAnDL programs can only be ensured with
    thorough testing, and it is important to keep in mind that CAnDL is targeted
    at expert compiler developers.

    In order to make the debugging of CAnDL programs more feasible, 
    supporting tools are provided.
    Most importantly, this includes an interactive GUI, where developers can
    test out corner cases of their CAnDL programs to find false positives and
    false negatives.
    This GUI is shown in \Cref{fig:gui}, with an example from one of the use
    cases presented in \Cref{sec:casestudies}.

    In the left half, part of a C program from the PolyBench benchmark suite
    is visible, which implements a two-dimensional Jacobi stencil.
    The GUI was configured to look for Static Control Parts (SCoPs), as
    described later.
    The user has clicked the ``analyze'' button, which triggered the analysis to
    run by invoking the modified Clang compiler.
    The GUI then read the report file and printed the results in the right-hand
    part of the figure.

    The solver found a SCoP in the IR code (corresponding to lines 459--468 of
    the C program).
    The text on the right shows the hierarchical structure of the solution, with
    IR values assigned to every variable.
    The corresponding C entities can be recovered using the debug information
    that is contained in the generated LLVM IR code.
    By modifying the C code, the developer can test the detection and verify
    that no SCoP is detected if irregular control flow is introduced.

\section{Case Studies}
\label{sec:casestudies}

    The effectiveness of CAnDL was evaluated in three different use cases.
    Firstly, it was used for detecting opportunities to apply a simple peephole
    optimisation.
    Secondly, CAnDL was applied to graphics shader code optimisation.
    Finally, the detection of Static Control Parts (SCoPs) that are
    amenable for polyhedral code transformations was implemented in CAnDL.
    Where possible, the evaluation compares the number of lines of CAnDL code,
    the program coverage achieved and performance against prior approaches.

\subsection{Case Study 1: Simple Optimisations}

\begin{figure}[t]
\vspace{3.6mm}
\begin{lstlisting}[language=CAnDL,label={fig:facopport},caption=
   {Factorisation opportunities in CAnDL: This captures some opportunities that
    LLVM ``{\tt instcombine}'' misses. ``{\tt SumChain}'', ``{\tt MulChain}''
    are themselves specified in CAnDL (16 LoC).}]
Constraint ComplexFactorisation
( opcode{value} = add
∧ {sum1.value} = {value}.args[0]
∧ {sum2.value} = {value}.args[1]
∧ include SumChain @ {sum1}
∧ {product1.value} = {sum1.last_factor}
∧ include MulChain @ {product1}
∧ {product1.last_factor} = {product2.last_factor}
∧ include SumChain @ {sum2}
∧ {product2.value} = {sum2.last_factor}
∧ include MulChain @ {product2}) End
\end{lstlisting}
\end{figure}

    Arithmetic simplifications in LLVM are implemented in the
    ``{\tt instcombine}'' pass.
    An example of this is the standard factorisation optimisation that uses the
    law of distributivity to simplify integer calculations, as shown in
    \Cref{fig:factorization1}.
    Within ``{\tt instcombine}'', this is implemented in 203 lines of code
    (commit 7de5f26d, InstructionCombining.cpp lines 549-756 excluding lines 637-641),
    and additionally uses supporting functionality shared with other
    optimisations.
    \begin{align}
        a*b+a*c\rightarrow a*(b+c)
        \label{fig:factorization1}
    \end{align}

    This analysis problem can be formulated in CAnDL, as shown in
    \Cref{fig:facopport}.
    Crucially, at lines 5,7,9,11, the specification makes use of
    ``{\tt SumChain}'' and ``{\tt MulChain}'', which allows the CAnDL program to
    capture a large, generalised class of opportunities for factorisation.
    The ``{\tt instcombine}'' pass has limited support for this, and first
    requires the application of associative and commutative laws to reorder the
    values.
    For example, this is needed for \Cref{fig:factorization2}, and only
    partially supported by LLVM with the additional ``{\tt reassociate}''
    pass.
    \begin{align}
        a*b+c+d*a*e->a*(b+d*e)+c
        \label{fig:factorization2}
    \end{align}

\subsubsection{Experimental Setup}

    The specification in \Cref{fig:facopport} was evaluated against the default
    factorisation optimisation in ``{\tt instcombine}'' on three different
    benchmark collections:
    the sequential C versions of the NAS Parallel Benchmarks
    \citep{Bailey1991NPB}, as provided by \citet{seo2011performance};
    the C/C++ Parboil programs by \citet{stratton2012parboil};
    and the OpenMP C/C++ programs of the Rodinia benchmark suite
    \citep{Che2009Rodinia}.
    The existing LLVM ``{\tt instcombine}'' pass was extended, so that it
    automatically logs every time that it successfully applied the
    ``{\tt tryFactorization}'' function.  

    % NPB:     29047 loc
    % Parboil:  7358 loc
    % Rodinia: 58510 loc
    The individual benchmark programs in the three benchmark
    suites consist of 94915 lines of code in total.
    For each benchmark suite, the total number of reported factorisations, as
    well as the total compilation time, were measured.
    The standard LLVM optimisation was then disabled, and the CAnDL-generated
    detection functionality was used instead.
    The same application programs were compiled with the same version of Clang
    and identical compiler options, reporting the number of factorisations
    found and measuring the total compilation time again.
    This timing includes all the other passes within LLVM, plus the CAnDL code
    path.

\subsubsection{Results}

\begin{table}[t]
  \centering
  \definecolor{tableShade}{gray}{0.8}
  \rowcolors{1}{}{tableShade}
  \begin{tabular}{lll}
    \toprule
    & {\bf LLVM} & {\bf CAnDL} \\
    \midrule
    Lines of Code & 203 & 12 \\
    Detected in NPB & 1 & 1 + 2 \\
    Detected in Parboil & 0 & 0 + 1\\
    Detected in Rodinia & 24 & 24 + 4\\
    Total Compilation time & 152.2s & 152.2s+7.8s \\
    \bottomrule
\end{tabular}
\caption{Factorisations enabled by LLVM vs CAnDL}
\label{fig:factorization_results}
\end{table}

    The results of the evaluation are shown in
    \Cref{fig:factorization_results}.
    In two of the benchmark collections -- NPB and Parboil -- there are
    only a limited number of factorisation opportunities.
    LLVM was unable to perform any factorisation in the entire Parboil suite.
    However, the Rodinia suite contains more opportunities, mostly in the
    ``particlefilter'' and ``mummergpu'' programs.

    In all three benchmarks suites, the CAnDL system found all factorisation
    opportunities that the ``{\tt instcombine}'' pass identified.
    In addition, it detected an additional 7 cases across all programs.
    Just 12 lines of CAnDL code were able to capture more factorisation
    opportunities than 200 lines of C++ code in LLVM.

    Using CAnDL on large benchmark suites increased total compilation time
    by $\sim$5\%.
    Given the small impact of individual peephole optimisations, an
    evaluation of the performance or code size impact vs ``{\tt instcombine}''
    is unlikely to yield significant results.

\subsection{Case Study 2: Graphics Shader Optimisations}

\begin{figure}[t]
\vspace{3.6mm}
\begin{lstlisting}[language=CAnDL,label={fig:Lewis},caption=
   {CAnDL defines multiplication chains with genuine vectors and hoisted
    scalars:
    After separating the two cases, some of the multiplications can be performed
    on scalars instead.}]
Constraint FloatingPointAssociativeReorder
( include VectorMulChain
∧ collect j N
  ( {hoisted[k]} = {factors[i]} forany i=0..N
  ∧ include ScalarHoist({hoisted[j]}->{out},
                       {scalar[j]}->{in})@{hoist[j]})
∧ collect j N
  ( {nonhoisted[j]}  = {factors[i]} forany  i=0..N
  ∧ {nonhoisted[j]} != {hoisted[i]} foreach i=0..N))
End
\end{lstlisting}
\end{figure}

    Graphics computations often involve arithmetic on vectors of
    single-precision floating-point values, which can represent vertex positions
    in space or colour values.
    Established graphics shader compilers utilise the LLVM intermediate
    representation internally \citep{cudacompiler}.

    In real shader code, there are often element-wise products of several
    floating-point vectors, where some of the factors are actually scalars
    that were hoisted to vectors.
    By reordering the factors and delaying the hoisting to vectors, some of the
    element-wise vector products can be simplified to products on scalars, as
    shown by example in the following equation.
    \begin{align*}
        \vec x={}&\vec a*_v\vec b*_v\text{vec3}(c)*_v\vec d*_v\text{vec3}(e)\\
        ={}&\text{vec3}(c*e)*\vec a*_v\vec b*_v\vec d
    \end{align*}

    For general-purpose code, such reordering can be problematic.
    This is due to computation artefacts in floating-point arithmetic.
    However, this is generally no problem in the domain of graphics processing.
    Instead, associative reordering can result in performance improvements
    when combined with lowering to scalar multiplications as discussed above.

    The required analysis functionality for this optimisation was implemented
    with CAnDL, as shown in \Cref{fig:Lewis}.
    Firstly, the specification uses ``{\tt VectorMulChain}'' to detect chains of
    floating-point vector multiplications.
    At lines 4--6, all the factors that are hoisted from some scalar are
    collected into the array ``{\tt hoisted}''.
    Correspondingly, all the other factors are collected into the array
    ``{\tt nonhoisted}'' at lines 7--9.

    ``{\tt VectorMulChain}'' and ``{\tt ScalarHoist}'' are in turn implemented
    as CAnDL programs.
    ``{\tt VectorMulChain}'' discovers chains of floating-point vector
    multiplications in the IR code.
    It is defined very similarly to ``{\tt SumChain}'' and ``{\tt MulChain}'',
    which were used in the previous case study.
    It guarantees chains of maximal length by checking that neither of the first
    two factors is a multiplication itself and that the last factor is not used
    in any multiplication.

    ``{\tt ScalarHoist}'' operates on ``{\tt in}'' and ``{\tt out}'', as well
    as some hidden internal variables.
    It specifies that ``{\tt out}'' is a vector generated from ``{\tt in}'' by
    setting all vector dimensions equal to ``{\tt in}''.
    The precise implementation of this is highly specific to LLVM and involves
    combinations of the LLVM IR instructions ``{\tt insertelement}'' and
    ``{\tt shufflevector}''.

\subsubsection{Experimental Setup}

    The LunarGLASS project \citep{lunarglass} was used in this work to
    transform shaders into LLVM IR and back after optimisation.
    The CAnDL specification was applied to all fragment shaders in the
    GFXBench 4.0 suite taken from \citet{gfxbench}.
    A corresponding transformation pass was added to LLVM, which uses the
    detected solutions to implement the described optimisation.
    This was done by constructing the appropriate scalar and vector
    multiplications from the arrays ``{\tt hoisted}'' and ``{\tt nonhoisted}'',
    and then replacing the result of the original multiplication chain with the
    final multiplication in these newly generated instructions.
    Standard dead code elimination automatically removes the remnants of the
    original calculation.

    The performance impact was evaluated on the Qualcomm Adreno 530 GPU.
    To measure the baseline of benchmark performance, all shaders were
    compiled with the default Qualcomm graphics stack.
    They were then compiled with LunarGLASS into LLVM, CAnDL was applied, and
    they were transformed back into GLSL code \citep{Rost:2009:OSL:1696393}.
    To evaluate the impact, the result was passed through the default graphics
    stack again, and the performance measured.

\subsubsection{Results}

\begin{figure}[t]
\centering
\includegraphics[width=0.66\linewidth]{figures/qualcomm_plot.pdf}
\caption{Speedup on Qualcomm Adreno 530 (evaluated on HTC 10, running Android 7.0)
         \leftskip=0pt\rightskip=0pt}
\label{fig:qualcommspeedup}
\end{figure}

    There were 19 solutions to the specification across the benchmarks, and
    the transformation had an impact on the performance of 8 fragment shaders.
    The resulting performance impact is shown in \Cref{fig:qualcommspeedup}.
    Evidently, there are opportunities for such associative reordering that
    the default graphics stack misses.
    Although the performance impact was moderate with 1--4\% speedup on 8
    of the fragment shaders, it shows how new analysis can be rapidly prototyped
    and evaluated with only a few lines of code.

\subsection{Case Study 3: Detection of Polyhedral SCoPs}

    The polyhedral model
    \citep{Karp:1967:OCU:321406.321418,benabderrahmane2010polyhedral} allows
    compilers to utilise powerful mathematical reasoning to detect parallelism
    opportunities in sequential code and to implement code transformations.
    However, this applies only for a restrictive class of well-structured loop
    nests.
    More precisely, conventional polyhedral code transformations are applicable
    to Static Control Parts (SCoPs).
    Detecting SCoPs is a fundamental and necessary first step for any later
    polyhedral optimisation.

    Implementations of the polyhedral model may differ in their precise
    definition of SCoPs.
    The definition of Semantic SCoPs from the Polly compiler by
    \citet{Lengauer2012Polly} was used for reference here.
    SCoP detection functionality was implemented in CAnDL and compared against
    Polly, which is also implemented as an extension to LLVM.
    The use of the same definition for SCoPS and the implementation in the same
    compiler infrastructure allow for a direct comparison between Polly and
    CAnDL.
    The specification of SCoPs is significantly more complex than the required
    CAnDL code for the previous case studies.
    However, it can be broken into several components, with some of them shown
    in \Cref{polyhedralCAnDL}.\footnote{The complete CAnDL code for this section
    is in Appendix~\ref{appendix:CAnDLpoly}.}
    The integer constants ``10'' and ``20'' in the \textit{collect}
    statements are required for the solver to restrict the search space to a
    finite set of variables.

    \vspace{2mm}
    \noindent
    \textbf{Structured Control Flow}
    SCoPs require well-structured control flow.
    This means that each conditional jump within the
    corresponding piece of LLVM IR is accounted for by for-loops and
    conditionals.
    This is ensured with the {\it collect} constraints, as in
    \Cref{fig:collectexample}.
    The construct is used with the CAnDL specifications ``{\tt For}''
    (lines 21--25) and ``{\tt IfBlock}'' (lines 26--30) that describe the
    control flow of for-loops and conditionals.
    All the involved conditional jump instructions are extracted, and it is
    checked that these are indeed all conditional jumps within the potential
    SCoP (lines 14--24 and lines 31-33).

    Once the control flow has been established, the iterators of the loops
    (lines 4--5) are used to define affine integer computations in the loop
    (lines 6--9).
    This is done in a brute-force fashion with a recursive constraint program
    ``{\tt AffineCalc}'' (line 50).
    It is checked that the iteration domain of all the for-loops is
    well-behaved, i.e.\ the boundaries are affine in the loop iterators.

    \vspace{2mm}
    \noindent
    \textbf{Affine Memory Access}
    All memory accesses in the SCoP must be affine.
    For this to be true, it needs to be verified that for each ``{\tt load}''
    and ``{\tt store}'' instruction, the base pointer is loop-invariant, and the
    index is calculated affinely.
    The loop-invariant base pointer is easily checked in ``{\tt MemoryAccess}
    with the ``{\tt LocalConst}'' program from \Cref{localconstant}.

    Checking the index calculations is more involved and is again based on the
    method that was demonstrated in \Cref{fig:collectexample}.
    The {\it collect} construct is used to find all of the affine memory
    accesses in all the loop nests (lines 43--54).
    Another {\it collect} gathers all ``{\tt load}'' and ``{\tt store}''
    instructions (36--42), guaranteeing that both collections are identical.

\begin{figure}[p]
\lstset {
 basicstyle=\linespread{0.915}\small\ttfamily
}
\vspace{2.7mm}
\begin{lstlisting}[language=CAnDL, label=polyhedralCAnDL, caption=
   {Fragments of the specification of Scalar Control Parts (SCoPs) using CAnDL:
    SCoPs are defined at lines 1--11 by applying multiple restrictions to the
    containing loop.
    These restrictions are then individually implemented in CAnDL, using
    ``{\tt StructuredControlFlow}'' and ``{\tt AffineMemAccesses}'', shown in
    lines 13--33 and lines 35--54, respectively
    (cf.\ Appendix~\ref{appendix:CAnDLpoly}).}]
Constraint SCoP
( include For @ {loop}
∧ include StructuredControlFlow({loop}->{scope}) @ {control}
∧ {inputs[0]} = {loop.iterator}
∧ {inputs[i]} = {control.loop[i-1].iterator} foreach i=1..10
∧ include AffineControlFlow({loop}->{scope},
                          {inputs}->{inputs}) @ {control}
∧ include AffineMemAccesses({loop}->{scope},
                          {inputs}->{inputs}) @ {accesses}
∧ include SideEffectFreeCalls({loop}->{scope}) @ {effects})
End

Constraint StructuredControlFlow
( collect i 20 ( opcode{branch[i].value} = branch
               ∧ {branch[i].target1} =
                     {branch[i].value}.successors[0]
               ∧ {branch[i].target2} =
                     {branch[i].value}.successors[1]
               ∧ include ScopeValue({scope}->{scope},
                          {branch[i].value}->{value}))
∧ collect i 10 ( include For @ {loop[i]}
               ∧ domination({scope.begin},
                          {loop[i].begin})
               ∧ strict_post_domination({scope.end},
                                      {loop[i].end}))
∧ collect i  10 ( include IfBlock @ {ifblock[i]}
               ∧ domination({scope.begin},
                       {ifblock[i].precursor})
               ∧ strict_post_domination({scope.end},
                                   {ifblock[i].successor}))
∧ {loop[0..10].end,ifblock[0..10].precursor}
      is the same set as {branch[0..20].value})
End

Constraint AffineMemAccesses
( collect x 20 ( include MemoryAccess({scope}->{scope})
                                      @ {newaccess[x]}
               ∧ opcode{newaccess[x].pointer} = gep
               ∧ domination({scope.begin},
                     {newaccess[x].pointer})
               ∧ {newaffine[x].value} =
                     {newaccess[x].pointer}.args[1])
∧ collect x 20 ( include MemoryAccess({scope}->{scope})
                                      @ {newaccess[x]}
               ∧ opcode{newaccess[x].pointer} = gep
               ∧ domination({scope.begin},
                     {newaccess[x].pointer})
               ∧ {newaffine[x].value} =
                     {newaccess[x].pointer}.args[1]
               ∧ include AffineCalc[M=10,N=6](
                                          {scope}->{scope},
                                         {inputs}->{input})
                                          @ {newaffine[x]}))
End
\end{lstlisting}
\end{figure}

\subsubsection{Experimental Setup}

    The reliable detection of SCoPs was evaluated on the PolyBench suite
    \citep{polybench}, a collection of 31 benchmark programs that contain SCoPs
    of differing complexity from several application domains.
    For both the CAnDL-based approach and for the evaluation of Polly, it was
    counted how many of the computational kernels contained in the benchmark
    suite were captured in their entirety by the respective analysis.

    Some post-processing of the generated constraint solutions was required to
    compare the results of CAnDL and Polly.
    This was needed because the output of CAnDL was not in the JSCoP format
    that Polly generates, but contained the raw constraint solution
    encoded as a JSON file.
    Furthermore, the CAnDL implementation did not merge consecutive outer level
    loops into a single SCoP of maximum size.
    Therefore, the detected loops from the CAnDL solver were extracted and
    grouped together.
    A Python script was then used to verify that they precisely covered the
    SCoPs detected by Polly.

\begin{table}[t]
    \centering
\definecolor{tableShade}{gray}{0.9}
\rowcolors{1}{}{tableShade}
\begin{tabular}{lll}
  \toprule
  & {\bf Polly} & {\bf CAnDL} \\
  \midrule
  Lines of Code & 1903 & 45 \\
  Detected in datamining & 2 & 2\\
  Detected in Linear-algebra & 19 & 19\\
  Detected in medley & 3 & 3\\
  Detected in stencils & 6 & 6\\
  \bottomrule
%Compilation time & 24.4s+37.5s & 24.4s+12.7s \\ \hline
\end{tabular}
    \caption{Polly and CAnDL detected all SCoPs.}
    \label{fig:candlvspolly}
\end{table}

\subsubsection{Results}

\begin{table}
    \centering
    \definecolor{tableShade}{gray}{0.8}
    \rowcolors{1}{}{tableShade}
    \begin{tabular}{lll}
        \toprule
        & Clang compile times& Clang+CAnDL compile times\\
        \midrule
        datamining/correlation     & 59 & 427 \\
        datamining/covariance      & 48 & 344 \\
        linear-algebra/2mm         & 53 & 599 \\
        linear-algebra/3mm         & 52 & 844 \\
        linear-algebra/atax        & 48 & 241 \\
        linear-algebra/bicg        & 51 & 246 \\
        linear-algebra/cholesky    & 52 & 245 \\
        linear-algebra/doitgen     & 46 & 424 \\
        linear-algebra/gemm        & 50 & 330 \\
        linear-algebra/gemver      & 53 & 329 \\
        linear-algebra/gesummv     & 46 & 170 \\
        linear-algebra/mvt         & 50 & 224 \\
        linear-algebra/symm        & 54 & 293 \\
        linear-algebra/syr2k       & 51 & 341 \\
        linear-algebra/syrk        & 49 & 327 \\
        linear-algebra/trisolv     & 49 & 187 \\
        linear-algebra/trmm        & 46 & 223 \\
        linear-algebra/durbin      & 54 & 309 \\
        linear-algebra/dynprog     & 47 & 291 \\
        linear-algebra/gramschmidt & 54 & 704 \\
        linear-algebra/lu          & 48 & 245 \\
        linear-algebra/ludcmp      & 52 & 521 \\
        medley/floyd-warshall      & 47 & 204 \\
        medley/reg\_detect         & 54 & 566 \\
        stencils/adi               & 54 & 753 \\
        stencils/fdtd-2d           & 52 & 500 \\
        stencils/fdtd-apml         & 61 & 977 \\
        stencils/jacobi-1d-imper   & 76 & 163 \\
        stencils/jacobi-2d-imper   & 56 & 299 \\
        stencils/seidel-2d         & 58 & 229 \\
        \bottomrule
    \end{tabular}
    \caption{Overhead of SCoP detection with CAnDL:
             Compile times are listed in milliseconds.
             The geomean increase in compile times due to SCoP detection
             with CAnDL was $556\%$.}
    \label{fig:PolyBenchCompileTimes}
\end{table}

    \Cref{fig:candlvspolly} shows that the CAnDL specification captured all the
    SCoPs that Polly detected.
    To measure the lines of code required, the CAnDL version was compared with
    the amount of code in {\tt ScopDetection.cpp} of Polly.
    The same detection results were achieved with much fewer lines of code in
    CAnDL.
    Note that the line count that is given for the CAnDL program does not
    include all the CAnDL code involved in the detection of polyhedral regions.
    Code that is not specific to this idiom (such as loop structures) is
    considered as part of the CAnDL standard library.
    In the same way, the line count for Polly does not account for additional
    code that Polly relies on when detecting SCoPS, e.g.\ the expansive
    ``ScalarEvolution'' pass.

    Detecting SCoPs with CAnDL incurred a significant overhead, but compile
    times remained below one second for each PolyBench program
    on an Intel i7-8665U processor.
    Detailed compile time results are presented in
    \autoref{fig:PolyBenchCompileTimes}.
    The geomean overhead of enabling the CAnDL detection of SCoPS during
    compilation was $556\%$, mostly due to the large number of variables
    that are required for expressing affine calculations.
    The compile time impact of all other idioms discussed in this thesis is
    much more moderate (compare \autoref{tab:compiletimecost}).
    This overhead can be prohibitive during software development when frequent
    recompilation is required.
    However, it pales in comparison to autotuners, superoptimisation, and many
    other compiler techniques that involve constraint solvers and should be
    unproblematic in many contexts.

    With a high-level representation of SCoPs, CAnDL allows polyhedral
    compiler researchers to explore the impact of relaxing or tightening the
    exact definition of SCoPs in a straightforward manner, enabling rapid
    prototyping.
    Simple commenting out of constraint statements in the CAnDL specification
    relaxes the conditions.

\section{Conclusions}

    Optimising compilers require sophisticated program analysis in order to
    generate performant code.
    The state-of-the-art approach for implementing this functionality manually
    in C++ is not satisfactory, as exemplified by the complicated and
    error-prone ``{\tt instcombine}'' pass in the LLVM compiler infrastructure.

    The domain-specific Compiler Analysis Description Language (CAnDL) provides
    a more efficient approach.
    CAnDL specifications can automatically generate compiler analysis
    passes from a declarative description.
    They are easier to program and significantly reduce the code size and
    complexity when comparing against manual C++ implementations.
    Although CAnDL is based on a constraint programming paradigm and uses a
    backtracking solver to analyse the LLVM IR code, compile times never
    exceeded one second.

    Many compiler analysis tasks are suitable for implementation with CAnDL.
    It can be used for the detection of standard peephole optimisation
    opportunities and the rapid prototyping of graphics shader optimisations.
    Despite its general approach, CAnDL scales to complex domain-specific code
    structures and can efficiently recognise large code regions suitable for
    polyhedral transformations.

    \paragraph*{Outlook}
    The final case study showed that CAnDL can express the algorithmic structure
    of complex loop nests, not just peephole optimisation
    opportunities.
    This enables the capturing of entire loops at a time.
    The following chapters expand on this observation by investigating how the
    recognition of computational idioms can be leveraged for performance.
    While SCoPs were reported without informing additional compiler
    transformations, \Cref{chapter:reductions} applies auto-parallelisation
    techniques to a broad generalisation of reduction computations.


\chapter[Automatic Parallelisation of Complex Reductions and Histograms]
        {Automatic Parallelisation of Complex Reductions and Histograms
         \footnote{This chapter is based on published research in
                   \citet{ginsbach2017discovery}.}}
    \label{chapter:reductions}
    This chapter is based on ASAPLOS.

\chapter[Formalizing Computational Idioms for Heterogeneous Acceleration]
        {Formalizing Computational Idioms for Heterogeneous Acceleration
         \footnote{This chapter is based on published research in
                   \citet{Ginsbach:2018:AML:3173162.3173182}.}}
    \label{chapter:idioms}
    
    The previous chapter introduced a computational idiom --
    Complex Reduction and Histogram Computations (CReHCs) -- and showed how the
    Idiom Description Language (IDL) enables its discovery and automatic
    parallelisation.
    This chapter takes a broader view of automatic idiom detection that also
    includes common sparse and dense linear algebra computations and stencil
    codes.
    Instead of implementing bespoke parallelisation passes, the chapter
    follows the vision laid out in the introduction, using the detection results
    to relevant program parts into stronger models.
    Existing tools are then used to leverage the implied domain knowledge.

    The goal of this chapter is to take abstract algorithmic concepts, which are
    conventionally explored outside the context of compiler analysis --
    computational idioms -- and to formalize them as IDL specifications,
    enabling detection and manipulation in optimising compilers.
    Heterogeneous acceleration will serve as the motivation for this effort.
    As many scientific codes are structured around idiomatic performance
    bottlenecks, efforts that focus on computational idioms can greatly
    improve performance, especially with accelerators that were designed
    with similar computations in mind.
    The focus is therefore on calculations that are well supported by
    accelerators and their software ecosystems: linear algebra,
    stencil codes and CReHCs.

    The IDL specifications are used to build a prototype compiler that
    automatically detects the idioms and uses them to circumvent the code
    generator with libraries and domain specific languages:
    BLAS implementations, cuSPARSE, clSPARSE, Halide and Lift.
    The funcionality is directly accessible in a modified version of the widely
    used Clang C/C++ compilers.
    The evaluation can therefore be performed on the well established benchmark
    suites NAS and Parboil, where 60 idiom instances are detected.
    In those cases where idioms are a significant part of the sequential
    execution time, the generated heterogeneous code achieves 1.26$\times$ to
    over 20$\times$ speedup on integrated and external GPUs.

\section{Introduction}

    Heterogeneous accelerators provide the potential for superior performance.
    However, realising this potential in practice is difficult and requires
    significant programmer effort.
    Programs have to be partially rewritten to target heterogeneous systems,
    using a diverse range of broad and narrow interfaces.
    General-purpose languages such as OpenCL \citep{nvidia11guide} provide
    some portability across heterogeneous devices, but the achieved performance
    often disappoints \citep{lee09openmp}.
    Despite the functional portability, rewrites are required in practice to
    achieve competitive performance.
    Optimised numerical libraries provide more reliable performance, but they
    are more specialised and often provided by hardware vendors without
    portability in mind \citep{clblas,cublas,cusparse,clsparse}.
    More narrow domain-specific languages (DSLs) have been proposed by
    \citet{Ragan-Kelley2013Halide,Franchetti09OL,Rompf:2012:LMS:2184319.2184345}
    among others in attempts to deliver both portability and performance.
    However, this class is quickly evolving, and with most of these DSLs being
    academic projects, the adoption and long term support remain unclear.

    Hardware is becoming increasingly heterogeneous, most recently with the
    development of deep neural network accelerators such as the Google TPU
    \citep{jouppi2017datacenter}.
    This means that library or DSL based programming is likely to become far
    more common.
    The problems with this trend that arise due to the aforementioned tradeoffs
    are evident:
    Firstly, application developers have to learn multiple specialised DSLs and
    vendor-specific libraries if they want good performance.
    Secondly, they will have to rewrite their existing applications to use them.
    Thirdly, this ties code into an ecosystem with unclear future support
    that might soon become obsolete.
    This situation is a severe impediment to the wide-spread efficient
    exploitation of heterogeneous hardware.

    The ideal would be a compiler that automatically maps existing code to
    heterogeneous hardware, with full performance and requiring no directions
    from the application programmer.
    While this is unrealistic in general, the chapter presents a system that
    approximates such a general-purpose system by utilising know-how that is
    already available and encapsulated in the existing backend interfaces.
    Instead of implementing code generation for each heterogeneous
    accelerator, the system maps user code to heterogeneous hardware using the
    existing libraries and domain-specific languages, effectively outsourcing
    the code generation to hardware and domain specialists.

    The approach is based on detecting specific {\em computational idioms} in
    application code that correspond to the functionality of existing interfaces
    -- libraries and DSLs -- for heterogeneous acceleration.
    In addition to the Complex Reduction and Histogram Computations (CReHCs)
    introduced in \Cref{chapter:reductions}, the focus is on
    sparse and dense linear algebra, as well as stencils.
    The covered idioms are a reflection of both the most relevant program
    bottlenecks and the available interfaces.
    Some computational idioms are more widely supported than others, potentially
    pointing to gaps in the accelerator landscape, but some backend was
    available for each of them.
    The Idiom Detection language (IDL) then enabled the automatic detection of
    idioms.

    Once detected, the idioms are mechanically translated into the appropriate
    DSL or replaced with a library call.
    This optimised code, or the pre-built optimised library, is then linked into
    the original program.
    As backends, the libraries cuSparse, clSparse, cuBLAS, clBLAS for
    sparse and dense linear algebra and the DSLs Halide
    \cite{Ragan-Kelley2013Halide} and Lift \cite{SteuwerRD17} were used in the
    evaluation.
    The Lift language is a data-parallel functional language that supports
    generalised reductions as well as stencils and linear algebra.
    The wide range of backends allows the freedom to target many APIs per idiom
    and pick the implementation that best suits the target platform.

    New computational idioms can be easily added thanks to the flexibility of IDL.
    This also provides a powerful means of determining whether a proposed
    heterogeneous interface matches existing code, without touching the core
    compiler.
    The idioms addressed in this paper can be expressed in less than 500 lines
    of IDL code.
    The approach is also highly robust, has been applied to the entire NPB
    and Parboil benchmark suites and was evaluated on three platforms.

    The chapter presents a novel approach that:
    \begin{itemize}
    \item Proposes the Idiom Description Language (IDL) for detecting idiomatic
          code sections that can be accelerated by domain-specific backends.
    \item Implements several common computational idioms in IDL to automatically
          discover opportunities for accelerator exploitation.
    \item Efficiently translates and maps the detected idioms to APIs for
          heterogeneous systems.
    \end{itemize}

    The work most similar in approach discovers stencil computations and maps to
    the Halide DSL for acceleration.
    The Helium tool \citep{Mendis2015Helium} recovers stencils from
    image-processing binaries.
    This requires large scale dynamic analysis of binary traces and replacing
    them with Halide calls. 
    This is significantly extended by \citet{Kamil2016Verified}, detecting
    stencils in Fortran code.
    The focus of that work is on inferring post invariants based on syntax
    guided synthesis in translation to Halide.
    However, it uses a narrow approach to selecting code snippets and relies on
    well-structured Fortran with occasional user annotations.
    The IDL approach is distinct in its use of an external programming language
    for the flexibility of describing arbitrary idioms.
    This allows an unbounded set of idioms to be considered and is not
    restricted to stencils. 

    To summarise, this chapter presents an automatic approach that discovers
    idioms in legacy code and maps them to heterogeneous platforms via libraries
    and DSLs.
    The tool was applied to 21 C/C++ programs from the NPB and Parboil benchmark
    suites, where it detected more reductions, stencils, matrix multiplications
    and sparse matrix-vector computations than existing schemes.
    For the programs where idioms dominate execution time, accelerator code
    was generated and evaluated on 3 platforms: a multi-core CPU, an integrated
    APU, and an external GPU.
    Overall, 60 idioms were detected, and they dominated execution time in
    10 programs.
    Speedup results for the accelerated code ranged from 1.26$\times$ to over
    20$\times$.

\section{Overview}

    The approach is automatic and has been implemented inside the LLVM compiler
    infrastructure.
    It takes arbitrary sequential C/C++ programs as input.
    Using the Clang compiler, the input source code is compiled into a Static
    Single Assignment (SSA) intermediate representation (LLVM IR).
    In this representation, the specified idioms are identified and replaced
    with calls to specific APIs.
    Finally, the code generated by the LLVM compiler and the output of the idiom
    specific code generators/libraries are linked together into a binary,
    producing an optimised program.
    LLVM was chosen as it is the best supported SSA-based compiler;
    the methodology could easily be transferred to other infrastructures such as
    GCC.

\subsection{Compiler Flow}

    The structure of the approach is described in more detail in
    \Cref{fig:methodology}.
    The compiler takes two programs as inputs: the first is the source code of
    the user program, the second describes the computational idioms that are to
    be detected specified in IDL.
    The same idioms, of course, can be detected across many user programs.
    Therefore, the IDL program does not have to change from one run to the next.

\begin{figure}[p]
    \centering
    \includegraphics[width=\linewidth]{figures/compiler_flow.pdf}
    \caption{Workflow of the IDL acceleration system:
             The IDL solver extracts idiomatic loops in the optimised LLVM IR
             of user programs.
             These loops are extracted and replaced with shim function calls.
             Domain-specific code generators implement the calls as library
             objects, or they are taken directly from pre-generated vendor
             libraries for idioms not containing kernel functions.}
    \label{fig:methodology}
\end{figure}

    The source code of the user program is compiled into optimised LLVM IR code,
    and the idiom description is parsed and represented internally as a C++
    object, as previously described in \Cref{chapter:theory},
    \Cref{subsec:impl}.
    The C++ representation of the constraints and the LLVM IR code
    are then passed as inputs to the backtracking solver, which detects all
    cases where the idioms can be found in the LLVR IR.

    The recognised idioms and the LLVM IR code are then passed on to the
    transformation phase of the system.
    The sections of code corresponding to computational idioms are extracted
    and then reformulated for the appropriate heterogeneous backends.
    For libraries, this means replacing the code covered by the idiom with a
    library call. 

    For DSL interfaces, the process is a little more involved.
    The user code captured by idioms is extracted and replaced with functions
    calls to shim interfaces.
    The extracted code features are translated into the appropriate DSL and
    passed on to the external DSL compiler, which optimises it and generates a
    library object that implements the required function interfaces.
    The translation to DSL mainly involves representing the kernel functions.
    Idioms without kernel functions correspond to fixed DSL programs and
    require no additional work.
    The generated code is then linked with the object code from the main program.

    Determining the best heterogeneous APIs to use for a given platform and the
    best idioms to exploit will become an essential consideration as the number
    of idioms and APIs grows.
    Currently, for this chapter, all applicable libraries and DSLs were
    evaluated, and the best-performing versions selected after profiling.

\subsection{Accelerating Sparse Linear Algebra}

    Sparse linear algebra is central to many scientific codes and increasingly
    important as a basis for graph algorithms and data analytics
    \cite{Kepner2015GraphsMA}, but they contain indirect data access that limits
    compiler optimisation.
    Instead, programmers rely on library implementations that are hand optimised
    and utilise accelerator hardware.
    This reliance on libraries comes at a cost, however, as it ties programs
    into vendor-specific software ecosystems and results in non-portable code.
    The IDL approach offers an alternative by recognising sparse linear algebra
    operations in compiler intermediate representation and then incorporating
    domain-specific backends without source code changes.

    The code at the top of \Cref{fig:spmvexample1} is the performance
    bottleneck of the ``Conjugate Gradient'' benchmark program in the NAS
    Parallel Benchmarks, with the corresponding LLVM~IR code underneath.
    This bottleneck loop implements a standard operation from sparse linear
    algebra, namely the multiplication of a sparse matrix in
    Compressed Sparse Row (CSR) format with a dense vector.
    This computation is supported on accelerator hardware, using well-optimised
    libraries such as cuSPARSE and clSPARSE.
    However, compilers are unable to recognise and accelerate the computation
    automatically.

    The structure of this sparse linear algebra computation has several features
    that make it unsuitable for most established compiler optimisations:
    Firstly, the iteration domain of the nested loop is memory dependent
    (line 3), and secondly, there is indirect memory access (line~4).
    This makes the iteration domain of the loop nest as a whole non-polyhedral
    and the access structure to memory non-affine.
    Under these conditions, simple data dependence models, but also
    sophisticated analysis based on the polyhedral model, fail.

    IDL can express this sparse idiom, as derived in \Cref{sec:idioms},
    \Cref{csr_lilacwhat_fig}.
    The ``Conjugate Gradient'' LLVM IR code, together with the ``SPMV-CSR''
    IDL specification, are input to the constraint solver, which outputs a
    constraint solution, as shown in \Cref{fig:spmvexample2}.
    In the solution, different values for the LLVM IR have been assigned to all
    IDL variables in the ``SPMV-CSR'' specification.

    \Cref{fig:spmvexample3} shows how this solution is used to generate a
    call to a cuSPARSE procedure.
    The individual solution variables are inserted into the {\tt cusparseDcsrmv}
    code template as function arguments. 
    The original code is then cut out and replaced with this function call.
    In practice, this involves a shim function that manages the device context
    and memory transfers from and to the GPU.
    Finally, the cuSPARSE library is linked with the object code produced by the
    Clang compiler, resulting in a speedup of $17\times$ on a GPU as described
    in more detail in\Cref{sec:idiomresults}.

    Central to this approach is the ability to detect computational idioms
    reliably.
    The next section derives in detail the formulation of linear algebra -- both
    sparse and dense -- and stencils in the Idiom Detection Language.

\begin{figure}[p]
    \lstset{
 basicstyle       = \linespread{0.826}\footnotesize\ttfamily,
 xleftmargin      = 2em,
 framexleftmargin = 2em
}
\centering
\vspace{1.9mm}
\begin{lstlisting}[language=MyCpp]
for (j = 0; j < ([{\bf m}]); j++) {
  d = 0.0;
  for (k = ([{\bf rowstr }])[j]; k < ([{\bf rowstr}])[j+1]; k++)
    d = d + ([{\bf a}])[k]*([{\bf z}])[([{\bf colidx}])[k]];
  ([{\bf r}])[j] = d; }
\end{lstlisting}
\vspace{-3.5mm}
\begin{lstlisting}[language={LLVM}, label={fig:spmvexample1}, caption=
   {Sparse matrix-vector product shown in C at the top, and in LLVM IR at the
    bottom\leftskip=0pt\rightskip=0pt}]
; <label>:2:
  %j = phi i64 [ %j_next, %12 ], [ 0, %1 ]
  %j_cond = icmp slt i64 %j, ([{\bf \%m}])
  br i1 %j_cond, label %3, label %13([\vspace{1mm}])
; <label>:3:
  %4 = getelementptr i32, i32* ([{\bf \%rowstr}]), i64 %j
  %5 = load i32, i32* %4
  %j_next = add nuw nsw i64 %j, 1
  %6 = getelementptr i32, i32* ([{\bf \%rowstr}]), i64 %j_next
  %7 = load i32, i32* %6
  %k_begin = sext i32 %5 to i64
  %k_end = sext i32 %7 to i64
  br label %8([\vspace{1mm}])
; <label>:8:
  %k = phi i64 [ %k_next, %9 ], [ %k_begin, %dnext ]
  %d = phi double [ 0.0, %3 ], [ %d_next, %9 ]
  %k_cond = icmp slt i64 %iv, %k_end
  br i1 %k_cond, label %9, label %12([\vspace{1mm}])
; <label>:9:
  %a_addr = getelementptr double, double* ([{\bf \%a}]), i64 %k
  %a_load = load double, double* %a_addr
  %cix_addr = getelementptr i32, i32* ([{\bf \%colidx}]), i64 %k
  %cix_load = load i32, i32* %cix_addr
  %10 = sext i32 %cix_load to i64
  %z_addr = getelementptr double, double* ([{\bf \%z}]), i64 %10
  %z_load = load double, double* %z_addr
  %11 = fmul double %a_load, %z_load
  %d_next = fadd double %d, %11
  %k_next = add nsw i64 %k, 1
  br label %8([\vspace{1mm}])
; <label>:12:
  %r_addr = getelementptr double, double* ([{\bf \%r}]), i64 %j
  store double %d, double* %r_addr
  br label %2
\end{lstlisting}

\vspace{0.05mm}
{{\Huge$\Downarrow$}
\hspace{-2.73mm}\phantom{\bf~~~Idiom Detection with IDL~~}\hspace{-3.69mm}
{\Huge$\Downarrow$}
\hspace{-2.73mm}\phantom{\bf~~Program in \Cref{csr_lilacwhat_fig}~~~}\hspace{-3.69mm}
{\Huge$\Downarrow$}}
\hspace{5.1mm}

\vspace{-10.5mm}
{\phantom{\Huge$\Downarrow$}
\hspace{-2.73mm}{\bf~~~Idiom Detection with IDL~~}\hspace{-3.69mm}
\phantom{\Huge$\Downarrow$}
\hspace{-2.73mm}{\bf~~Program in \Cref{csr_lilacwhat_fig}~~~}\hspace{-3.69mm}
\phantom{\Huge$\Downarrow$}}
\hspace{5.1mm}
\vspace{4mm}

{\footnotesize\renewcommand{\arraystretch}{0.8}
\definecolor{tableShade}{gray}{0.8}
\rowcolors{1}{}{tableShade}
\begin{tabular}{p{4.1cm}l}
    \toprule
    \textbf{IDL Variable Name} & \textbf{Assigned Value}\\
    \midrule
    outer\_loop.iterator                      & {\tt\%j}\\
    inner\_loop.iter\_begin                   & {\tt\%k\_begin}\\
    inner\_loop.iter\_end                     & {\tt\%k\_end}\\
    inner\_loop.iterator                      & {\tt\%k}\\
    inner\_loop.src1 ($\approx$ val.value)    & {\tt\%z}\\
    col\_ind.value                            & {\tt\%cix\_load}\\
    inner\_loop.src2 ($\approx$ vector.value) & {\tt\%a\_load}\\
    \bottomrule
\end{tabular}
\hfill
\rowcolors{1}{}{tableShade}
\begin{tabular}{p{4.1cm}l}
    \toprule
    \textbf{IDL Variable Name} & \textbf{Assigned Value}\\
    \midrule
    output.address          & {\tt\%r\_addr}\\
    outer\_loop.iter\_begin & {\tt0}\\
    outer\_loop.iter\_end   & {\tt\%\bf m}\\
    val.base\_pointer       & {\tt\%\bf a}\\
    col\_ind.base\_pointer  & {\tt\%\bf colidx}\\
    vector.base\_pointer    & {\tt\%\bf z}\\
    \dots                   & \dots\\
    \bottomrule
\end{tabular}}

\vspace{-1.8mm}
\caption{Solution to ``{\tt SPMV\_CSR}'': Fitting LLVM IR values were assigned
         to all IDL variables.\leftskip=0pt\rightskip=0pt}
\label{fig:spmvexample2}

\vspace{3.25mm}
{{\Huge$\Downarrow$}
\hspace{-2.73mm}\phantom{\bf~~~Code Generation: Insert~~}\hspace{-3.69mm}
{\Huge$\Downarrow$}
\hspace{-2.73mm}\phantom{\bf~~Arguments, Replace Code~~~}\hspace{-3.69mm}
{\Huge$\Downarrow$}}
\hspace{-3.4mm}

\vspace{-10.5mm}
{\phantom{\Huge$\Downarrow$}
\hspace{-2.73mm}{\bf~~~Code Generation: Insert~~}\hspace{-3.69mm}
\phantom{\Huge$\Downarrow$}
\hspace{-2.73mm}{\bf~~Arguments, Replace Code~~~}\hspace{-3.69mm}
\phantom{\Huge$\Downarrow$}}
\hspace{-3.4mm}
\vspace{-2.75mm}

\vspace{6.3mm}
\begin{lstlisting}[language=MyCpp, label={fig:spmvexample3}, caption=
   {GPU acceleration: Solution values are used to call ``{\tt cusparseDcsrmv}''
    backend.\leftskip=0pt\rightskip=0pt}]
cusparseDcsrmv(context,
    CUSPARSE_OPERATION_NON_TRANSPOSE, ([{\bf m}]), n,
    ([{\bf rowstr}])[([{\bf m}])+1]-([{\bf rowstr}])[0], &gpu_1, descr, gpu_([{\bf a}]),
    gpu_([{\bf rowstr}]), gpu_([{\bf colidx}]), gpu_([{\bf z}]), &gpu_0, gpu_([{\bf r}]));
\end{lstlisting}
\end{figure}

\section{Specification of Idioms in IDL}
\label{sec:idioms}

    The specification of computational idioms in IDL requires a careful
    handling of the arising complexity, using the modularity functionality that
    IDL provides.

    Control flow constructs, memory access patters as well as basic data flow
    patterns are specified independently and then combined together in order to
    define the complete constraint specifications.

\subsection{Sparse Linear Algebra}
At the heart of our approach is a simple language to specify sparse and dense
linear algebra operations.
This serves two purposes in our LiLAC system: Firstly, it is used to generate
a detection program for finding the computation in user code.
Secondly, it identifies the variables that are arguments to the library, thus
defining the harness interface.

The key challenge in the design of this language was to stay simple enough
to allow automatic generation of robust detection functionality, yet to be able
to capture interesting functionality.
Crucial for sparse linear algebra routines is capturing the many different
memory access patterns, the control flow on the other hand is very rigid.

\subsection{Sparse Matrix Variations in LiLAC-What}
Sparse matrices can be stored in different formats.
In this section we introduce two of them and show how LiLAC-What can express
the corresponding computations.

\begin{figure}[p]
\lstset {
 basicstyle=\linespread{1.0}\small\ttfamily
}
\begin{lstlisting}[language=IDL, label={spmvbase}, caption=
   {Skeleton of the sparse matrix-vector product (SPMV) constraint
    specification in IDL: The precise sparse access patterns are specific to
    chosen storage formats for sparse matrices.}]
Constraint SPMV
( inherits For at {outer_loop} and
  inherits DotProductFor at {inner_loop} and
  {outer_loop.begin} strictly
      control flow dominates {inner_loop.begin} and
  {outer_loop.end} strictly
      control flow post dominates {inner_loop.end} and
([\dots])
\end{lstlisting}
\end{figure}

\begin{figure}[p]
\hfill
\begin{minipage}[b]{0.3\linewidth}
\includegraphics[width=0.95\linewidth]{figures/csrorder.png}
\end{minipage}
\hfill
\begin{minipage}[b]{0.5\linewidth}
\begin{align*}
\text{\bf val} =& \begin{bmatrix}
1\ \ 1\ \ 2\ \ 2\ \ \text{-}1\ \ 3\ \ 2\ \ 2\ \ \text{-}1\ \ 1\\
\end{bmatrix}\\[4mm]
\text{\bf col\_ind} =& \begin{bmatrix}
0\ \ 2\ \ 1\ \ 3\ \ 1\ \ 2\ \ 3\ \ 3\ \ 2\ \ 4\\
\end{bmatrix}\\[4mm]
\text{\bf row\_ptr} =& \begin{bmatrix}
0\ \ 2\ \ 4\ \ 7\ \ 8\ \ 10\\
\end{bmatrix}
\end{align*}
\end{minipage}
\hfill

\vspace{0.8em}
\hrule
\vspace{0.3em}

\includegraphics[width=\linewidth]{figures/spmvcsrwhat.pdf}
\vspace{-1.5em}
\lstset {
 basicstyle=\linespread{1.0}\small\ttfamily
}
\begin{lstlisting}[language=IDL,firstnumber=7]
([\dots])
  inherits VectorStore
      with {outer_loop}          as {scope}
       and {outer_loop.iterator} as {input_index} at {output} and
  inherits VectorRead
      with {outer_loop}          as {scope}
       and {inner_loop.src1}     as {value}
       and {inner_loop.iterator} as {input_index} at {val} and
  inherits VectorRead
      with {outer_loop}      as {scope}
       and {inner_loop.src2} as {value}
       and {col_ind.value}   as {input_index} at {vector} and
  inherits VectorRead
      with {outer_loop}          as {scope}
       and {inner_loop.iterator} as {input_index} at {col_ind} and
  inherits ReadForLoopRanges
     with {outer_loop}          as {scope}
      and {inner_loop}          as {for}
      and {outer_loop.iterator} as {input_index} at {row_ptr})
End
\end{lstlisting}
\caption{Compressed Sparse Row in IDL:
         The top section of the figure shows the different involved arrays.
         The pseudocode in the middle row of the
         figure then directly gives a completion of \Cref{spmvbase}.
         Walking through the expressions and emitting IDL code on-by-one is
         sufficient.}
\label{csr_lilacwhat_fig}
\end{figure}

\begin{figure}[p]
\hfill
\begin{minipage}[b]{0.3\linewidth}
\includegraphics[width=0.9\linewidth]{figures/jdsorder.png}
\end{minipage}
\hfill
\begin{minipage}[b]{0.65\linewidth}
\footnotesize
\begin{align*}
\text{\bf perm} =& \begin{bmatrix}1\ \ 2\ \ 0\ \ 4\ \ 3\\\end{bmatrix}\\[-0.75mm]
\text{\bf val} =& \begin{bmatrix}\text{-}1\ \ 1\ \ 2\ \ \text{-}1\ \ 2\ \ 3\ \ 1\ \ 2\ \ 1\ \ 2\\\end{bmatrix}\\[-0.75mm]
\text{\bf col\_ind} =& \begin{bmatrix}1\ \ 0\ \ 1\ \ 2\ \ 3\ \ 2\ \ 2\ \ 3\ \ 4\ \ 3\\\end{bmatrix}\\[-0.75mm]
\text{\bf jd\_ptr} =& \begin{bmatrix}0\ \ 5\ \ 9\ \ 10\\\end{bmatrix}\\[-0.75mm]
\text{\bf nzcnt} =& \begin{bmatrix}3\ \ 2\ \ 2\ \ 2\ \ 1\end{bmatrix}
\end{align*}
\end{minipage}
\hfill

\vspace{0.8em}
\hrule
\vspace{0.3em}

\includegraphics[width=\linewidth]{figures/spmvjdswhat.pdf}
\vspace{-1.5em}
\lstset {
 basicstyle=\linespread{1.0}\small\ttfamily
}
\begin{lstlisting}[language=IDL,firstnumber=7]
([\dots])
  inherits VectorStore
      with {outer_loop} as {scope}
       and {perm.value} as {input_index} at {output} and
  inherits VectorRead
      with {outer_loop}          as {scope}
       and {outer_loop.iterator} as {input_index} at {perm} and
  inherits VectorRead
      with {outer_loop}      as {scope}
       and {inner_loop.src1} as {value}
       and {tmp1.value}      as {input_index} at {val} and
  inherits Addition
      with {jd_ptr.value}        as {input}
       and {outer_loop.iterator} as {addend} at {tmp1} and
  inherits VectorRead
      with {outer_loop}          as {scope}
       and {inner_loop.iterator} as {input_index} at {jd_ptr} and
  inherits VectorRead
      with {outer_loop}      as {scope}
       and {inner_loop.src2} as {value}
       and {col_ind.value}   as {input_index} at {vector} and
  inherits VectorRead
      with {outer_loop} as {scope}
       and {tmp1.value} as {input_index} at {col_ind} and
  inherits ReadForLoopIterations
     with {outer_loop}          as {scope}
      and {inner_loop}          as {for}
      and {outer_loop.iterator} as {input_index} at {read_range}
\end{lstlisting}
\caption{Jagged Diagonal Storage in IDL:
         The top section of the figure shows the different involved arrays.
         The pseudocode in the middle row of the
         figure then directly gives a completion of \Cref{spmvbase}.
         Walking through the expressions and emitting IDL code on-by-one is
         sufficient.}
\label{jds_lilacwhat_fig}
\end{figure}

\subsubsection{Compressed Sparse Row}
For Compressed Sparse Row (CSR) \cite{doi:10.1137/1.9780898718003}, all non-zero
entries are stored in a flat array \textbf{val}.
The \textbf{col\_ind} array stores the column position for each value.
Finally, the \textbf{row\_ptr} array stores the beginning of each row of the
matrix as an offset into the other two arrays.
The number of rows in the matrix is given directly by the length of the
\textbf{row\_ptr} array minus one, however the number of columns is not
explicitly stored.
In \autoref{csr_lilacwhat_fig}, a 5x5 matrix is shown represented in this
format.

\subsubsection{Jagged Diagonal Storage}
For Jagged Diagonal Storage (JDS) \cite{doi:10.1137/0910073}, the rows of the
matrix are permuted such that the number of nonzeros  per row  decreases. The
permutation is stored in a vector \textbf{perm}, the number of nonzeros in
\textbf{nzcnt}.
The nonzero entries are then stored in an array \textbf{val} in the following
order: The first nonzero entry in each row, then the second nonzero entry in
each row etc.
The array \textbf{col\_ind} stores the column for each of the values and
\textbf{jd\_ptr} stores offsets into \textbf{val} and \textbf{col\_idx}.
The product of a sparse matrix in JDS format with a dense vector is specified 
in LiLAC-What at the bottom of \autoref{jds_lilacwhat_fig}.

\subsection{Dense Linear Algebra}

    The generalized matrix multiplication idiom is described in \Cref{fig:gemm}.
    The control flow is captured by three nested for loops.
    Inside these loops, the memory access is characterized by three matrix
    accesses, each with a different subset of the loop iterators.
    The corresponding \texttt{MatrixRead} and \texttt{MatrixWrite} idioms model
    generic access to matrices allowing strides, transposed matrices etc.
    The actual computation is encapsulated by the \texttt{DotProductLoop} idiom.
    This also contains the linear combination with factors \texttt{alpha} and
    \texttt{beta} that is part of the generalized matrix multiplication.

    The sparse matrix vector multiplication defined in \Cref{csr_lilacwhat_fig}
    is different to the other idioms in that
    the control flow of the skeleton of the idiom does not consist of perfectly
    nested for loops.
    Instead, the iteration space of the inner loop is read from an array using
    the \texttt{ReadRange} idiom.
    The actual computation that SPMV performs is a dot product and thus it uses
    the same \texttt{DotProductLoop} idiom as
    \texttt{GEMM} but the memory access pattern is different, with indirect
    memory access in \texttt{indir\_read}.

\begin{figure}[h]
\begin{lstlisting}[language=IDL,
                   label={fig:gemm},caption={IDL specification of GEMM}]
Constraint GEMM
( inherits ForNest(N=3) and
  inherits MatrixStore
      with {iterator[0]} as {col}
       and {iterator[1]} as {row}
       and {begin} as {begin} at {output} and
  inherits MatrixRead
      with {iterator[0]} as {col}
       and {iterator[2]} as {row}
       and {begin} as {begin} at {input1} and
  inherits MatrixRead
      with {iterator[1]} as {col}
       and {iterator[2]} as {row}
       and {begin} as {begin} at {input2} and
  inherits DotProductLoop
      with {loop[2]}        as {loop}
       and {input1.value}   as {src1}
       and {input2.value}   as {src2}
       and {output.address} as {update_address})
End
\end{lstlisting}
\end{figure}

\subsection{Stencils}
    \Cref{fig:stencilcompute} shows the base version of the stencil idiom.
    Stencils consist of a loop nest with a multi-dimensional memory access to
    store the updated cell value.
    The updated value is computed by a kernel function using a number of
    values that are constrained by
    \texttt{StencilRead}, which specifies multidimensional array access
    with only constant offsets in all dimensions.

\begin{figure}[t]
\begin{lstlisting}[language=IDL,
                   label={fig:stencilcompute},caption={IDL specification of simple stencil}]
Constraint Stencil
( inherits ForNest and
  inherits PermMultidStore
      with {iterator} as {input}
       and {begin} as {begin} at {write} and
  collect i
  ( inherits StencilRead
      with {write.input_index} as {input}
       and {kernel.input[i]} as {value}
       and {begin} as {begin} at {reads[i]}) and
  {kernel.output} is first argument of {write.store} and
  inherits KernelFunction
      with {begin}      as {outer}
       and {body.begin} as {inner} at {kernel})
End
\end{lstlisting}
\end{figure}

\subsection{Not Syntactic Pattern Matching}
The idiom descriptions may at first appear to be shallow syntactic pattern matching.
In fact, because it operates on the IR level, it can detect idioms that are written in superficially distinct style but are semantically equivalent.
For example, there are two syntactically distinct programs in \Cref{fig:gemmexamples}, which in fact are both implementations of general matrix multiplication.
The IDL in \Cref{fig:gemm} discovers they are both instances of GEMM and they can both be replaced with an API call to GEMM.

\begin{figure}[ht]
\begin{lstlisting}[language=MyCpp]
for (int mm = 0; mm < m; ++mm) {
  for (int nn = 0; nn < n; ++nn) {
    float c = 0.0f;
    for (int i = 0; i < k; ++i) {
      float a = A[mm + i * lda]; 
      float b = B[nn + i * ldb];
      c += a * b;
    }
    C[mm+nn*ldc] =
        C[mm+nn*ldc] * beta + alpha * c;
  }
}
\end{lstlisting}
\begin{lstlisting}[language=MyCpp,label={fig:gemmexamples},caption=
   {Two matching instances of GEMM}]
for(int i = 0; i < 1000; i++)
    for(int j = 0; j < 1000; j++) {
        M3[i][j] = 0.0f;
        for(int k = 0; k < 1000; k++)
           M3[i][j]+=M1[i][k]*M2[k][j]; }
\end{lstlisting}
\end{figure}

    There are limitations to this semantic matching.
    In particular, the use of low level manual optimizations that circumvent the
    usual IR representation, {\em e.g.}  SIMD compiler intrinsics, would distort
    the algorithms beyond recognition by our system.
    In practice, this is rarely encountered.

\newpage
\phantom{placeholder}
\newpage

\section{Compilation Process and Implementation}
\label{sec:compilation}

    Idiom definitions are compiled to C++ functions that perform idiom
    recognition on LLVM IR.
    In a first step, the compiler eliminates
    $\left<\text{\tt inheritance}\right>$, $\left<\text{\tt forall}\right>$,
    $\left<\text{\tt forsome}\right>$, $\left<\text{\tt if}\right>$,
    $\left<\text{\tt rename}\right>$ and $\left<\text{\tt rebase}\right>$.
    They are replaced with simpler $\left<\text{\tt conjunction}\right>$ and
    $\left<\text{\tt disjunction}\right>$ constructs.
    This also involves removing all parameterizations from the formula and
    flattening all variable names.
    Next, variables are collected and ordered to assist constraint solving.
    The ordering impacts performance, as it determines how well the search space
    is pruned. 
    For each variable, all the constraints associated with the variable are
    assembled.

    The compiler then emits C++ code which is passed to a generic solver based on \cite{ginsbach2017discovery} to search for idiom instances.
    This solver is based on standard backtracking.
    As shown in the results section, this increases compilation time, but the overhead is modest.

\section{Targeted Heterogeneous APIs}

    After idiom detection, we must transform the user program to exploit the
    relevant API.
    Two types of heterogeneous APIs are currently targeted: libraries and domain
    specific languages with their optimizing compilers.

    \subsection{Domain Specific Libraries}
    Libraries provide narrow interfaces but are often highly optimised.
    For example, the cuBLAS library is only suitable for a limited set of dense
    linear algebra operations and only works on Nvidia GPUs, but its
    implementation provides outstanding performance.
    For sparse linear algebra we use the vendor provided cuSPARSE, clSPARSE, and
    MKL libraries.
    For dense BLAS routines cuBLAS, clBLAS, CLBlast, and MKL are used.

    \subsection{Domain Specific Code Generators}
    Domain Specific Languages provide wider interfaces than libraries and allow
    problems to be expressed as composition of dedicated language constructs.
    An optimizing compiler then specializes the program for the target hardware.
    We currently support Halide and Lift as domain specific code generators.

    \paragraph*{Halide}~\cite{Ragan-Kelley2013Halide} is a language and
    optimizing compiler targeted at image processing applications.
    Optimised code is generated for CPUs as well as GPUs.
    Halide separates the functional description of the problem from the
    description of the implementation.
    This involves a separate execution \emph{schedule}.
    This allows retargeting of Halide programs to different platforms.
    We translate some of the stencil idioms and linear algebra idioms into
    Halide.
    Stencils involving control flow in their computations are not easily
    expressible in Halide.

    \paragraph*{Lift}~\cite{steuwer15rewrite, SteuwerRD17, HagedornSSGD18} is an
    optimizing code generator based on rewrite rules.
    The Lift language consists of functional parallel patterns such as
    \emph{map} and \emph{reduce} which  express a range of parallel
    applications.
    For this work we translate stencil idioms, complex reductions and linear
    algebra idioms to Lift.

\section{Translating Computational Idioms}

    This section describes how the detected idioms are mapped to the previously
    described library APIs domain specific languages.
    The two types of APIs (library interfaces and domain specific languages) are
    treated individually.

\subsection{Library}

    For library call interfaces, the original code is removed and an appropriate
    function call is inserted.
    The solution that is generated by the solver using the IDL program contains
    both the IR instructions to remove as well as the arguments that are to be
    used for the function call.

    For example, in the case of the {\tt GEMM} program that was shown in
    \Cref{fig:gemm}, the original code is removed by deleting the IR
    instruction at {\tt output.store\_instr} explicitly, which captures the
    store instruction of the {\tt MatrixStore} subprogram.
    The remaining cleanup is left to the standard dead code elimination pass.
    The arguments that specify the matrix dimensions are taken from
    {\tt ForNest} in combination with the stride and offset determined by
    {\tt MatrixRead} and {\tt MatrixWrite}.

    The mapping of solution variables to the arguments of the generated function
    call needs to be implemented individually for each backend, as we have no
    way to describe it in IDL itself.
    Once the code is replaced, LLVM continues with code generation as usual.

\subsection{DSL}

    For domain specific languages, the situation is a bit more involved.
    Reduction, histogram and stencil idioms are higher order functions that
    contain a kernel function or reduction operator that has to be represented
    for the DSL.

    For each combination of idiom and DSL there is a parameterized
    skeleton program.
    This skeleton is then specialized for the appropriate data types and numeric
    parameters as well as the kernel function or reduction operator.

    Numerical parameters are picked from the constraint solution in the same way
    that was described previously for library call interfaces.
    Also from the constraint solution, we have the loop body that contains the
    kernel function or reduction operator, as well as the input values and the
    result value used.
    We use this information to cut out the kernel function that is then used to
    generate code appropriate for the DSL backends:

\paragraph*{Lift}  expects stencil kernels or reduction operators to be sequential C code with a specific function interface that
is used internally by Lift when generating OpenCL code.
We therefore implemented a rudimentary LLVM IR to C backend for generating this function.

\paragraph*{Halide} is a language embedded in C++, it requires a syntax tree of the kernel functions built using a class hierarchy.

\begin{figure}[ht]
\begin{lstlisting}[language=LIFT,escapechar=|,
                   label={fig:liftmxm},caption=
   {Example of matrix multiplication in Lift}]
float mult(float x, float y) { return x*y; }
float add(float x, float y) { return x+y; }

gemm_in_lift(A, B, C, alpha, beta) {
 map(fun(a_row, c_row) {
  map(fun(b_col, c) {
   map(fun(ab){ add(mult(alpha, ab), mult(beta, c))},
    |\label{line:dot}|reduce(add, 0.0f, map(mult, zip(a_row, b_col))))
  }, zip(transpose(B), c_row))
 }, zip(A, C))
}
\end{lstlisting}
\end{figure}

\paragraph*{}
After code for the DSLs is generated, it is passed to the DSL code generator.
\Cref{fig:liftmxm} shows an example of the Lift code generated for GEMM (\texttt{gemm\_in\_lift}).
It performs a dot product (expressed in line~\ref{line:dot} using the Lift skeletons \texttt{zip}, \texttt{map}, and \texttt{reduce}) for each row of matrix A (\texttt{a\_row}) and column of matrix B (\texttt{b\_col}).
This code is compiled by Lift into optimised OpenCL code.

Finally, we again replace the idiom code in the user's code with a call to the code generated by the DSL and continue once again with LLVM code generation.

\subsection{Aliasing}

    Since idiom detection works statically, we are unable to fully rule out
    aliasing of pointers, which can make transformations unsound.
    For dense linear algebra this is easily solved with some basic run time
    checks for non-overlapping memory.
    However, for sparse linear algebra this is not as straightforward and in
    corner cases our approach is unsound.
    In practice this did not cause problems on any of the benchmark programs,
    however this means that optimizations based on these techniques will have to
    provide appropriate feedback to the programmer.

\section{Experimental Setup}

    \paragraph*{Benchmarks}
    We applied our approach to all of the sequential C/C++ versions of the NAS
    Parallel Benchmarks.
    We use the SNU NPB implementation by the Seoul National University,
    containing the original 8 NAS benchmarks plus two of the newer unstructured
    components UA and DC.
    We also evaluated our approach on all Parboil benchmarks, giving 21 programs
    in total. 

    \paragraph*{Platform and Evaluation}
    We use an AMD A10-7850K APU with a multi-core CPU and an integrated Radeon
    R7 GPU on the same die using driver version 1912.5, as well as an Nvidia GTX
    Titan X as an external GPU using driver version 375.66.
    We report the median runtime of 10 executions for each program.

    \paragraph*{Alternative detection approaches}
    \hspace{0.2cm}There are no easily available compilers to compare against
    that perform idiom detection.
    Instead, we consider two parallelizing compilers and examine
    whether they detect idioms as part of their parallelization approach.
    As their goal is parallelization and not idiom detection, this should be
    borne in mind in the results section.

    Polly \cite{Doerfert2015Polly} is an LLVM based polyhedral compiler
    capable of finding parallel loops and reductions in Static Control
    Parts (SCoP) of programs.  This allows comparison against
    another approach that uses the same compiler infrastructure.
    We gathered the SCoPs that Polly detected with the options
    \texttt{-O3 -mllvm -polly -mllvm -polly-export} and manually inspected
    the reported SCoPs for stencil like parallel loops and reduction operations.
    When Polly captured such a loop as a SCoP, we counted it as an idiom
    detection, although Polly itself has no concept of idioms.
    This gives an optimistic estimate as to what idiom coverage a polyhedral
    based approach can achieve.

    The Intel C++ Compiler (ICC) is a mature industry strength compiler that
    provides a detection mechanism for parallelizing reduction idioms based on
    data dependency analysis.
    We use the \texttt{-parallel -qopt-report} command line options and checked
    in the optimization report files whether the corresponding loop is
    considered parallelizable.

\newpage
\section{Results}
\label{sec:idiomresults}

    The approach was evaluated in several steps.
    Firstly, the number of detected idioms and their distribution over the
    benchmark programs was established.
    During this analysis, the run time of the IDL-enabled Clang compiler was
    measured, and the compile time overhead of the solver over standard
    compilation evaluated.
    Next, the run time coverage of the idioms was determined for each
    benchmark program, to see where exploitation might be beneficial.

    Where run time coverage was substantial, speedups over sequential C code
    are reported.
    Detailed results capture the performance of each trageted backend interface.
    Finally, the evaluation includes comparison to the handwritten OpenMP and
    OpenCL implementations that are included with the benchmark suites as
    reference implementations.
    These provide a suitable estimator for the upper bound of available
    performance.

\begin{figure}[t]
  \centering
  \includegraphics[width=\textwidth]{figures/asplosplots/detection.pdf}
  \caption{The different computational idioms found in all benchmarks.}
  \label{detection-figure}
\end{figure}
\begin{figure}[t]
  \centering
  \includegraphics[width=\textwidth]{figures/asplosplots/coverage.pdf}
  \caption{Runtime coverage of the detected idioms in all benchmarks.}
  \label{coverage-figure}
  \vspace{0.5em}
\end{figure}

\subsection{Idiom Detection}

    \Cref{tab:detection} shows the number of computational idioms found by IDL,
    Polly, and ICC.
    Polly found 3 scalar reductions and 6 stencils while ICC recognised only
    28 scalar reduction.
    Polly was unable to perform idiom specific optimizations on GEMM.
    Other approaches did not detect any histograms or sparse matrix operations,
    because such code involves indirect and thus non-affine memory accesses.
    This fundamentally contradicts assumptions that these tools rely on and is
    not merely an implementation artifact.
    Our IDL approach detects 60 idioms overall with the compile time cost shown
    in figure \Cref{tab:compiletimecost}.
    On average, the compilation time is increased by 82\%, which can be reduced
    further by optimizing the solver.

    \Cref{detection-figure} shows the different idioms detected across
    the  benchmarks. We detect both scalar
    and histogram reductions as well as stencils, dense matrix operations
    and sparse matrix-vector multiplication.
    Polly and ICC are only capable of detecting simple scalar
    reductions, but we are able to detect histogram reductions, {\em e.g.} in
    the \emph{histo} benchmark as well.  For stencils, Polly detects two
    in \emph{lbm} and \emph{stencil} while our approach
    detects all the stencils in \emph{lbm}, \emph{stencil} and \emph{MG}.
    Unlike any existing approach, we detect sparse matrix-vector
    operations in {\emph CG} and {\emph spmv} as well as dense matrix
    operations in \emph{sgemm}. It is worth repeating, however, that both
    Polly and ICC are parallelizing compilers, not idiom recognition
    tools.

\begin{table}[t]
\centering
  \begin{tabular}{lP{2.16cm}P{2.16cm}P{2.16cm}P{2.163cm}P{2.163cm}}
  \toprule
  \hspace{1.18cm} & Scalar\newline{}Reduction & Histogram\newline{}Reduction & Stencil & Matrix~Op. & Sparse\newline{}Matrix~Op.\\
  \midrule
  Polly &  3  &  --- &   5  &  --- & --- \\
  ICC   &  28 &  --- &  --- &  --- & --- \\
  IDL   &  45 &   5  &   6  &   1  &  3  \\
  \bottomrule
\end{tabular}
\caption{Idioms detected by IDL, ICC, Polly}
\label{tab:detection}
\end{table}

\subsection{Runtime Coverage}

    To determine if the detected idioms are actually important,
    \Cref{coverage-figure} shows the percentage of time spent in the detected
    computational idiom.
    This data shows that either the detected idioms have a low runtime
    contribution or they dominate almost the entire execution.
    \emph{EP} is the only exception where about 50\% of the runtime is spent
    inside a detected histogram reduction.
    We focus on the 10 programs which spend a significant amount of time in the
    detected idioms, as only these can reasonably expect a performance gain
    using our approach.

\subsection{Performance Results}

\begin{table}[t]
\centering
\begin{tabular}{lcccccccccccc}
  \toprule
  & \hspace{0.17mm}BT\hspace{0.17mm}
  & \hspace{0.17mm}CG\hspace{0.17mm}
  & \hspace{0.17mm}DC\hspace{0.17mm}
  & \hspace{0.17mm}EP\hspace{0.17mm}
  & \hspace{0.17mm}FT\hspace{0.17mm}
  & \hspace{0.17mm}IS\hspace{0.17mm}
  & \hspace{0.17mm}LU\hspace{0.17mm}
  & \hspace{0.17mm}MG\hspace{0.17mm}
  & \hspace{0.17mm}SP\hspace{0.17mm}
  & \hspace{0.17mm}UA\hspace{0.17mm}
  & \hspace{0.17mm}bfs\hspace{0.17mm}
  & \hspace{0.17mm}cutcp\hspace{0.17mm} \\
  \midrule
without IDL    & 1.9 & 0.5 & 1.0 & 0.3 & 0.6 & 0.3 & 1.9 & 0.8 & 1.6 & 2.7 & 0.4 & 0.4 \\[0.25em]
with IDL       & 4.0 & 0.8 & 1.6 & 0.6 & 1.2 & 0.5 & 3.9 & 4.5 & 3.2 & 7.3 & 0.5 & 0.6 \\[0.75em]
overhead in \% & 116 &  77 &  57 &  77 &  93 &  62 & 103 & 484 &  97 & 169 &  30 &  65 \\
  \bottomrule
\end{tabular}
\vspace{5mm}

\begin{tabular}{lccccccccc}
  \toprule
  & \hspace{0.44mm}histo\hspace{0.44mm}
  & \hspace{0.44mm}lbm\hspace{0.44mm}
  & \hspace{0.44mm}mri-g\hspace{0.44mm}
  & \hspace{0.44mm}mri-q\hspace{0.44mm}
  & \hspace{0.44mm}sad\hspace{0.44mm}
  & \hspace{0.44mm}sgemm\hspace{0.44mm}
  & \hspace{0.44mm}spmv\hspace{0.44mm}
  & \hspace{0.44mm}stencil\hspace{0.44mm}
  & \hspace{0.44mm}tpacf\hspace{0.44mm} \\
  \midrule
without IDL    & 0.2 & 0.3 & 0.2 & 0.2 & 0.4 & 0.6 & 0.3 & 0.2 & 0.2 \\[0.25em]
with IDL       & 0.2 & 0.6 & 0.4 & 0.3 & 0.6 & 0.7 & 0.7 & 0.2 & 0.4 \\[0.75em]
overhead in \% &  35 &  87 & 100 &  52 &  58 &  24 & 115 &  36 &  54 \\
  \bottomrule
\end{tabular}
\caption{Compile time cost in seconds}
\label{tab:compiletimecost}
\end{table}

\paragraph*{Speedup vs. Sequential}
\Cref{fig:speedup-figure} shows the end-to-end speedup obtained by accelerating idioms with heterogeneous APIs on a CPU, an integrated GPU, and an external GPU.
All results include data transfer overhead to and from the GPUs.
Here the best performing API is shown;
\Cref{tab:detailed-results} provides detailed results for all APIs. 

\begin{figure}[t]
  \centering
  \includegraphics[width=\textwidth]{figures/asplosplots/speedup_vs_sequential_wide.pdf}
  \caption{Speedup over sequential:
           Results for the best-performing backend on each platform are shown.
           The red bars indicate a manual modification for minimising redundant
           data transfers.}
  \label{fig:speedup-figure}
  \vspace{1.5em}
  \centering
  \includegraphics[width=\textwidth]{figures/asplosplots/comparison.pdf}
  \caption{IDL versus manual expert parallelisation:
           Speedup over the sequential baseline was measured for IDL
           (selecting best performing backend; red bars) and the
           handwritten reference OpenCL and OpenMP implementations
           (provided by the benchmark developers; yellow bars).}
  \label{fig:speedup-figure-2}
  \vspace{0.5em}
\end{figure}

    For five benchmarks we obtain moderate speedups from 1.26$\times$ for
    \emph{histo} up to 4.5$\times$ for \emph{IS}.
    All of these benchmarks besides \emph{MG} have a scalar or histogram
    reduction as their performance bottleneck and are, therefore, not
    computationally expensive.
    Interestingly, we can see that for different benchmarks, different hardware
    is beneficial:
    for \emph{tpcaf}  the CPU is the best platform, beating the GPU  for which
    the data transfer time dominates;
    for \emph{MG} and \emph{histo} the integrated GPU strikes the right balance
    between computational power while avoiding the movement of data to the
    external GPU;
    and, finally, for \emph{EP} and \emph{IS} the data transfer to the GPU pays
    off exploiting the high GPU internal memory bandwidth.
    These results emphasize the significance of heterogeneous code generation
    flexibility.

For five of the benchmarks we achieve significantly higher performance gains, from 17$\times$ for \emph{CG} and up to over 275$\times$ for \emph{sgemm}.
These benchmarks are computationally expensive and the external GPU is always the fastest architecture by a considerable margin.

\begin{landscape}
\newlength{\txtwd}
\newcommand{\msb}[1]{\settowidth{\txtwd}{#1}{\tiny\ttfamily\bfseries \hfill #1}}
\newcommand{\ms}[1]{\settowidth{\txtwd}{#1}{\tiny\ttfamily \hfill #1}}
\addtolength{\tabcolsep}{-0.64mm}
\begin{table}[p]
  \centering
  \small
  \begin{tabular}{|l||cccccc|ccccc|cccc|}
  \hline
  & \multicolumn{6}{c|}{\bfseries\large CPU} & \multicolumn{5}{c|}{\bfseries\large iGPU} & \multicolumn{4}{c|}{\bfseries\large GPU} \\
  & MKL & libSPMV & Halide & clBLAS & CLBlast & Lift & clSPARSE & libSPMV & clBLAS & CLBlast & Lift & cuSPARSE & libSPMV & cuBLAS & Lift \\
  \hline
  \hline
   CG      & \msb{1504.21} & --- & --- & --- & --- & --- & \msb{644.02} & --- & --- & --- & --- &  \msb{113.51} & --- & --- & --- \\[3mm]
   EP      & --- & --- & --- & --- & --- &  \msb{32762.50}  & --- & --- & --- & --- & \msb{30983.40}  & --- & --- & --- & \msb{24680.70} \\[3mm]
   IS      & --- & --- & \msb{426.95} & ---  & --- & \ms{1765.61}  &  --- & --- & --- & --- & \msb{547.28}  & --- & --- & --- & \msb{99.95} \\[3mm]
   MG      & --- & --- & --- & --- & --- &  \msb{4699.63}  & --- & --- & --- & --- & \msb{1439.58}  & --- & --- & --- & \msb{2211.56} \\[3mm]
   histo   & --- & --- & --- & --- & --- &  \msb{27.42}  & --- & --- & --- & --- & \msb{17.20}  & --- & --- & --- & \msb{19.54} \\[3mm]
   lbm     & --- & --- & --- & --- & --- &  \msb{6457.93}  & --- & --- & --- & --- & \msb{5335.09}  & --- & --- & --- & \msb{590.60} \\[3mm]
   sgemm   & \msb{53.50} & --- & --- & \ms{1661.75} & \ms{660.44} & \ms{1339.15}  & --- & --- & \msb{14.73} & \ms{19.03} & \ms{15.04}  & --- & --- & \msb{5.99} & \ms{7.87} \\[3mm]
   spmv    & --- & \msb{218.17} & --- & --- & --- & --- & --- &\msb{102.233} & --- & --- & --- & --- &\msb{18.437} & --- &  ---\\[3mm]
   stencil & --- & ---& \msb{5760.81} & --- & --- & \ms{21951.80}  & --- & ---& --- & --- & \msb{2261.48} & --- & ---& --- & \msb{279.38} \\[3mm]
   tpacf   & --- & ---& --- & --- & --- & \msb{19276.40}  & --- & ---& --- & --- & \msb{61111.90} & --- & ---& --- & \msb{23358.20} \\
  \hline
\end{tabular}
\caption{Detailed performance results for each heterogeneous backend interface:
         The run time of benchmark program is measured in milliseconds for every
         compatible combination in each platform.
         The fastest implementations for each benchmark and target hardware are
         highlighted in bold.}
\label{tab:detailed-results}
\end{table}
\end{landscape}


The red highlighting in the plot indicates an important runtime optimization:
redundant data transfers for the iterative \emph{CG}, \emph{lbm}, \emph{spmv} and \emph{stencil} benchmarks.
All of these benchmarks execute computations inside a for loop and do not require access to the data on the CPU between iterations.
We manually applied a straightforward lazy copying technique by flagging memory objects to avoid redundant transfers, similar to~\cite{jablin11automatic}.
As can be seen this runtime optimization is crucial for achieving high performance for these benchmarks.

\paragraph*{API performance comparison}
\Cref{tab:detailed-results} shows a breakdown of the performance of each API on each program and platform.
Not all APIs
target all platforms, {\emph{e.g.} cuSPARSE only targets NVIDIA GPUs and in the
case of Halide, the current version that we have access to failed to generate
valid GPU code for any of the benchmarks we tried.
The best performing API is highlighted in bold in the table entries.
The \emph{spmv} benchmark uses an unusual sparse matrix format, so that we
implemented a custom library libSPMV for this benchmark.

On the multicore CPUs, the Intel MKL library gives the best linear algebra performance, outperforming the other libraries and Lift.
Halide achieves good performance for the NPB \emph{IS} and Parboil \emph{stencil} benchmarks on the CPU, outperforming Lift due to its more advanced vectorization capabilities.
In the programs where scalar reductions dominate, Lift performs well.
On the iGPU, clBLAS provides a better matrix-multiplication implementation than either CLBlast or Lift.
On the external GPUs, the libraries provide better linear algebra implementations, while Lift performs well on stencils and reductions.

\paragraph*{Speedup vs. Handwritten Parallel Implementations}
\Cref{fig:speedup-figure-2} shows the performance of our approach compared to hand-written reference OpenMP and OpenCL implementations.
For some of the benchmarks, the parallel versions are significantly modified using different algorithms beyond the domain of automation.
We can see that for benchmarks where the handwritten implementation does not make algorithmic changes (\emph{CG}, \emph{histo}, \emph{lbm}, \emph{sgemm}, \emph{spmv}, \emph{stencil}), we achieve comparable -- or better -- performance.
For four benchmarks (\emph{EP}, \emph{IS}, \emph{MG}, and \emph{tpacf}) it is beneficial to parallelize the entire application -- which is beyond the scope of this paper. Future work will examine outer loop parallelism as an idiom to exploit.

For the \emph{sgemm} and \emph{stencil} benchmarks we improved the baseline implementation provided by the benchmarks as these had extremely poor performance.
A simple interchange of two loops improved performance by almost 20 times.

\paragraph*{Summary}

    60 idioms were detected across the benchmark suites and significant
    performance improvements were achieved by targeting different heterogeneous
    APIs for those benchmarks where idioms dominate execution time.

\section{Conclusion}

    This paper develops a compiler based approach that automatically
    detects a wide class of idioms supported by libraries or domain
    specific languages for heterogeneous processors. This approach is
    based on a constraint based description language that identifies
    program subsets that adhere to idiom specifications.  Once
    detected, the idioms are mechanically translated into API calls to
    external libraries or code generated by DSL compilers.

    The approach is robust and was evaluated on C/C++ versions of two well
    known benchmark suites: NAS and Parboil. We detected more stencils,
    sparse matrix operations and generalized reductions and histograms than
    existing approaches and generated fast code.

    Future work will extend the constraint formulation to consider other common
    idioms.
    As the number of idioms detected and of implementations available grows, a
    smart profitability analysis will be needed and is the subject of future
    work.



\chapter[Building a Fully integrated Idiom Specific Optimization Pipeline]
        {Building a Fully integrated Idiom Specific Optimization Pipeline
         \footnote{This chapter is based on published research in
                   \citet{lilacpaper}.}}
    \label{chapter:lilac}
    
    This thesis introduced a constraint programming methodology that operates on
    SSA compiler intermediate representation.
    The Compiler Analysis Description Language (CAnDL) and the extended Idiom
    Detection Language (IDL) were developed, and implemented in the LLVM
    framework.
    This made the constraint programming method available for program analysis.

    Several computational idioms were specified using CAnDL and IDL, enabling
    automatic recognition of adhering user code sections during compilation.
    The well-studied kernels among these idioms were stencils and varied
    forms of sparse and dense linear algebra.
    Complementing these established kernels, Complex Reduction and Histogram
    Computations (CReHCs) were introduced as a new grouping of computations.
    The evaluation on the established benchmark suites NPB and Parboil demonstrated
    that all of these computational idioms covered significant performance
    bottlenecks.

    Recognising these computational idioms enabled generic compilers to apply
    idiom-specific optimising transformations, which are traditionally available
    only within domain-specific tools.
    Such transformations included the automatic parallelisation of programs that
    were inaccessible to previous analysis approaches.

    Given this background, the idiom detection was performed on sequential C/C++
    programs, achieving automatic heterogeneous parallelisation.
    Idiomatic loops in the code were redirected to domain-specific code
    generators.
    These specialised tools leveraged the domain knowledge that is available for
    code in restrictive idioms.
    This approach was applicable to 10 of the 21 benchmark programs from NPB and
    Parboil, resulting in speedups between 1.26$\times$ and 275$\times$ over the
    sequential baseline versions.

    In summary, the thesis demonstrated that computational idioms are a
    suitable interface to heterogeneous acceleration and developed the
    methodology for automatically recognising such idioms.
    The main contributions are the method of constraint programming on SSA
    intermediate representation, the design and implementation of CAnDL and IDL,
    and the identification of generalised reductions as a significant class of
    benchmark bottlenecks.

\section{Contributions}

\subsection*{Constraint Programming on SSA Code}

    \Cref{chapter:theory} introduced constraint programming on
    SSA compiler intermediate representation.
    Using a characterisation of the static structure of SSA programs,
    {\it SSA constraint problems} were defined.
    These are formulas that impose restrictions on compiler intermediate code,
    turning the detection of adhering program parts into a constraint
    satisfiability problem.
    This thesis derived efficient algorithms for solving SSA constraint problems
    and discussed several significant types of constraint formulas, reflecting
    compiler analysis methods such as data flow and dominance relationships.

    The constraint programming methodology was derived starting from algebraic
    formulation through to developing the implementation in C++.
    It applies to any SSA compiler intermediate representation and is suitable
    for a wide range of analysis problems.

\subsection*{CAnDL and IDL}

    \Cref{chapter:candl,chapter:reductions} designed and implemented two
    related declarative programming languages: the Compiler Analysis
    Description Language (CAnDL) and the extended Idiom Detection Langauge
    (IDL).
    These languages make the constraint programming methodology available in the
    LLVM framework.
    This enables constraint programming on the SSA intermediate code of user
    programs during compilation with the Clang C/C++ compiler.

\subsection*{Complex Reduction and Histogram Computations}

    Complex Reduction and Histogram Computations (CReHCs) were introduced in
    \Cref{chapter:reductions} as a computational idiom.
    CReHCs are a generalisation of the well-understood scalar reduction that
    covers the indirect array accesses typically found in histogram
    calculations.
    This type of loop was not previously studied, but this thesis showed that
    shared parallelisation opportunities exist.
    Furthermore, an evaluation of established benchmark suites demonstrated
    that several of the performance bottlenecks in each of NPB, Parboil, and
    Rodinia are, in fact,  CReHCs.

\subsection*{Formulation of Computational Idioms}

    \Cref{chapter:candl,chapter:reductions,chapter:idioms} formulated a range of
    different computational idioms in CAnDL and IDL.
    These idioms included stencils, different forms of sparse and dense linear
    algebra, polyhedral Static Control Parts, and CReHCs.
    The formulations enabled the automatic recognition of high-level algorithmic
    structure during compilation, making it possible for generic compilers to
    apply domain-specific reasoning.

\subsection*{Idiomatic Heterogeneous Acceleration Pipeline}

    \Cref{chapter:idioms} implemented an IDL-based integrated heterogeneous
    acceleration pipeline.
    The code sections recognised by IDL specifications were redirected to
    domain-specific code generators for heterogeneous accelerator hardware.
    This resulted in significant parallelisation speedups on benchmark
    programs.

\section{Critical Analysis}

    The approaches of this thesis were built on the derivation of
    \Cref{chapter:theory} and then evaluated in diverse scenarios on C/C++
    program code.
    Despite the effort of bridging between the algebraic formulation and the
    application in real-world scenarios, the work remains a prototype, and
    several issues need to be addressed before it becomes viable for productive
    use.
    Importantly, this includes questions about the prevalence of idiomatic code,
    the affordability of significant compile time overhead, and the ability to
    compensate fundamental limitations of the underlying solver technology.

\subsection*{Generality of Computational Idioms}  

    Computational idioms were defined in
    \Cref{chapter:reductions,chapter:idioms} and
    evaluated successfully on a range of established benchmark collections.
    However, the NPB and Parboil collections both are from areas of scientific
    computing.
    It remains unclear how generally applicable these idioms are on codebases
    from other domains.
    Similar concerns are faced by competing approaches, including the
    polyhedral model.

    More generally, it is unclear how much code in large-scale applications
    could be captured as ``idiomatic'' with even a very range and diverse set of
    idioms.
    Even some benchmark programs within NPB and Parboil contain idiosyncratic
    computations that are unlikely to reoccur in other contexts.
    An example is the ``sad'' program that computes the
    ``Sum of Absolute Differences'' algorithm used by the reference H.264 video
    encoder.

\subsection*{Compile Time Cost}

    The compile-time cost of idiom detection was evaluated in
    \Cref{chapter:idioms}, showing that overheads between 35$\%$ and 115$\%$
    occurred across the benchmark programs.
    Given that the approach is built on constraint satisfiability methods, this
    is a reassuring result, as the compile times remain within one order of
    magnitude.
    Nonetheless, from the perspective of compiler optimisations, this might be
    a prohibitive cost.
    Moreover, the compile-time overhead is unevenly distributed, with
    disproportionately longer solver times on large functions and especially on
    code sections that are near misses to satisfying the specifications.

\subsection*{Finite Number of Variables}

    CAnDL and IDL are limited to finite numbers of variables in underlying
    constraint problems.
    This results in upper bounds for the number of features many idioms:
    CReHCs can only contain a maximum of two histograms, stencil codes can only
    use a neighbourhood of up to 32 items, and so on.
    In practice, this involves a tradeoff between performance and the generality
    of the idiom specification, as each additional variable introduces a slight
    overhead for the solver.

\section{Future Work}

    Future work will focus on extending the approach to new idioms,
    improving the usability of specification languages, and
    applying dynamic analysis and machine learning techniques to complement the
    static analysis.
    Furthermore, constraint specifications will eventually be generated from
    code examples automatically, removing the difficulty of writing them
    manually.

\subsection*{Complementing with Dynamic and Machine Learning Approaches}

    The presented static methods will be complemented with dynamic approaches
    and machine learning.
    Such methods complement the entirely static reasoning of constraint
    programming on compiler immediate representation naturally but were
    outside the scope of this thesis.

    Dynamic analysis could be used to preselect candidate hot loops, drastically
    reducing the compile-time overhead.
    In particular, dynamic analysis could collect program features that are
    then fed into machine learning algorithms.
    The use of neural networks for guiding compiler optimisations has been
    successfully studied in the relevant literature
    \citep{DBLP:journals/pieee/WangO18}.
    Dynamic methods are also required to automate the efficient memory transfers
    that are required between heterogeneous participants, and to rule out
    pointer aliasing.

\subsection*{Infinite Number of Variables}

    Other research disciplines worked around some limitation of underlying
    solvers.
    For example, Bounded Model Checking \cite{Clarke:2001:BMC:510986.510987}
    uses SAT solvers and involves the complete unrolling of loops.
    In order to fit the underlying solver, this requires the introduction of a
    certain finite iteration limit $k$.
    The entire checking process can then be repeated with larger and larger
    values of $k$, successively ruling out more possible violations.

    The same approach could be applied to constraint solving on SSA:
    The solver is invoked repeatedly with successively increasing numbers of
    variables.
    When no \texttt{collect} statement exceeds variable capacity anymore, the
    process is terminated.
    This ensures that small solutions are found without unnecessary
    overhead, yet solutions that use large numbers of variables are not
    discarded entirely.
    No upper limits for collect statements are required with this scheme.

\subsection*{Pointer-chasing Idioms}

    The ability to express sparse linear algebra sets the detection ability of
    constraint programming on SSA immediate representation apart from previous
    approaches using the polyhedral model or data-flow analysis.
    Going beyond simple data access indirection, IDL will be used on data access
    patterns that involve pointer chases.
    This will enable it to detect graph operations, such as depth-first search
    and the PageRank algorithm.
    Furthermore, this approach could detect list traversal, insertion, and
    removal of elements.
    Such operations are not typically performance-critical but could be the
    first step to a deeper semantic understanding of programs in compilers.

\subsection*{Simpler Specification Languages}

    Compared to previous approaches, the specification languages CAnDL and IDL
    simplify the implementation of compiler analysis functionality and enable
    the detection of more complex idiomatic structures.
    However, profound knowledge of compiler internals is still required to
    write correct specifications.
    Furthermore, while idiom specifications are mostly independent of
    LLVM IR as the specific underlying intermediate representation, the precise
    extent of this independence has not been explored thoroughly.

    Future work will investigate languages that abstract the IR-specific nature
    of CAnDL and IDL, and provide a better programming experience.
    The pseudocode and corresponding IDL specifications in
    \Cref{csr_lilacwhat_fig,jds_lilacwhat_fig} in \Cref{chapter:idioms} already
    suggest how this can be achieved:
    For restricted domains (e.g.\ SPMV), generating IDL from high-level
    expressions is straightforward.

\subsection*{Generating Specifications by Example}

    Future work will also expand on the automatic generation of idiom
    specifications from example code fragments.
    Initial work has shown promise \citep{DBLP:conf/IEEEpact/CollieGO19}.
    Using a graph-matching algorithm that operates on SSA code, codes that
    implement variations of the same idiom will be matched together optimising
    a quality metric.
    This matching will automatically identify code structures that are shared
    between the examples and discard those features that were unique to
    specific samples.
    The common structures will then be separated and turned into constraint
    conditions.

    Suitable quality metrics for graph matching will be critical to the success
    of this approach and will be eventually be tuned with machine learning
    approaches, such as neural networks.
    The manually implemented idioms from this thesis will provide suitable
    training data.

    Eventually, the manual curation of example codes that group together into
    computational idioms could become redundant.
    Automatic profiling of large quantities of code will allow the automatic
    identification of all the relevant hot loops.
    The hot loops could then be grouped into computational idioms using
    automatic clustering algorithms that use a distance metric between SSA code
    based on a success score for the previously discussed graph matching
    approach.


\chapter{Conclusion}
    \label{chapter:conclusion}
    
\section{Contributions}

\section{Critical Analysis}

\section{Future Work}


\bibliographystyle{plainnat}
\bibliography{references}
\end{document}

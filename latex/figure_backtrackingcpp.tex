\begin{lstlisting}[language=MyCpp,
                   label={cppsolver},caption=
  {Complete C++ implementation of \Cref{backtrackalg}:
    The if-else statements in lines 19--41 precisely correspond to those
    in lines 5--17 of that algorithm.
    The \texttt{BacktrackingPart} objects iterate over $\protect{B_k[C](M,x)}$ via the
    \texttt{skip\_invalid} method.
    The remaining member function calls in lines
    $\text{23},\text{27},\text{29},\text{36}$ precompute structures that are
    used in \texttt{skip\_invalid} for quick evaluation.\parfillskip=0pt}]
// This class corresponds to Definition 2.7.
class BacktrackingPart {
public:
  virtual SkipResult skip_invalid(unsigned& c) const;
  virtual void begin();
  virtual void fixate(unsigned c);
  virtual void resume();
};

void yield(const vector<int>& solution);

// The solver function corresponds to Algorithm 1.
void solver(vector<BacktrackingPart*> P) {
    int         k = 0;
    vector<int> x(P.size(), 0);
    while(true) {
        SkipResult result = P[k]->skip_invalid(x[k]);

        if(result != SkipResult::FAIL) {
            if(k + 1 == P.size())
            {
                yield(x);
                P[k]->resume();
                x[k]++;
            }
            else {
                P[k]->fixate(x[k]);
                k++;
                P[k]->begin();
                x[k] = 0;
            }
        }
        else {
            if(k > 0) {
                k = k - 1;
                P[k]->resume();
                x[k] = x[k] + 1;
            }
            else
                return {};
        }
    }
}
\end{lstlisting}

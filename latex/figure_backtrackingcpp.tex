\begin{lstlisting}[language=MyCpp,
                   label={cppsolver},caption=
  {Complete C++ implementation of \Cref{backtrackalg}:
    The if-else statements at lines 19--41 precisely correspond to those
    at lines 5--17 of the algorithm.
    The ``\texttt{BacktrackingPart}'' objects iterate over $\protect{B_k[C](M,x)}$ via the
    ``\texttt{skip\_invalid}'' method.
    The remaining member function calls at lines
    $\text{23},\text{27},\text{29},\text{36}$ precompute structures
    used in ``\texttt{skip\_invalid}'' for quick evaluation.}]
// This class corresponds to Definition 2.10.
class BacktrackingPart {
public:
  virtual SkipResult skip_invalid(unsigned& c) = 0;
  virtual void begin() = 0;
  virtual void fixate(unsigned c) = 0;
  virtual void resume() = 0;
};

void yield(const vector<unsigned>& solution);

// This function corresponds to Algorithm 1.
void solver(vector<BacktrackingPart*> B) {
    unsigned         k = 0;
    vector<unsigned> x(B.size(), 0);
    while(true) {
        SkipResult result = B[k]->skip_invalid(x[k]);
        if(result == SkipResult::CHANGE) continue;
        if(result != SkipResult::FAIL) {
            if(k + 1 == B.size())
            {
                yield(x);
                B[k]->resume();
                x[k]++;
            }
            else {
                B[k]->fixate(x[k]);
                k++;
                [k]->begin();
                x[k] = 0;
            }
        }
        else {
            if(k > 0) {
                k = k - 1;
                B[k]->resume();
                x[k] = x[k] + 1;
            }
            else
                return;
        }
    }
}
\end{lstlisting}

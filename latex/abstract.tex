    The scaling limitations of multi-core processor development have led to a
    diversification of the processor cores used within individual computers.
    Heterogeneous computing has become widespread, involving the cooperation of
    several structurally different processor cores.
    Central processor (CPU) cores are most frequently complemented with graphics
    processors (GPUs), which despite their name are suitable for many highly
    parallel computations besides computer graphics.
    Furthermore, deep learning accelerators are rapidly gaining relevance.

    Many applications could profit from heterogeneous computing but are held
    back by the surrounding software ecosystems.
    Heterogeneous systems are a challenge for compilers in particular, which
    usually target only the increasingly marginalised homogeneous CPU cores.
    Therefore, heterogeneous acceleration is primarily accessible via libraries
    and domain-specific languages (DSLs), requiring application rewrites and
    resulting in vendor lock-in.

    This thesis presents a compiler method for automatically targeting
    heterogeneous hardware from existing sequential C/C++ source code.
    A new constraint programming method enables the declarative
    specification and automatic detection of {\em computational idioms} within
    compiler intermediate representation code.
    Examples of computational idioms are stencils, reductions, and
    linear algebra.
    Computational idioms denote algorithmic structures that commonly occur in
    performance-critical loops.
    Consequently, well-designed accelerator DSLs and libraries support
    computational idioms with their programming models and function interfaces.
    The detection of computational idioms in their middle end enables
    compilers to incorporate DSL and library backends for code generation.
    These backends leverage domain knowledge for the efficient utilisation of
    heterogeneous hardware.

    The constraint programming methodology is first derived on an abstract
    model and then implemented as an extension to LLVM.
    Two constraint programming languages are designed to target this
    implementation:
    the Compiler Analysis Description Language (CAnDL), and the extended
    Idiom Detection Language (IDL).
    These languages are evaluated on a range of different compiler problems,
    culminating in a complete heterogeneous acceleration pipeline integrated
    with the Clang C/C++ compiler.
    This pipeline was evaluated on the established benchmark collections NPB
    and Parboil.
    The approach was applicable to 10 of the benchmark programs, resulting in
    significant speedups from 1.26$\times$ on ``histo'' to 275$\times$ on
    ``sgemm'' when starting from sequential baseline versions.

    In summary, this thesis shows that the automatic recognition of
    computational idioms during compilation enables the heterogeneous
    acceleration of sequential C/C++ programs.
    Moreover, the declarative specification of computational idioms is derived
    in novel declarative programming languages, and it is demonstrated that
    constraint programming on Single Static Assignment intermediate
    code is a suitable method for their automatic detection.

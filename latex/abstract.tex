    The scaling limitations of multi-core processor development have led to a
    diversification of the processor cores used within individual computers.
    Heterogeneous computing has become widespread, involving the cooperation of
    several structurally different processor cores.
    Central processor (CPU) cores are most frequently complemented with graphics
    processors (GPUs), which despite their name are suitable for many highly
    parallel computations besides computer graphics.
    Furthermore, deep learning accelerators are rapidly gaining relevance.

    Many applications could profit from heterogeneous computing but are held
    back by the surrounding software ecosystems.
    Heterogeneous systems are a challenge for compilers in particular, which
    usually only target the homogeneous -- increasingly marginalised -- CPUs.
    Therefore, heterogeneous acceleration is primarily accessible via libraries
    and domain-specific languages (DSLs), requiring expensive application
    rewrites and resulting in vendor lock-in.

    This thesis presents a compiler approach to automatically target
    heterogeneous hardware from existing sequential C/C++ source code.
    A novel constraint programming methodology enables the declarative
    specification and automatic detection of {\em computational idioms} within
    compiler intermediate representation.
    Computational idioms denote algorithmic structure that is common in
    performance-critical loops.
    Examples of computational idioms are stencil codes, linear algebra,
    reductions and histogram computations.
    Well-designed accelerator DSLs and libraries support computational idioms
    with their programming models and function interfaces.
    The detection of computational idioms during compilation enables the
    compiler to offload code generation to these dedicated backends, which
    incorporate domain knowledge for efficient heterogeneous code generation.

    The constraint programming methodology is first derived on an abstract
    model, and then implemented as an extension of LLVM.
    Two constraint programming langauges are desiged for this
    extension:
    the Compiler Analysis Description Language (CAnDL), and the extended
    Idiom Detection Language (IDL).
    These languages are evaluated on a range of different compiler problems,
    culminating in a complete heterogeneous acceleration pipeline within the
    LLVM compiler framework.
    The pipeline was evaluated on established benchmark collections NPB and
    Parboil.
    The approach was effective on 10 of the benchmark programs, resulting in
    significant speedups from 1.26$\times$ on ``histo'' to 275$\times$ on
    ``sgemm'' when starting from sequential baseline versions.

    The automatic detection of computational idioms during compilation and the
    subsequent redirection of idiomatic program parts to domain-specific code
    generators can enable automatic heterogeneous acceleration.
    Constraint programming on compiler intermediate representation is a suitable
    method for formulating computational idioms that can be detected
    automatically with a constraint solver.
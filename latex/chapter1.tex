    The end of Moores Law and the end of Dennard Scaling have required new
    approaches in hardware.
    In response, accelerator processors have become widespread, first in the
    form of general purpose graphics processing units (GPGPU) and now
    increasingly with even more deep learning accelerators.

    Heterogeneous computing platforms is the natural and necessary reaction to
    the scaling limitations of general purpose processors, but they pose a huge
    challenge to the surrounding software ecosystem.
    As opposed to microarchitectural improvements, existing software is not able
    to automatically profit from new acccelerators designs.

    A new hardware software contract is needed instead and heterogeneous
    hardware should become a responsibility of the compiler.
    While compilers have lagged behind the developments in the hardware domain,
    research into this area can profit from experience with multi-core
    processors.

    Auto-parallelizing compilers have failed to solve the problems and have only
    had major success for specific kernels and using auto-tuning.
    Heterogeneous computing is a superset of parallel computing and so an
    approach to fully automatically optimize code is unlikely.

    At the same time, a library and DSL based approach has been successful,
    however fails to become mainstream due to adoption cost.
    What is promising therefore is a combination of hand-optimized libraries
    together with compiler automatisms.

    This requires a disruptive improvement in compiler analysis capabilites.
    The detection of higher level algorithmic structures in compilers has been
    investigated before, but established approaches cannot scale to what
    this challenge requires.
    Syntactic matching on programming languages has become
    unviable for the complexities of both modern programming languages and
    complex code bases.

    Instead, this thesis develops an entirely novel approach based on concepts
    from constraint solving, building a pragmatic methodology for detecting
    complex algorithmic structures -- computational idioms.

\pagebreak
\section{Structure of the Thesis}

    This PhD thesis is divided into six chapters.
    Following the introduction in {\bf\cref{chapter:introduction}}, a broad
    overview of the related work is given in {\bf\cref{chapter:literature}}.
    This literature survey covers the four main research areas that this work
    is placed in.
    Firstly {\em constraint programming}, which is the basis of the methodoloy
    for most of this research.
    Secondly {\em compiler analysis and auto-parallelisation}, against which the
    results in this thesis are evaluated.
    Thirdly {\em heterogeneous computing}, which motivates many of the compiler
    approaches that this research enables.
    Lastly {\em computational idioms}, a term for several overlapping concepts
    of algorithmic patterns.

    {\bf\Cref{chapter:theory}} develops the foundations of the constraint
    programming methodology that is forms the core of the later
    \cref{chapter:candl,chapter:idioms,chapter:reductions,chapter:lilac}.
    Based on a novel mathematical model of static single assignment (SSA) form
    compiler representaions, the core of the constraint programming approach
    is derived and some particular challenges are discussed in detail.

    {\bf\Cref{chapter:candl,chapter:idioms,chapter:reductions,chapter:lilac}}
    are each based on a published research article and elaborate on different
    applications of constraint programming in compilers.

    {\bf\Cref{chapter:candl}} develops a full-fledged constraint programming
    language called CAnDL, with an implementation in the LLVM compiler
    infrastructure that automatically generates compiler analysis passes from
    declarative descriptions.
    Using CAnDL, the chapter explores severeal complex compiler analysis
    challenges, concluding with a full polyhedral code analysis.
    The work of this chapter has been published as
    {\bf\citet{Ginsbach:2018:CDS:3178372.3179515}}.

    {\bf\Cref{chapter:reductions}} develops an auto-parallelising compiler for
    complex reduction and histogram computations using CAnDL-powered analysis
    functionality.
    This covers many computations that are inaccesible to established approaches
    based on data flow or polyhedral analysis and is to demonstrate the greater
    flexibility of constraint programming.
    The work of this chapter has been published as
    {\bf\citet{ginsbach2017discovery}}.

    {\bf\Cref{chapter:idioms}} extends CAnDL into the Idiom Description Language
    (IDL) and applies it to algorithmic concepts that go beyond traditional
    compiler analysis: stencil computations, complex reductions and histograms
    as well as sparse and dense linear algebra.
    The resulting automatic detection passes in LLVM enable automatic
    heterogeneous acceleration of sequential code and result in significant
    speedups on established benchmark suites.
    The work of this chapter has been published as
    {\bf\citet{Ginsbach:2018:AML:3173162.3173182}}.

    {\bf\Cref{chapter:lilac}} builds on top of the idiom descriptions for linear
    algebra that were introduced in the previous chapter and demonstrates how
    compiler analysis based on constraint programming can be made accessible to
    non-experts.
    With the novel LiLAC language, constraint programs in IDL are automatically
    generated from a high-level specification for many different sparse linear
    algebra representations.
    This powers a fully integrated compiler approach, handling everything from
    algorithmic detection, library call insertion and run time data transfer
    optimisations.
    The work of this chapter has been published as \citet{lilacpaper}.

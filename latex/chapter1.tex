    The end of Moores Law and the end of Dennard Scaling require new approaches
    in hardware.

    Heterogeneous computing platforms are the natural reaction to this.

    However, with heterogeneous hardware becoming an essential component of
    computing capabilities, thinking of it as library accelerators becomes
    wrong.

    We need a new hardware software contract instead and heterogeneous hardware
    should become a responsibility of the compiler.

    While compilers have lagged behind the developments in the hardware domain,
    we can profit from experience with multi-core processors.

    Auto-parallelizing compilers have failed to solve the problems and have only
    had major success for specific kernels and using auto-tuning.

    Heterogeneous computing is a superset of parallel computing and so an
    approach to fully automatically optimize code is unlikely.

    At the same time, a library and DSL based approach has been successful,
    however fails to become mainstream due to adoption cost.

    What is promising therefore is a combination of hand-optimized and
    -parallelized libraries together with compiler automatisms.

    This is what we consider an idiomatic approach.

\section{Structure of the Thesis}

    The thesis will be structures as follows.
    Firstly, there will be a literature survey investigating related work.

    We will then establish a theory for constraint programming on static single
    assignment compiler intermediate representation code.
    Using this theory, a constraint programming language will be presented and
    an implementation provided: The Compiler Analysis Description Language
    (CAnDL).

    Using this language, the thesis will derive computational idioms and show
    how they can be used to get heterogeneous performance.
    Lastly, we build a fully integrated system.
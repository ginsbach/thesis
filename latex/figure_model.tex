\begin{definition}{Mathematical Characterisation of SSA Functions}{ssamodel}
    The {\em SSA model} of the function $\mathcal F$ is the tuple
    \begin{align*}
        (DFG_\mathcal{F},
         CFG_\mathcal{F},
         T_\mathcal{F},
         P_\mathcal{F},
         I_\mathcal{F},
         G_\mathcal{F},
         C_\mathcal{F}),
    \end{align*}
    where
    \begin{itemize}
    \item $DFG_\mathcal{F}\subset\mathbb N^3$ and
          $CFG_\mathcal{F}\subset\mathbb N^3$ are the data flow and control
          flow graph;
    \item $T_\mathcal F\subset\textit{Types}\times\mathbb N$ is the {\it type model},
          defined by the property
          \begin{align*}
              (t,k)\in T_\mathcal F\iff\left(L_{val}\leq k\leq R_{val}\right)
                  \mathrel\land(val_k\text{ has type }t);
          \end{align*}
    \item $P_\mathcal F\subset\mathbb N$ is the {\it parameter model}, defined
          by the property
          \begin{align*}
              k\in P_\mathcal F\iff L_{par}\leq k\leq R_{par};
          \end{align*}
    \item $I_\mathcal F\subset\textit{Opcodes}\times \mathbb N$ is the
          {\it instruction model}, defined by the property
          \begin{align*}
              (c,k)\in I_\mathcal F\iff\left(L_{ins}\leq k\leq R_{ins}\right)
                  \mathrel{\land}(ins_{k-L_{ins}+1}\text{ has opcode }c);
          \end{align*}
    \item $G_\mathcal F\subset\textit{GlobalNames}\times\mathbb N$ is the
          {\it global model}, defined by the property
          \begin{align*}
              (n,k)\in G_\mathcal F\iff\left(L_{glb}\leq k\leq R_{glb}\right)
                  \mathrel{\land}(glb_{k-L_{glb}+1}\text{ has name }n);
          \end{align*}
    \item $C_\mathcal F\subset\mathbb R\times\mathbb N$ is the
          {\it constant model}, defined by the property
          \begin{align*}
              (x,k)\in C_\mathcal F\iff\left(L_{cst}\leq k\leq R_{cst}\right)
                  \mathrel{\land}(cst_{k-L_{cst}+1}\text{ has numeric value }x).
          \end{align*}
    \end{itemize}
\end{definition}

\begin{definition}{Notation for Reducing Dimensionality}{convenience}
    For a set $A$, any $a\in A$, and $S\subset A\times\mathbb N^k$ for some
    $k>0$, the following are defined:
    \begin{align*}
        heads(S)={}&\{a\in A\hspace{1.8mm}\mid(a,b_1,\dots,b_k)\in S\text{ for some }b_1,\dots,b_k\in\mathbb N\}\\
        tails(S)={}&\{b\in\mathbb N^k\mid(a,b_1,\dots,b_k)\in S\text{ for some }a\in A\}\\
        select(a,S)={}&\{b\in\mathbb N^k\mid(a,b_1,\dots,b_k)\in S\}.
    \end{align*}

    Note that the case $A=\mathbb N$ is common.
    Furthermore, for $S'\subset\mathbb N^k$, the set $rev(S')$ is defined as
    $rev(S')=\{(a,b)\mid(b,a)\in S'\}$.
    Finally, the following shorthand is used
    \begin{align*}
        \begin{aligned}
        DFG_\mathcal F^*={}&tails(DFG_\mathcal F)\\
        CFG_\mathcal F^*={}&tails(CFG_\mathcal F)
        \end{aligned}&&
        \begin{aligned}
        I_\mathcal F^*={}&tails(I_\mathcal F)\\
        C_\mathcal F^*={}&tails(C_\mathcal F)
        \end{aligned}
    \end{align*}
\end{definition}
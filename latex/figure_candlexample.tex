\centering
\begin{minipage}[t]{0.67\textwidth}
\centering
{\bf(a)} CAnDL program:
\begin{lstlisting}[language=CAnDL]
Constraint SqrtOfSquare
( opcode{sqrt_call} = call
([$\tt \land$]) {sqrt_call}.args[0] = {sqrt_fn}
([$\tt \land$]) function_name{sqrt_fn} = sqrt
([$\tt \land$]) {sqrt_call}.args[1]  = {square}
([$\tt \land$]) opcode{square} = fmul
([$\tt \land$]) {square}.args[0] = {a}
([$\tt \land$]) {square}.args[1] = {a})
End
\end{lstlisting}
\end{minipage}

\vspace{1em}
\begin{minipage}[t]{\textwidth}
\centering
\begin{minipage}[t]{\textwidth}
\centering
{\bf(b)} C program code:
\begin{lstlisting}[numbers=none,framexleftmargin=0pt,xleftmargin=0pt,language=C,basicstyle=\small\ttfamily]
 double example(double a, double b) {return sqrt(a*a) + sqrt(b*b); }
\end{lstlisting}
\end{minipage}
\begin{minipage}[t]{7.1cm}
\centering
{\bf(c)} Resulting LLVM IR:
\begin{lstlisting}[language={LLVM}]
define double @example(    
 double %0,                
 double %1) {              
 %3 = fmul double %0, %0   
 %4 = call double @sqrt(%3)
 %5 = fmul double %1, %1   
 %6 = call double @sqrt(%5)
 %7 = fadd double %4, %6   
 ret double %7 }
declare double @sqrt(double)      
\end{lstlisting}
\end{minipage}
\hfill
\begin{minipage}[t]{3.5cm}
\centering
{\bf(d)} First solution:
\begin{lstlisting}[numbers=none,framexleftmargin=0pt,xleftmargin=0pt,language=LLVM]

a = %0

square = %3
sqrt_call = %4 




sqrt_fn = @sqrt
\end{lstlisting}
\end{minipage}
\hfill
\begin{minipage}[t]{3.5cm}
\centering
{\bf(e)} Second solution:
\begin{lstlisting}[numbers=none,framexleftmargin=0pt,xleftmargin=0pt,language=LLVM]


a = %1


square = %5
sqrt_call = %6


sqrt_fn = @sqrt
\end{lstlisting}
\end{minipage}
\end{minipage}


\vspace{1em}
\begin{minipage}[t]{\textwidth}
\centering
{\bf(f)} C++ transformation code:
\begin{lstlisting}
void transform(map<string,Value*> solution, Function* abs) {
    ReplaceInstWithInst(
       dyn_cast<Instruction>(solution["sqrt_call"]),
       CallInst::Create(abs, {solution["a"]}));
}
\end{lstlisting}
\end{minipage}

\vspace{1em}
\begin{minipage}[t]{\textwidth}
\centering
{\bf(g)} Transformed LLVM IR after DCE:
\begin{lstlisting}[escapeinside={(*}{*)},language={LLVM}]
define double @example(double %0, double %1) {              
 %3 = call double @abs(double %0) 
 %4 = call double @abs(double %1)
 %5 = fadd double %3, %4   
 ret double %5 }
\end{lstlisting}
\end{minipage}

\caption{Demonstration of a simple CAnDL specification ({\bf a}) on an example C
         program ({\bf b}):
         In the generated LLVM intermediate code ({\bf c}), two instances
         ({\bf d},{\bf e}) of {\tt SqrtOfSquare} are detected.
         Applying an optimization using these results is trivial ({\bf g}) and
         results in efficient code ({\bf e}).}

    New computer processors used to be faster than the previous versions also
    because the transistors got smaller and more efficient.
    In recent years, the engineers have been unable to continue this process.
    Because the speed of processors is stagnating, companies try to improve
    other metrics.
    Specialised processors are now widespread, which cooperate with the central
    processor and complement its abilities.
    However, many software programs were not designed for this and ignore the
    specialised abilities.

    This thesis presents an approach to make existing programs use such
    specialised hardware efficiently.
    This approach was implemented in a software prototype.
    The prototype detects particular mathematical methods in other programs.
    Other researchers have already found the best ways to compute these
    methods on new hardware, and stored them in ``libraries''.
    By detecting that programs rely on the methods, the prototype can make the
    program use these efficient ``libraries'' instead.
    However, detecting the methods is hard.
    This thesis introduces new specification languages to express them.
    With precise formulations, it is possible to recognise the mathematical
    methods automatically with algorithms.

    The prototype software was evaluated by using it on standard test programs.
    Some of these programs are from NASA and imitate the calculations on their
    supercomputers.
    The prototype understood the important parts of many programs.
    This allowed executing them on specialised processors and made five of them
    more than ten times faster.

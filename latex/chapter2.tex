\section{Computational Idioms}

    The concept of computational idioms has been observed in different contexts
    and remains a rather vague concept.
    While terms such as \texttt{reduction}, \texttt{stencil} and
    \texttt{linear algebra} are commonly used, the concrete concepts can be
    surprisingly vague, although previous work has established several formal
    approaches.
    We will not try to create our own formal definitions in this chapter, but
    instead want to give an overview of the literature that sheds different
    perspectives on this topic.

    The basic observation is that software programs don't cover the space of
    possible programs evenly, instead, they tend to be structured among certain
    design principles.
    The same is true algorithmically and particularly for performance intensive
    applications.

\section{Literature Survey}
\subsection{Algorithmic Skeletons}

\subsection{Berkeley Parallel Dwarves}
\subsection{Computational Patterns}

\section{Idiom Specific Optimizations}

\subsection{Important Approaches}
\subsection{Ways of Encapsulating Expertise}


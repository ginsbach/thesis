
    The CAnDL language from the previous chapter makes the powerful constraint
    programming methods available directly in the LLVM infrastructure.
    This chapter will demonstrate a case study of an often overlooked
    computational idiom that allows automatic parallelisation of several
    important benchmark programs that were previously largely inaccesible to
    compiler analysis.

    By grouping complex reductions and histogram computations together and
    formulating them in CAnDL, a completely automatic detetion in C/C++ code is
    possible.
    Complemented with well-established parallelisation approaches, good speedups
    are achieved on several important programs.

\section{Introduction}
\section{Motivation}

\section{Approach}

\section{Experimental Setup}

\section{Results}

\section{Related Work}

\section{Conclusion}